\section{Outlook}
With the scarce data available, only fungal growth and changing \eps{} concentrations could be reliably shown. The questions which led to the fermentations and the subsequent analyses could not be answered. As the product showed poor solubility, it was resilient against analysis. Therefore, the fermentations would have to be repeated, ideally with a sample size of at least three, better five, for each fungus and fermentation time. 

But, in order for this to be successful, the analytical methods will need to be established and, more importantly, the \eps{}s purified in a way which facilitates excellent re-dissolution in water and DMSO. Some pointers into the future direction of these works can be derived directly from \vref{subsec-fun-pf-pre-sol}:
\begin{itemize}
	\item Precipitation optimization including precipitant, volume ratio and mixing unit
	\item Steps to reduce impurities prior to precipitation such as dialysis or cross-flow filtration
	\item Fractionated precipitation trials
	\item Drying process trials including freeze-drying of precipitate re-dissolved in water
\end{itemize}

On the other hand, the results of the aniline blue assay point towards a way to reliably quantify β-1,3-β-1,6-glucans possibly superior to precipitation altogether.

Removing the major roadblocks would allow to run fermentations and produce \eps{}s which give clear and reproducible results for the evaluation of the properties of \scl{} and \shz{}.

