\section{Parallel Fermentation of \rolf{} and \comm{}\label{sec-fun-pf}}
%* PMP analysis of precipitated polymers
\subsection{Controls \& Deviations}
As stated under \vref{par-met-mibi-ferm-fungi-sampling}, one sample type was \enquote{large with rheometry}. No rheometric measurements were conducted with these samples.

Due to issues with the foam sensor of fermenter 6 approximately \SIml{150} of anti-foam was pumped into the fermenter overnight triggering the overflow protocol of the fermenter. All samples of fermenter 6 were heavily contaminated by anti-foam and could not be considered for comparisons. Also, the p\ce{O2} probe of fermenter 6 was overgrown resulting in too low oxygen readings.

The large samples at \SIh{60} and \SIh{84} were skipped and instead were taken at \SIh{66} and \SIh{90}, respectively. The \SIh{108} and \SIh{132} samples were shifted to \SIh{104} and \SIh{128}, respectively.

The major deviation encountered during the fungal fermentations is the poor solubility of the products. This is discussed in more detail in \vref{subsec-fun-pf-pre-sol}. Since all fermentations were carried out only once, no statistical analyses were conducted.

\subsection{Cell Dry Masses at the End of the Fermentation}
\begin{figure}
	\begin{center}
		\includegraphics[width=7.0cm]{fig/fun-pf_cdm_600dpi.png}
		\caption[Fungal Fermentation Cell Dry Mass Courses]{Cell dry masses at the end of the fermentations of \rolf{} and \comm{}. The dry masses increase over time. The \SIh{96} value of \comm{} was lower than expected which is attributed to the high amount of anti-foam in this fermenter. Each point represents a single measurement.\label{fig-fun-pf-cdm}}
	\end{center}
\end{figure}

Cell dry masses increased with fermentation time for both fungi as can be seen in \vref{fig-fun-pf-cdm}. \comm{} showed a slight dent at \SIh{96} which is attributed to the high amount of anti-foam in this fermenter.

\subsection{\EPS{} Courses\label{sec-fun-pf-eps-courses}}
\begin{figure}
	\begin{center}
		\includegraphics[width=7.0cm]{fig/fun-pf_eps-end_600dpi.png}
		\caption[Fungal Fermentation Final \EPS{} Concentrations]{\EPS{} concentrations at the end of the fermentations of \rolf{} and \comm{}. The \eps{} concentrations increase with fermentation time, the only exception being the \SIh{144} fermentation of \comm{}. The reason for the low \eps{} production in the \SIh{144} process is unknown. The final concentrations are in good agreement with the concentrations obtained from the final samples (see \vref{tbl-fun-pf-eps-comp}). Each point represents a single measurement.\label{fig-fun-pf-eps-end}}
	\end{center}
\end{figure}
\begin{figure}
	\begin{center}
		\includegraphics[width=14cm]{fig/fun-pf_eps-courses_600dpi.png}
		\caption[Fungal Fermentation \EPS{} Courses]{\EPS{} concentration courses of the fermentations of \rolf{} and \comm{}. \EPS{} concentrations increase up to \SIh{24} and drop to approximately one tenth of the previous value. This is most likely an artifact caused by using a different sample purification protocol: all samples up to and including the \SIh{24} sample were used directly, while all samples thereafter were 1:10 diluted with ultra-pure water. It is assumed that a considerable part of the undiluted samples' precipitates are no \eps{}. The final concentrations are in good agreement with the concentrations obtained from precipitating the harvested raw broth (see \vref{tbl-fun-pf-eps-comp}). Fermenters 1, 3 and 5 were inoculated with \rolf{}, fermenters 2, 4, 6, 7 and 8 with \comm{}. Each point represents a single measurement. Abbreviations: F$n$: fermenter no. $n$ with $n \in \mathbb{N} \land 0 < n < 9$.\label{fig-fun-pf-eps-courses}}
	\end{center}
\end{figure}
The \eps{} concentrations at the end of the fermentations were determined by precipitating the remaining fermentation broth and are shown in \vref{fig-fun-pf-eps-end}. The highest concentrations were \SIgpl{3.25} after \SIh{96} and \SIgpl{2.21} after \SIh{120} for \rolf{} and \comm{}, respectively. The concentrations increased over time, except for the \SIh{144} fermentation of \comm{}. The reason for the low performance of this fermenter is unclear. Compared to fermentations of \mo{Sclerotium rolfsii} ATCC 201126 and \mo{Sclerotium glucanicum} NRRL 3006 reported by \textcite{Farina1998} and \textcite{Taurhesia1994a}, respectively, the overall \eps{} concentrations achieved were low: \SIgpl{20.6} after \SIh{72} and \SIgpl{8} after \SIh{96}, respectively, vs. \SIgpl{1.9} after \SIh{72} and \SIgpl{3.4} after \SIh{96}. While the real reason(s) must remain speculative, it is unlikely that these striking differences were caused by random events. Possible reasons include mixing issues, slow adaptation to the fermentation conditions, too high or low nutrient levels including \ce{O2}, degradation, fungal growth on the fermenter walls, the stirrer shaft and foam-breakers which slid down and general hardware-related issues such as miscalibrated instruments. The last explanation is deemed unlikely as neither operators prior nor operators after the fungal fermentations noticed any such issues.

The courses of the \eps{} concentrations over the whole fermentations are depicted in \vref{fig-fun-pf-eps-courses}. The sharp drop after the \SIh{24} sample was caused by using another sample purification method: the raw broth was first diluted 1:10 with ultra-pure water. Precipitate purity was never analysed\footnote{Since only some milliliters of the fermentation broth were sampled, the absolute precipitate masses were too low for most analyses: at most \SImg{20.5}, arithmetic mean: \SImg{2.3}, median: \SImg{1.7}.}, but it is assumed that most of the precipitate of the undiluted samples did not constitute \eps{}. High residual \eps{} in the biomass seems unlikely, as the biomass of fermenter 8 was washed and the precipitate of the supernatant amounted to only \SIpct{4.8} of the total precipitate mass of that fermenter (data not shown). One possible explanation for the apparent drop after starting 1:10 dilutions is that all components get diluted 1:10 and that might be enough to keep these components in solution at approximately \SIpct{65} isopropanol.

On the other hand, the concentrations obtained via the last samples were generally higher than the concentrations obtained via the precipitation of the whole fermentation broth at the end of the fermentation. Given the aforementioned drop after applying a 1:10 dilution, the only 1:3 diluted fermentation broth should have given considerably higher concentrations than the last samples. An adequate explanation is missing. 

These shortcomings combined point to the necessity to conduct pre-tests to find reliable protocols for measuring the \eps{} concentration of samples and at the end of the fermentation. Work on one such method spawned a publication \cite{Koenig2017} and is outlined in detail in \vref{subsec-fun-pf-sirofluor}. For future fermentations, such analytical methods should be used in any case.

Nonetheless, the final samples are in good agreement with concentrations obtained through precipitation of the complete fermentation broth at the end of the fermentation (see \vref{tbl-fun-pf-eps-comp}). 

\subsection{Precipitate Solubility\label{subsec-fun-pf-pre-sol}}
The solubility of the precipitates was of utmost importance for all further analyses, because all of them required the \eps{}s to be well-dissolved in the respective solvents. Over the course of the development of the analytical methods after the fermentations had been conducted, it became apparent that none of the precipitates would dissolve completely. Some samples gave a clearer solution, others even gave large undissolving gel clumps.

In order to facilitate better dissolution \SIml{20} of the solution was prepared in a \SIml{50} tube and mixed with 50 glass pearls (diameter: \SImm{4.0}). The tube was shaken at \SIrpm{250} and \SIdC{60} for one night to up to four days. The idea was to use the principle behind ball mills: the \eps{} flakes start to dissolve immediately forming a swollen outer layer which shields the inner undissolved \eps{} from the solvent. By shearing off that outer layer the solvent gains access to the still undissolved \eps{} on the inside. Visual inspection revealed that the dissolution was improved by this method, but still far from perfect. The increasing viscosity is most likely to be responsible for slowing down the glass pearls to ineffective speeds.

In the end, any dissolution enhancing method tackles the wrong problem: the \eps{} should not be poorly dissolvable to start with. Therefore, efforts should be directed towards the production of easily dissolvable \eps{}s instead. That would mean to conduct pre-tests to answer the following questions, among others:
\begin{itemize}
	\item Are the precipitation parameters---precipitant, volume ratio, mixing unit---optimal?
	\item Do other substances co-precipitate with the \eps{}? If so, which and how could they be separated prior to precipitation? Dialysis? Cross-flow filtration? Enzymatic treatments?
	\item Do co-precipitates influence the redissolution?
	\item Do longer polymer chains precipitate before shorter ones? If so, would a fractionated precipitation process be sensible and feasible? Could one obtain first molar mass information that way?
	\item How does the polymer concentration influence the precipitation?
	\item What does a good precipitate appear like? Small and \enquote{dry} flakes or long and \enquote{wet} threads?
	\item Does a redissolution of the gathered precipitate in ultra-pure water followed by another precipitation increase the purity? What about the overall yield?
	\item How does the drying process influence redissolution behaviour? Would freeze-drying of highly concentrated redissolved solutions yield better redissolvable polymer?
	\item How can the polymer be redissolved in a homogeneous manner? How can the homogeneity be assessed?
\end{itemize}

The analyses in the upcoming sections were conducted without answering these questions first, because the product had already been produced and \textit{only} needed to be analysed. Therefore, the results are to be taken with great care.

\subsection{Dynamic Viscosity and Thixotropy}
\nomenclature[formula_e]{$\eta$}{dynamic viscosity}
\begin{figure}
	\begin{center}
		\includegraphics[width=14cm]{fig/fun-pf_visc_600dpi.png}
		\caption[Dynamic Viscosities of Fungal Fermentation Polymers]{Dynamic viscosities at \SIps{1} and \SIps{1000} of \scl{} and \shz{} harvested at different times. Fermenters 1, 3 and 5 were inoculated with \rolf{} for \SIh{48}, \SIh{72} and \SIh{96}, respectively, fermenters 4, 6, 7 and 8 with \comm{} for \SIh{72}, \SIh{96}, \SIh{120} and \SIh{144}, respectively. Fermenter 2 is not listed as only insufficient amounts of \eps{} were produced. The \eps{} of fermenter 6 was most likely highly contaminated with anti-foam which resulted in a relatively low viscosity. Bars depict the arithmetic mean of three measurements except for fermenter 7. For fermenter 7, the sample size was two. Error bars depict the standard deviation from the arithmetic mean. Abbreviations: F$n$: fermenter no. $n$ with $n \in \mathbb{N} \land 0 < n < 9$; $\eta$: dynamic viscosity.\label{fig-fun-pf-visc}}
	\end{center}
\end{figure}
\begin{figure}
	\begin{center}
		\includegraphics[width=14cm]{fig/fun-pf_thix_600dpi.png}
		\caption[Thixotropic Behaviour of Fungal Fermentation Polymers]{Thixotropic behaviour of fungal fermentation polymers. Time to regain the initial dynamic viscosity after shearing at \SIps{100} for \SImin{2}. Fermenters 1, 3 and 5 were inoculated with \rolf{} for \SIh{48}, \SIh{72} and \SIh{96}, respectively, fermenters 4, 6, 7 and 8 with \comm{} for \SIh{72}, \SIh{96}, \SIh{120} and \SIh{144}, respectively. Fermenter 2 is not listed as only insufficient amounts of \eps{} were produced. The \eps{} of fermenter 6 was most likely highly contaminated with anti-foam which resulted in a relatively low viscosity and might have affected viscosity regain as well. The time resolution was \SIs{0.5} and therefore, the \SIpct{10} and \SIpct{25} times are all the same. The x-axis uses logarithmic scale. Bars depict the arithmetic mean of three measurements except for fermenter 1. For fermenter 1, the sample size was two. Error bars depict the standard deviation from the arithmetic mean. Abbreviations: F$n$: fermenter no. $n$ with $n \in \mathbb{N} \land 0 < n < 9$.\label{fig-fun-pf-thix}}
	\end{center}
\end{figure}

The total amount of \eps{} from the \SIh{48} fermenter 2 (\comm{}) was insufficient for rheological characterization. Therefore, data on the \eps{} of that fermenter are missing. Dynamic viscosity data are given in \vref{fig-fun-pf-visc}, thixotropy data in \vref{fig-fun-pf-thix}.

\paragraph{Dynamic Viscosity}
The general trend of measurements at \SIps{1} does not deviate from \SIps{1000}. There seemed to be no correlation between fermentation time and dynamic viscosity for \rolf{}, but---apart from the contaminated fermenter 6---the viscosity of \shz{} seems to increase until it reaches a plateau after \SIh{120} at around \SIPas{2.5}.

\paragraph{Thixotropy}
Due to the low time resolution of only \SIs{0.5} the values for \SIpct{10} and \SIpct{25} viscosity regain are all equal. After \SIh{96}, \scl{} appears to regain viscosity in less than a tenth of the time it took the \SIh{48} and \SIh{72} products. There does not seem to be a strong correlation between fermentation time and viscosity regain for \shz{}\footnote{The \eps{} of fermenter 6 was most likely highly contaminated with anti-foam. Therefore, rheologic properties might have been affected and the data was not be compared to the uncontaminated fermenters.}: \SIpct{90} of the dynamic viscosity was regained after \SIrange{2.5}{4.3}{\second}.

\subsection{Molar Mass Determination}
\begin{figure}
	\begin{center}
		\subfloat[Actigum Cs 11 in aqueous solution separated on Suprema columns.]{
				\label{fig-fun-pf-sec-malls-suprema}%
				\includegraphics[width=0.9\textwidth]{fig/fun-pf_scl_suprema_600dpi.png}
		}

		\subfloat[Actigum Cs 11 in DMSO separated on a TSKgel column.]{
				\label{fig-fun-pf-sec-malls-tskgel}%
				\includegraphics[width=0.9\textwidth]{fig/fun-pf_scl_tskgel_600dpi.png}
		}
	\caption[Reference \SCL{} Separations in Water and DMSO]{Reference \scl{} separations in water and DMSO. The reference \scl{} Actigum Cs 11 was dissolved in two different solvents and separated using two different columns. In \subref{fig-fun-pf-sec-malls-suprema}, the \scl{} was dissolved in \SIM{0.1} \ce{LiNO3} and separated on three Suprema columns. In \subref{fig-fun-pf-sec-malls-tskgel}, the \scl{} was dissolved in DMSO and separated on one TSKgel column. The steep increase of the aqueous sample is interpreted as being in the exclusion volume. Dark curves: concentration (left-most y-axis); light curves: molar mass in \si{\gram\per\mol} (second y-axis from left).\label{fig-fun-pf-sec-malls}}
	\end{center}
\end{figure}
The analytical method was first tested with a reference \scl{} (Actigum Cs 11) and it was found that in aqueous systems the samples would elute in the exclusion volume meaning that the sample molecules were too big to enter the cavities of the chromatographic media. This is in line with published research and attributed to the stable triple helices that native β-1,3-glucans form in aqueous solution \cite{Yanaki1981, Bluhm1982, Yanaki1983a, Sato1983, Farina2001, PatentUS_20040265977}. Tests in DMSO with a different column showed reproducible symmetrical peaks. For a comparison, see \vref{fig-fun-pf-sec-malls}. The molar mass distributions of the \eps{}s could not be analysed due to very low to no solubility of the precipitates in DMSO. Therefore, no data on the molar mass distributions is available.

\subsection{Periodate Test}
\begin{figure}
	\begin{center}
		\includegraphics[width=14cm]{fig/fun-pf_periodate_600dpi.png}
		\caption[Periodate Test: Ratios of Periodate Consumption to \FORA{} Formation]{Periodate test: ratios of periodate consumption to \fora{} formation. Molar ratios of the periodate consumed and the \fora{} formed after defined reaction times of the \eps{}s and periodate. Fermenters 1, 3 and 5 were inoculated with \rolf{} and fermenters 4, 7 and 8 with \comm{}. Fermenter 2 is not listed as only insufficient amounts of \eps{} were produced, fermenter 6 is missing, because of the excessive amounts of anti-foam which were pumped into the fermenter and subsequently contaminated the precipitate. Abbreviations: F$n$: fermenter no. $n$ with $n \in \mathbb{N} \land 0 < n < 9$.\label{fig-fun-pf-periodate}}
	\end{center}
\end{figure}
The periodate test was used to assess whether the polymers contained β-1,6-linked \glc{} units and whether the content differs over time or between polymers. The molar ratios of periodate consumed to \fora{} formed of all fermenters except two and six are shown in \vref{fig-fun-pf-periodate}. Under the assumptions that all products were of the same purity and that the polymer concentrations used were equal, there are no considerable differences between the different polymers and they all appear to gather around the theoretical value of \num{2.0} \cite{BuchRobyt1998}. The only exception is the polymer of fermenter 4, a \shz{}, which consumed all the available periodate and produced the least amount of \fora{}. The reason for that behaviour is not known and the author welcomes any help for finding an adequate explanation.

Similarly, periodate consumption (\vref{tbl-fun-pf-periodate}) or \fora{} formation (\vref{tbl-fun-pf-formic-acid}) alone did not give any clear indication that the \scl{} and \shz{} are different or that the polymers harvested at different times are different.

\subsection{Metabolite Analysis}
Trials to quantify \textsc{l}-malic acid, succinic acid, citric acid, fumaric acid, glyoxalic acid, itaconic acid and oxalic acid on a Rezex ROA-Organic Acid H+ (\SIpct{8}) column were unsuccessful as several components eluted at the same time: citric acid and succinic acid; succinic acid (second peak), glyoxalic acid and \glc{}; itaconic acid and fumaric acid. \textsc{l}-malic acid and oxalic acid were the only two metabolites which did not share elution time with other tested substances. No other method was available and the results were seen as secondary, which is why the metabolite analyses were not pursued any further.

\subsection{Aniline Blue Assay for the Quantitative Determination of β-1,3-β-1,6-Glucans\label{subsec-fun-pf-sirofluor}}
As a consequence of the difficulties encountered during the determination of the \eps{} concentration from fermentation broths (see \vref{sec-fun-pf-eps-courses}), a precipitation-free method was needed. This subsection deals with what eventually led to the article \enquote{Quantitative assay of β-(1,3)–β-(1,6)–glucans from fermentation broth using aniline blue} \cite{Koenig2017} and serves to give an outline of the assay's background, its development, and use. Finally, the supplemental material contains a complete protocol for the quantitative assay including a qualitative variant (see \vref{supp-sirofluor}). Generally, data in this section are not shown; the reader is referred to the publication for details.

\subsubsection{Fluorescence Properties of Aniline Blue}
The search for a specific reaction of β-1,3-β-1,6-glucans led to several different assays \cite{Ko2004, Shedletzky1997, Nitschke2011, Semedo2015a, Semedo2015b}, but most covered only β-1,3-glucans, not β-1,3-β-1,6-glucans. \textcite{Nitschke2011} reported an assay for β-1,3-β-1,6-glucans based on Congo red. Since the assays of \textcite{Ko2004} and \textcite{Shedletzky1997} used aniline blue, aniline blue was given precedence over Congo red.

As early as 1949 \textcite{Arens1949} described that aniline blue showed fluorescence in conjunction with callose. Callose is a β-1,3-glucan present in plants, but \textcite{Faulkner1973} found that the specificity of aniline blue was not limited to callose. Laminarin and pachyman, also β-1,3-glucans, and cellulose, a β-1,4-glucan, also exhibited fluorescence.

\textcite{Smith1978} were the first to take a closer look at the fluorescence properties of aniline blue and substantiated the findings of \textcite{Faulkner1973} regarding the specificity. They also found that commercial aniline blue preparations contained considerably varying fluorophore contents. Finally, \textcite{Evans1982} published the complete structure and further fluorescence properties of the fluorophore. The fluorophore was later named \enquote{Sirofluor} and its interactions with different polymers were reported by \textcite{Evans1984}. \SCL{} and \shz{} were among the polymers tested and both exhibited strong fluorescence with Sirofluor.

The high-throughput assay for β-1,3-glucan synthases by \textcite{Shedletzky1997} and the assay for β-1,3-glucan in foodstuffs by \textcite{Ko2004} were used as the basis for further experiments. Therefore, aniline blue and not pure Sirofluor was used.

\subsubsection{Optimization of the Aniline Blue Fluorescence}
While the first test was only a proof of concept, the fluorescence intensity was already dependent on the \scl{} concentration. Subsequent optimization steps included finding optima for the aniline blue concentration, the excitation and emission wavelengths, the buffer pH and additives, incubation times, and upper and lower bounds for the calibration curve.

The first test run with actual samples from fungal fermentations revealed a major flaw in the assay: the sample-to-reagent ratio was 2:1 allowing the sample matrix to have a strong influence on the assay conditions. While the rather pure reference \scl{} Actigum Cs 11 could be detected reliably, readings from actual samples gave only unusable results.

\subsubsection{Final Aniline Blue Assay}
After further optimization work, the final assay \cite{Koenig2017} had the following properties:
\begin{itemize}
	\item Calibration range\footnote{The calibration curve was constructed using the \scl{} Actigum CS 11 from Cargill.}: \SImgpl{30} to \SIgpl{6} with $R^2 > \SIpct{99.8}$
	\item Sample-to-reagent ratio: 1:9
	\item Excitation wavelength: \SInm{405}
	\item Emission wavelength: \SInm{495}
	\item Robustness\footnote{Concentrations denote the maximum concentration of the respective component in the sample without a statistically significant influence on the fluorescence intensity. Results were seen as statistically significant, when $p < 0.05$. $p$ was calculated from the two-tailed Student's \textit{t}-test for independent samples of references and samples.}
		\begin{itemize}
			\item \GLC{}: \SIgpl{50}
			\item Oxalic acid: \SIgpl{22.5}
			\item Potassium chloride: \SIgpl{13.3}
			\item Bovine serum albumine: \SIgpl{0.667}
		\end{itemize}
	\item Reagent composition: \SImM{183} glycine, \SImM{229} \ce{NaOH}, \SImM{130} \ce{HCl}, \SImgpl{618} aniline blue in ultra-pure water. Final pH $\approx$ 9.9.
\end{itemize}

The calibration range should be highlighted here as it is superior by at least one order of magnitude to the ones reported by \textcite{Nitschke2011} and \textcite{Ko2004}.

\subsubsection{Application of the Aniline Blue Assay for Fermentation Broth Samples}
The last and most important step for the future use of the aniline blue assay was the quantification of samples drawn from cultures of \rolf{} and \comm{}. The fungi were cultivated for two days to five days and the concentrations calculated from precipitations compared to concentrations from the aniline blue assay conducted at different stages of the purification process: raw broth, blended broth, neutralized broth, diluted broth, heat-inactivated broth, and supernatant after the final centrifugation before the precipitation.

For both fungi, suitable correction factors\footnote{The correction factor is calculated by dividing the concentration from the precipitation by the concentration from the aniline blue assay.} were found and allowed to achieve values with the aniline blue assay mimicking precipitation results. For \rolf{}, the correction factor was 2.46 using the aniline blue concentration determined with the supernatant after the final centrifugation. For \comm{}, the correction factor was 3.83 using the aniline blue concentration determined with the blended broth.

