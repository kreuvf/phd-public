\chapter[\SCL{} and \SHZ{} Production]{Fermentative Production of \SCL{} and \SHZ{}\label{chap-fun-pf}}
\SHZ{} and \scl{} are thought to be identical as outlined in the introduction (see \vref{subsubsec-intro-eps-commercial-shz}). On the other hand, %years of experience with these fungal polymers casted some doubts on this conclusion. Furthermore, 
a closer look at the relevant literature revealed that the basis for claiming chemical identity usually are analyses of carefully selected fractions of the polymers \cite{Norisuye1980, Yanaki1980, Kashiwagi1981, Yanaki1981, Sato1983, Yanaki1983a, Yanaki1983b, Yanaki1985}. Therefore, it was hypothesized that \shz{} and \scl{} are not identical. In order to gather evidence supporting this hypothesis the following plan was adopted:% native: Stokke1992 (nur scl, nur pH/Ionenstärke)
\begin{itemize}
	\item Fermentatively produce both polymers using the respective fungi in one parallel fermentation.
	\item Analyse the fungal \eps{}s (precipitates of supernatants) with respect to parameters used to claim identity: molar mass distribution via SEC-MALLS, dynamic viscosity and thixotropy, solubility, frequency of β-1,6-linked \glc{}.
	\item If present, resolve differences in the polymer over the course of the fermentation.
	\item If present, resolve differences between the polymers at the same fermentation duration or different fermentation durations.
\end{itemize}
The results were disheartening and are presented together with the discussion in \vref{sec-fun-pf}. The discussion focuses on the methods employed and how the experiments could have been conducted better. The author sees the publication and discussion of \enquote{negative} results as an integral part of the scientific method, one that presently is woefully neglected by too many journals and tossed aside by scientists as a consequence of the dominant publish or perish \enquote{culture}. Hopefully, the following track of failures will aid others in getting things right at first go.

