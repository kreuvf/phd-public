\subsection{Microorganisms}
%\fxnote{Strain EPS1.E4/Xyl1.H8 is very likely not Sphingomonas (anymore), see EPS-Treff 2013-12.}
\nomenclature[latabbr_EPS1]{EPS1}{plate 1 with \eps{} producers}
\nomenclature[latabbr_EPS2]{EPS2}{plate 2 with \eps{} producers}
The plates \enquote{EPS1} and \enquote{EPS2} were constructed from the \eps{} producer collection at the Chair of Chemistry of Biogenic Resources by a co-worker. The bacteria were isolated from diverse habitats and screened for their ability to produce \eps{}s. The genera and species designations are given in \vref{tbl-mat-eps1-layout,tbl-mat-eps2-layout}.

\nomenclature[mo_Agr]{Agr}{\mo{Agrobacterium}}
\nomenclature[mo_Anc]{Anc}{\mo{Ancylobacter}}
\nomenclature[mo_Art]{Art}{\mo{Arthrobacter}}
\nomenclature[mo_Bac]{Bac}{\mo{Bacillus}}
\nomenclature[mo_BeI]{BeI}{\mo{Beijerinckia indica}}
\nomenclature[mo_BM\textasciitilde{}]{BM\textasciitilde{}}{similar to \mo{Beijerinckia mobilis}}
\nomenclature[mo_Bre]{Bre}{\mo{Brevundimonas}}
\nomenclature[mo_Br\textasciitilde{}]{Br\textasciitilde{}}{similar to \mo{Burkholderia}}
\nomenclature[mo_Bur]{Bur}{\mo{Burkholderia}}
\nomenclature[mo_Cau]{Cau}{\mo{Caulobacter}}
\nomenclature[mo_Cel]{Cel}{\mo{Cellulosimicrobium}}
\nomenclature[mo_Cur]{Cur}{\mo{Curtobacterium}}
\nomenclature[mo_Dye]{Dye}{\mo{Dyella}}
\nomenclature[mo_Er\textasciitilde{}]{Er\textasciitilde{}}{similar to \mo{Erwinia}}
\nomenclature[mo_Glu]{Glu}{\mo{Gluconacetobacter}}
\nomenclature[mo_Her]{Her}{\mo{Herbaspirillum}}
\nomenclature[mo_KaA]{KaA}{associated with \mo{Kaistobacter}}
\nomenclature[mo_Koz]{Koz}{\mo{Kozakia}}
\nomenclature[mo_Noc]{Noc}{\mo{Nocardiopsis}}
\nomenclature[mo_Pae]{Pae}{\mo{Paenibacillus}}
\nomenclature[mo_Par]{Par}{\mo{Paracoccus}}
\nomenclature[mo_P/R]{P/R}{\mo{Paracoccus}/\mo{Rhodobacter}}
\nomenclature[mo_Pse]{Pse}{\mo{Pseudomonas}}
\nomenclature[mo_Rah]{Rah}{\mo{Rahnella}}
\nomenclature[mo_Rao]{Rao}{\mo{Raoultella}}
\nomenclature[mo_Rhi]{Rhi}{\mo{Rhizobium}}
\nomenclature[mo_SbC]{SbC}{close to \mo{Sphingobacterium}}
\nomenclature[mo_She]{She}{\mo{Shewanella}}
\nomenclature[mo_Sin]{Sin}{\mo{Sinorhizobium}}
\nomenclature[mo_Sph]{Sph}{\mo{Sphingomonas}}
\nomenclature[mo_Sp\textasciitilde{}]{Sp\textasciitilde{}}{similar to \mo{Sphingomonas}}
\nomenclature[mo_Xan]{Xan}{\mo{Xanthomonas}}
\nomenclature[mo_μbA]{μbA}{associated with \mo{Microbacterium}}
\nomenclature[mo_μba]{μba}{\mo{Microbacterium}}
\nomenclature[mo_μb\textasciitilde{}]{μb\textasciitilde{}}{similar to \mo{Microbacterium}}
\nomenclature[mo_μco]{μco}{\mo{Micrococcus}}
\nomenclature[mo_ϱba]{ϱba}{\mo{Rhodobacter}}
\nomenclature[mo_ϱco]{ϱco}{\mo{Rhodococcus}}
\nomenclature[mo_ϱd\textasciitilde{}]{ϱd\textasciitilde{}}{similar to \mo{Rhodanobacter}}
\begin{table}
	\centering
	\captionof{table}[Plate Layout of EPS1]{The plate layout of the plate \enquote{EPS1}. Wells A1 and E12 were left empty on purpose. Abbreviations: Agr: \mo{Agrobacterium}; Anc: \mo{Ancylobacter}; Art: \mo{Arthrobacter}; Bac: \mo{Bacillus}; Cur: \mo{Curtobacterium}; Dye: \mo{Dyella}; Er\textasciitilde{}: similar to \mo{Erwinia}; Her: \mo{Herbaspirillum}; KaA: associated with \mo{Kaistobacter}; Koz: \mo{Kozakia}; P/R: \mo{Paracoccus}/\mo{Rhodobacter}; Pae: \mo{Paenibacillus}; Par: \mo{Paracoccus}; Pse: \mo{Pseudomonas}; Rah: \mo{Rahnella}; Rao: \mo{Raoultella}; SbC: close to \mo{Sphingobacterium}; She: \mo{Shewanella}; Sin: \mo{Sinorhizobium}; Sp\textasciitilde{}: similar to \mo{Sphingomonas}; Sph: \mo{Sphingomonas}; Xan: \mo{Xanthomonas}; μb\textasciitilde{}: similar to \mo{Microbacterium}; μbA: associated with \mo{Microbacterium}; μba: \mo{Microbacterium}; μco: \mo{Micrococcus}; ϱba: \mo{Rhodobacter}; ϱco: \mo{Rhodococcus}.\label{tbl-mat-eps1-layout}}
	\begin{tabular}{l*{12}c}
		\toprule
		 & 1 & 2 & 3 & 4 & 5 & 6 & 7 & 8 & 9 & 10 & 11 & 12 \\
		\hline
		\TablesafeInputIfFileExists{data/m-and-m/eps1_layout.tex}{}{\fxfatal{File not found: data/m-and-m/eps1_layout.tex}}
		\bottomrule
	\end{tabular}
\end{table}

\begin{table}
	\centering
	\captionof{table}[Plate Layout of EPS2]{The plate layout of the plate \enquote{EPS2}. Abbreviations: Agr: \mo{Agrobacterium}; Anc: \mo{Ancylobacter}; Art: \mo{Arthrobacter}; Bac: \mo{Bacillus}; BeI: \mo{Beijerinckia indica}; BM\textasciitilde{}: similar to \mo{Beijerinckia mobilis}; Br\textasciitilde{}: similar to \mo{Burkholderia}; Bre: \mo{Brevundimonas}; Bur: \mo{Burkholderia}; Cau: \mo{Caulobacter}; Cel: \mo{Cellulosimicrobium}; Glu: \mo{Gluconacetobacter}; Her: \mo{Herbaspirillum}; Noc: \mo{Nocardiopsis}; Pae: \mo{Paenibacillus}; Pse: \mo{Pseudomonas}; Rah: \mo{Rahnella}; Rhi: \mo{Rhizobium}; Sph: \mo{Sphingomonas}; Xan: \mo{Xanthomonas}; μb\textasciitilde{}: similar to \mo{Microbacterium}; μbA: associated with \mo{Microbacterium}; μba: \mo{Microbacterium}; ϱco: \mo{Rhodococcus}; ϱd\textasciitilde{}: similar to \mo{Rhodanobacter}.\label{tbl-mat-eps2-layout}}
	\begin{tabular}{l*{12}c}
		\toprule
		 & 1 & 2 & 3 & 4 & 5 & 6 & 7 & 8 & 9 & 10 & 11 & 12 \\
		\hline
		\TablesafeInputIfFileExists{data/m-and-m/eps2_layout.tex}{}{\fxfatal{File not found: data/m-and-m/eps2_layout.tex}}
		\bottomrule
	\end{tabular}
\end{table}

