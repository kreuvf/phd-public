\subsection{Determination of Attenuance at \SInm{600}}
The term \enquote{attenuance} is recommended by IUPAC. It is reserved for the quantity which takes into account the effects of absorbance, scattering and luminescence. The former name was \enquote{extinction}. The symbol for attenuance is \enquote{$D$} \cite{IUPACgoldbook}. In short, it is the same value which is colloquially referred to as \enquote{optical density} ($OD_{\lambda}$) in microbiology labs.

\subsubsection{Determination of Linear Range}
Attenuance measurements exhibit a limited linear range and undiluted samples easily exceed this range. Therefore, undiluted samples need to be diluted prior to measurement. The linear range was determined by measuring one sample at different dilutions and calculating the undiluted value from the diluted value and the dilution factor. The linear range is marked by a stable undiluted value. If no such measurements were possible or feasible, default dilution factors were used.

\subsubsection{Cuvettes}
In order to estimate the growth and cell concentration in pre-culture shake flasks and fermenters, the attenuance at \SInm{600} $D_{600}$ was measured using an Ultrospec 10 Cell Density Meter (GHE). Samples were measured in PS semi-micro cuvettes with an optical path length of \SIcm{1.0}.
The sample was drawn from the culture and diluted appropriately. The attenuance at \SInm{600} was measured against pure diluant---ultra-pure water, if not stated otherwise. A dilution is considered appropriate if the attenuance of the diluted sample is within the linear range.

\subsubsection{96-Well Plates}
% Source: SK1, p. 208
\SIul{150} of the culture was transferred into a well of a 96-well plate (F bottom, GBO). The attenuance $D_{600}$ was measured using a plate reader (Multiskan Spectrum or Varioskan, TFS).

