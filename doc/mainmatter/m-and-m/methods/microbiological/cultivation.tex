\subsection{Cultivation}
\subsubsection{Agar Plates}
\begin{figure}
	\begin{center}
		\includegraphics{fig/dilution_streaking_600dpi.png}
		\caption[Dilution Streaking of Bacterial Suspension on Agar Plates]{Dilution streaking of bacterial suspension on agar plates. Single colonies were obtained by dilution streaking of bacterial suspensions. The sterile inoculation loop was submerged in the bacterial suspension and rows a to c were streaked. The loop was flame-sterilized and rows d to f were streaked. Flame-sterilization and streaking were repeated until most of the plate was covered. Adapted from: \cite{Mack2007}.\label{fig-dil-streak}}
	\end{center}
\end{figure}

\paragraph{Bacteria}
Bacteria were inoculated by streaking a suspension using an inoculation loop. In order to obtain single colonies, a fractionating pattern as depicted in \vref{fig-dil-streak} was used. Bacteria growing in tiny and tightly packed colonies were separated by transferring colonies to \SIml{1.0} sterile \SIpct{0.9} \ce{NaCl} solution, vortexing vigorously and dilution streaking of the suspension\footnote{This technique shall be called \enquote{eppi dilution}.}. Incubation usually took place at \SIdC{30}. For the purpose of colony counting, \SIul{100} of an appropriately diluted bacterial suspension was evenly distributed on agar plates.

\paragraph{Fungi}
New cultures were started by placing a sclerotium from \longrolf{} or a small amount of deep-frozen mycelium from \longcomm{} into the centre of a YEPD agar plate and incubating it until the plate was covered completely with mycelium. Such a plate was considered as \enquote{fresh}. Using a double-bladed lancet needle, approximately \SI{1}{\centi\metre\tothe{2}} of mycelium was cut out and transferred to a new plate.

\subsubsection{Liquid Cultures}
Liquid cultures were prepared in baffled or unbaffled Erlenmeyer flasks. Generally, the flask volume chosen was at least five times, preferably ten times, the liquid volume to ensure sufficient oxygen transfer \cite{Maier2001}. Cultures were shaken on orbital shakers, usually at \SIrpm{150}, \SIdC{30} and an eccentricity $e$ of \SIcm{1.9}.

For high-throughput experiments, cultivation was carried out in 96-well plates (\SIml{2.0}; V bottom, sterile, GBO) using \SIml{1.0} culture per well and plates were incubated at \SIrpm{1000} and \SIdC{30} on a shaker with incubation hood (TiMiX 5 Control with TH15, EBG) and an eccentricity $e$ of \SImm{1.5}.

