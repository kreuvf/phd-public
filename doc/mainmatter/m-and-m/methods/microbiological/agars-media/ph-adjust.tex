\subsubsection{Sterile pH Value Adjustment}
Measuring the pH value of a sterile solution with an unsterile electrode would instantly make the solution unsterile. Therefore, whenever it was necessary to adjust the pH value of a sterile solution the following procedure was used.

The pH value was adjusted by taking a sterile aliquot, adjusting the pH in the sterile aliquot using \SIM{2} \ce{NaOH} and adding the corresponding amount of \SIM{2} \ce{NaOH} under sterile conditions to the remaining medium.
Example: Draw an aliquot of \SIml{10} from \SIml{500} medium. Adjust the pH value with \SIul{45} \SIM{2} \ce{NaOH} to the desired pH value. Adjust the pH value of the remaining medium by adding \num{49} $\cdot$ \SIul{45} = \SIul{2205} ≈ \SIml{2.2}  \SIM{2} \ce{NaOH} under sterile conditions.

