\subsubsection{Lysogeny Broth}
\nomenclature[chem_LB]{LB}{lysogeny broth}
\begin{itemize}
	\item \SIg{10.0} Casein tryptone
	\item \SIg{5.00} Yeast extract
	\item \SIg{10.0} \ce{NaCl}
\end{itemize}
This medium is abbreviated with \enquote{LB\footnote{This abbreviation is often claimed to stand for \enquote{Luria Broth}, \enquote{Lennox Broth} or \enquote{Luria-Bertani medium}. In \cite{Bertani2004} the creator of this medium, Giuseppe Bertani, explains that the abbreviation stands for \enquote{Lysogeny Broth}: \enquote{The acronym has been variously interpreted, perhaps flatteringly, but incorrectly, as Luria broth, Lennox broth, or Luria-Bertani medium. For the historical record, the abbreviation LB was intended to stand for \enquote{lysogeny broth.}}}}. Contrary to the original recipe \cite{Bertani2004}, the LB medium used in this work did not contain \glc{}. All components were autoclaved together.

