\subsubsection{Plate Preparation for Fungi}
Mycelial growth into the agar makes it nearly impossible to cleanly slice off part of the fungi without also having parts of the agar \enquote{contaminate} the sample. In order to prevent mycelial in-growth, agar plates were covered with a cellophane membrane (BRL) which allows small molecules to pass, but not mycelia.
Cellophane membranes came with a piece of paper between each sheet as a separator. A cellophane membrane together with one separator was cut out using scissors and a petri dish as a template. The cut out pieces were gathered in a glass petri dish. Once ultra-pure water was added to the petri dish and everything was soaked in the water, the lid was closed and the plate was wrapped into aluminium foil and autoclaved as a liquid.
After autoclaving, sterile cellophane membranes were removed from the petri dish using a flamed pair of tweezers. The cellophane membrane was transferred onto the surface of an agar plate such that the whole surface was covered and as much of the surface as possible was in direct contact with the membrane. After preparing the plates, they were stored at \SIdC{4}.

