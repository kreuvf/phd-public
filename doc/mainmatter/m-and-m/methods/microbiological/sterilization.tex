\subsection{Sterilization}
\paragraph{Liquids}
Liquids were autoclaved for \SImin{20} at \SIdC{121} using the programme \enquote{Flüssigkeit\footnote{German for \enquote{liquid}.} RO} of the autoclave (135 S, HPM or 135 S, TEL). When the liquids were autoclaved in closed vessels, the vessel was slightly opened to allow pressure compensation, e.~g. the cap was slightly unscrewed. Correct autoclaving was ensured by using an autoclave tape.

\paragraph{Sterile Filtration}
Sensitive material such as vitamin solutions were sterile filtered through \SIum{0.2} or \SIum{0.45} cellulose acetate syringe filters (VWR). Larger volumes (minimum: \SIl{2}) were sterile filtered using an autoclaved vacuum filtration unit and an \SIum{0.45} cellulose nitrate filter (SSB).

\paragraph{Fermenters}
DASGIP fermenters were autoclaved for \SImin{30} at \SIdC{121} using the programme \enquote{Flüssigkeit RO} of the autoclave. The \SIl{10} fermenter was sterilized by an \textit{in situ} sterilization process.

\paragraph{Plastic Items}
Dry autoclavable plastic items such as pipette tips were autoclaved for \SImin{20} at \SIdC{121} using the programme \enquote{Instrumente\footnote{German for \enquote{instruments}.} ST} of the autoclave. Correct autoclaving was ensured by using autoclave tape.

\paragraph{Glassware}
Glassware and other suitable instruments were sterilized for at least \SIh{3} at \SIdC{200} in a drying cabinet (Function Line T12, TFS). Correct autoclaving was ensured by using a baking tape.
