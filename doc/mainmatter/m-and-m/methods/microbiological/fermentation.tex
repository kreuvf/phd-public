\subsection{Fermentation\label{subsec-met-mibi-ferm}}
Fermentations were carried out in two different systems: two parallel fermenter blocks with four fermenters each were used for small-scale parallel fermentations, while a 10-litre fermenter was used for single fermentations and scale-up experiments.

\subsubsection{Parallel Fermentation of Fungi\label{subsubsec-met-mibi-ferm-fungi}}
\paragraph{Preculture Conditions}
Precultures of \longrolf{} and \longcomm{} were prepared in modified EPSmax13 medium with \SIgpl{40.0} \glc{} as carbon source. \SIml{50} medium was incubated in 250 ml Erlenmeyer flasks without baffles and were agitated by magnetic stir bars on a magnetic stirrer (Telesystem 15, TFS) inside an incubation cabinet (KBF 240, BG). Preculture parameters were \SIpct{30} agitation power, \SIdC{30}, \SIrpm{350} at low viscosity and \SIrpm{200} at higher viscosity.

Five days prior to inoculation of the fermenters with \rolf{}, six YPD plates with \SI{4}{\centi\metre\squared} of fresh mycelium were prepared and incubated for two days at \SIdC{30}. Three days prior to inoculation of the fermenters with \rolf{}, \SIml{250} modified EPSmax13 medium with \SIgpl{40.0} \glc{} as carbon source and \SImgpl{75} ampicillin in a 2-litre Erlenmeyer flask were inoculated with a quarter of an agar plate cut into tiny pieces. On the day of inoculation of the fermenters, \SIml{50} of the preculture was drawn from the flask to a syringe using a sterile tubing.
%Rolf: 2013-05-09: VK in 2-l-Kolben o. S. mit "Prismarührfisch" autoklaviert; 2013-05-10: 6 Platten mit 4cm² angeimpft; 2013-05-11: Rührfisch gegen 8 cm langen ausgetauscht; 2013-05-12: Rührfischwechsel o. Kontamination; keine Platte zu 100\% bewachsen; VK daher mit bisschen mehr als 1/4 einer ~90+\% bewachsenen Platte animpfen; 250 ml in präpariertem VK-Kolben (mit 75 mg/l Ampicillin)

Three days prior to inoculation of the fermenters with \comm{}, \SIml{300} modified EPSmax13 medium with \SIgpl{40.0} \glc{} as carbon source and \SImgpl{50} ampicillin was inoculated with \SIml{10} of preculture from a six day old preculture with \SImgpl{34} chloramphenicol. On the day of inoculation of the fermenters, \SIml{50} of the preculture was drawn from the flask to a syringe using a sterile tubing.
%Kommunist: 2013-05-06: Chloramphenicolversuch gestartet (34 mg/l CA); 2013-05-09: VK in 1-l-Kolben o. S. mit "Prismarührfisch" autoklaviert; 2013-05-10: 8 Platten mit ?cm² (vermutlich 4) überimpft; 2013-05-11: Rührfisch gegen 8 cm langen ausgetauscht; 2013-05-12: Rührfischwechsel o. Kontamination; Animpfen mit 10 ml VK mit kleinen Kugeln von 2013-05-06 (Chloramphenicolversuch); 300 ml in präpariertem VK-Kolben (mit 50 mg/l Ampicillin)

\paragraph{Fermenter Setup\label{par-met-mibi-ferm-fungi-ferm-setup}}
Each fermenter was equipped with a p\ce{O2} probe, a sampling tube, an inoculation port, an acid port, a foam probe, an anti-foam port, a base port, a septum, an air inlet with tube and sparger, an exhaust gas cooler, a pH probe and a temperature sensor. The exact layout used is given in \vref{fig-dasgip-top}. Fermenter 2 deviated from the default layout: the septum and the pH probe were swapped.

\begin{figure}
	\begin{center}
		\includegraphics{fig/dasgip_layout_top_600dpi.png}
		\caption[DASGIP Fermenter Top View]{DASGIP fermenter top view. In the default layout, DASGIP fermenters were equipped with a motor for the stirrer (M) and seven ports for a p\ce{O2} probe (1), ports for acid, sampling and inoculation (2), ports for base, anti-foam and foam probe (3), a septum (4), the air inlet with tube and sparger (5), an exhaust gas cooler (6) and a pH probe (7). A sheath for the temperature sensor and two connections for ground cables were present as well (\num{8}).\label{fig-dasgip-top}}
	\end{center}
\end{figure}

Improvised foam breakers were prepared from cable ties as described by \textcite{Riedel2011}. Sterilizing the fermenters left several cable ties loose, so they slid down on top of the stirrer. The following foam breakers were affected: the two bottommost foam breakers of the fermenters 2 and 4, and all foam breakers of the fermenters 7 and 8. Some foam breakers were too long and thus, could not rotate: all foam breakers of the fermenters 2, 7 and 8; the two bottommost foam breakers of fermenter 4; and only the bottommost foam breakers of the fermenters 3 and 5. The dimensions of a single fermenter are given in \vref{fig-dasgip-side}.

\nomenclature[formula_D]{$D$}{fermenter inner diameter}
\nomenclature[formula_Hf]{$H_f$}{fermenter total height}
\nomenclature[formula_H]{$H$}{medium height}
\nomenclature[formula_hB]{$h_B$}{distance between stirrer and fermenter bottom}
\nomenclature[formula_H']{$H^\prime$}{medium as measured from the sparger}
\nomenclature[formula_hfR]{$\Delta{}h_{fR}$}{distance between bottom foam breaker and stirrer}
\nomenclature[formula_hf]{$\Delta{}h_f$}{distance between neighbouring foam breakers}
\nomenclature[formula_dsh]{$d_{sh}$}{shaft diameter}
\nomenclature[formula_bw]{$b_w$}{stirrer blade width}
\nomenclature[formula_bh]{$b_h$}{stirrer blade height}
\nomenclature[formula_dR]{$d_R$}{stirrer diameter}
\nomenclature[formula_df]{$d_f$}{form breaker diameter}
\begin{figure}
	\begin{center}
		\includegraphics{fig/dasgip_layout_side_600dpi.png}
		\caption[Setup of DASGIP Fermenter for Fungal Fermentation (Side)]{Schematic drawings of a DASGIP 1-litre fermenter filled up to approximately \SIml{500} (upper left corner) and a six-blade Rushton impeller (lower right corner). The fermenter inner diameter $D$ was \SIcm{9.9}, the total height $H_f$ \SIcm{14.4}. The medium height $H$ was approximately \SIcm{6.7} and the medium height as measured from the sparger $H^\prime$ was approximately \SIcm{5.7}. The space between the stirrer and the fermenter bottom $h_B$ was approximately \SIcm{2.6}. The space between the bottom foam breaker and the stirrer $\Delta{}h_{fR}$ was \SIcm{7.8}. The distance of each foam breaker to neighbouring foam breakers $\Delta{}h_f$ was \SIcm{1.0}. The shaft diameter $d_{sh}$ was \SImm{7.9}. The stirrer blade width $b_w$ was \SIcm{1.2} and the stirrer blade height $b_h$ was \SIcm{1.2}. The stirrer diameter $d_R$ was \SIcm{4.5}. The foam breaker diameter $d_f$ was approximately \SIcm{4.8}. Sampling tube and probes are not depicted. In reality, foam breakers were positioned at an angle of \SI{90}{\degree} to neighbouring foam breakers.\label{fig-dasgip-side}}
	\end{center}
\end{figure}

%Fermenter inner diameter: D = ; Fermenter total height: Hf = ; Fermenter medium height: H = ; Fermenter medium height from sparger: H' = ; Space between stirrer and fermenter bottom: hB = ; Space between bottom foam breaker and stirrer: hfR = ; Space between neighbouring foam breakers: hf = 1 cm; Shaft diameter: dsh = ; Stirrer blade width: bw = ; Stirrer blade height: bh = ; Stirrer diameter: dR = ; Foam breaker diameter: df = 

%Rührerunterkante von unterem Ende der Rührerwelle: 1,5 cm; Improvisierte Schaumschläger, 2 cm lang; oberster: Unterkante 7,8 cm von Rühreroberkante entfernt; 3 Stück, jeweils 1 cm unter obersten, jeweils im 90°-Winkel zu Vorgänger

Calibrations of pH probe, p\ce{O2} probe, peristaltic pumps and off-gas analyser were carried out in accordance with manufacturers' instructions. After off-gas calibration, the value of fermenter 1 was slightly lower than that of the other fermenters.
%Fermenterlayout (außer F2): siehe dasgip_layout_top.svg f. Zuordnung; 1: pO2-Sonde, 2: Probenahmerohr, Animpfstutzen, Säureanschluss, 3: Schaumsonde, Antischaumanschluss, Baseanschluss, 4: Septum, 5: Zuluft, 6: Abluft, 7: pH-Sonde, 8: Anschluss für Temperaturfühler, Anschlüsse für Erdungskabel
%Fermenterlayout (F2): siehe dasgip_layout_top.svg f. Zuordnung; 1: pO2-Sonde, 2: Probenahmerohr, Animpfstutzen, Säureanschluss, 3: Schaumsonde, Antischaumanschluss, Baseanschluss, 4: pH-Sonde, 5: Zuluft, 6: Abluft, 7: Septum, 8: Anschluss für Temperaturfühler, Anschlüsse für Erdungskabel
%Abgaskalibrierung: F1 problematisch, zeigt niedrigeren Wert an als sollte (9,0 statt 10)
%weitere Abweichungen: F1: Abgasanalytik zeigt geringeren O2-Wert an als sie sollte; F2: Schaumschläger 2, 3 (Zählung immer von oben) heruntergerutscht, liegen auf Rührer auf; Schaumschläger 1, 2, 3 drehen nicht, da Sonden im Weg; F3: Schaumschläger 3 dreht nicht, da Sonden im Weg; F4: Schaumschläger 2+3 heruntergerutscht, liegen auf Rührer auf; Schaumschläger 2, 3 drehen nicht, da Sonden im Weg; F5: Schaumschläger 3 dreht nicht, da Sonden im Weg; F6: keine; F7: Schaumschläger 1, 2, 3 (Zählung immer von oben) heruntergerutscht, liegen auf Rührer auf; * F8: Schaumschläger 1, 2, 3 (Zählung immer von oben) heruntergerutscht, liegen auf Rührer auf

\paragraph{Fermentation Parameters}
\nomenclature[formula_vvm]{vvm}{Gas volume per minute relative to the fermenter liquid volume. Shorthand for \enquote{gas \textbf{v}olume per liquid \textbf{v}olume and \textbf{m}inute}. Standard conditions are assumed for the gas volume.}
\nomenclature[formula_lN]{\si{\litre\norm}}{litre under standard conditions}
Each fermenter contained \SIml{450} of 10:9 concentrated modified EPSmax13 medium and was inoculated with \SIml{50} of three day old preculture. Fermenters 1, 3 and 5 were inoculated with \rolf{}, fermenters 2, 4, 6, 7 and 8 were inoculated with \comm{}. After inoculation, \SIul{225} ampicillin solution were added so that the final concentration in the fermenter was \SImgpl{50}. At the start of fermentation, the initial pH value was measured by a pH electrode (405-DPAS-SC-K8S, MTG) and controlled to reach \num{6.0} with \SIM{1.0} \ce{HCl} and \SIM{1.0} \ce{NaOH}, control was disabled afterwards. Dissolved oxygen was controlled at \SIpct{20} relative to oxygen saturated medium and measured using an EasyFerm p\ce{O2} probe (HBA). Dissolved oxygen was kept constant with a cascade: first, agitation would increase from \SIrange{400}{1200}{\rpm}, then the gas flow would increase from \SIrange{0.2}{0.6}{\vvm} \footnote{\SIrange{0.2}{0.6}{\vvm} corresponds to \SIrange{6.0}{18}{\litre\norm\per\hour} in this setup.}. Temperature was maintained at \SIdC{30} and was measured using a Pt100 temperature sensor. Anti-foam B 1:10 in ultra-pure water was added automatically once foam reached a foam probe. Agitation was limited to \SIrpm{400} \SIh{43} after inoculation. Five days after inoculation, gas flow was limited to \SI{0.2}{\vvm}. Fermenters were harvested after \SIh{48} (fermenters 1 and 2), \SIh{72} (fermenters 3 and 4), \SIh{96} (fermenters 5 and 6), \SIh{120} (fermenter 7) and \SIh{144} (fermenter 8).

\paragraph{Sampling Plan\label{par-met-mibi-ferm-fungi-sampling}}
For the first \SIh{48}, samples were drawn every three hours. Three different sample types were distinguished: small, large and large with rheometry. Small samples were taken every three hours, large samples every six hours and large samples with rheometry every twentyfour hours. Sample volume was \SIrange{1.5}{2.0}{\milli\litre} for small samples and \SIrange{6.0}{7.0}{\milli\litre} for large samples. After \SIh{48}, small samples were taken every six hours, large samples every twelve hours and large samples with rheometry every twentyfour hours.

Small samples were taken for analyzing \glc{} consumption using the \glc{} assay and also certain metabolites using HPLC later. The metabolites were \textsc{l}-malic acid, succinic acid, citric acid, fumaric acid, glyoxalic acid, itaconic acid and oxalic acid. Small samples up until and including the \SIh{24} sample were not diluted and directly centrifuged for \SImin{10} at \SIG{10000} and \SIdC{20}, the supernatant transferred into a new tube and stored at \SIdC{-20}. Later, samples were diluted 1:10 with \SIml{13.5} ultra-pure water in 15 ml tubes and centrifuged for \SImin{30} at \SIG{5000} and \SIdC{20}; the supernatant was transferred into a new tube and stored at \SIdC{-20}.

Large samples up until and including the \SIh{18} sample were not diluted and directly centrifuged for \SImin{30} at \SIG{5000} and \SIdC{20} and \SIml{4.5} supernatant were precipitated in \SIml{9.0} isopropanol. Later, samples were diluted 1:10 with \SIml{45} ultra-pure water in 50 ml tubes and centrifuged for \SImin{30} at \SIG{5000} and \SIdC{20}. Three times \SIml{15} supernatant were precipitated in \SIml{30} isopropanol each.

At all times large samples with rheometry were diluted 1:10 with \SIml{45} ultra-pure water in 50 ml tubes and centrifuged for \SImin{30} at \SIG{5000} and \SIdC{20}, \SIml{5.0} supernatant transferred into a tube for later rheometry measurements and stored at \SIdC{-20}, two times \SIml{15} supernatant precipitated in \SIml{30} isopropanol. The precipitates of all precipitation duplicate and triplicate samples were combined for the determination of polymer mass.

Cell dry mass at the end of the fermentation was estimated from the pelleted mycelia after centrifugation. Pelleted mycelia were stored at \SIdC{-20} in pre-weighed \SIml{50} tubes. The samples were dried at \SIdC{60} to constant mass.

\subsubsection{Parallel Fermentation of Bacteria\label{subsubsec-met-mibi-ferm-dasgip}}
\paragraph{Preculture Conditions}
From a two day old LB agar plate of \strain{}, a single colony was used to inoculate \SIml{10} LB medium in a \SIml{25} Erlenmeyer flask without baffles. After \SIh{24} incubation at \SIdC{30}, \SIrpm{150} and an eccentricity $e$ of \SIcm{1.9}, \SIml{1.00} of this preculture was used to inoculate \SIml{100} SM with \SIgpl{24.0} \glc{} and \SIgpl{6.00} \xyl{} with \SIpct{100} peptone (SM19 P100) in a 1-litre Erlenmeyer flask with baffles. The preculture was incubated for \SIh{24} at \SIdC{30}, \SIrpm{150} and an eccentricity $e$ of \SIcm{1.9} and then used to directly inoculate the fermenters with an initial $D_{600}$ of \num{0.05}.
%* 2015-03-29: Vorvorkultur angeimpft von LB-Platte von 2015-03-27 (von SM1-P100-Platte v. 2015-03-24 aus Eppiverdünnung; ursprünglich aus Kryoplatte); 10 ml LB in 25-ml-EMKs mit crappy Alu; @30 °C, 150 rpm ab 21:30 Uhr
%* 2015-03-30: 100 ml SM19 P100 in 1-l-Schikanekolben mit 1 ml LB-Kultur von VOrtrag; VK@30 °C, 150 rpm ab 21:20 Uhr; Ersatz-VK in 100 ml LB in 250-ml-SK (wg. Platzmangel) ab 21:45 Uhr @30°C, 150 rpm
%* 2015-03-31: Animpftag :D

\paragraph{Fermenter Setup}
The fermenter setup was the same as for the fungal fermentations (see \vref{par-met-mibi-ferm-fungi-ferm-setup}) with the following differences: pH probe and p\ce{O2} probe were swapped, the base port was connected to (2) and the sampling tube was used for inoculation and flushed with sterile air afterwards instead of using a dedicated inoculation port. The numbers correspond to the numbers given in \vref{fig-dasgip-top}. Dimensions were the same as for the fungal fermentations given in \vref{fig-dasgip-side} with the following deviations: the space between the stirrer and the fermenter bottom $h_B$ was approximately \SIcm{3.6}; the space between the stirrer and the bottommost foam breaker $\Delta{}h_{fR}$ was \SIcm{4.0}; foam breakers were positioned at an angle of \SI{60}{\degree} to each other in line with stirrer disks.
%Fermentersetup EPS2.H7 (inkl. Fig.)
%* Abstand Boden zu Oberfläche (500 ml): 6,65 cm
%* Mitte des Rührer 3,6 cm von Boden weg (= 2,5 cm vom Ende der Welle bis Rührermitte)
%* Schaumbrecher: 4,65 cm über Mitte des Rührers, jeweils parallel zu einem Rührerblattpaar, 2. und 3. jeweils 1 cm weiter oben
%* pH-Sonde Umfang: 3,8 cm (d = 11,80 mm)
%* Außendurchmesser Probenahmerohr: 3,98 mm
%* Rührerblatt: Höhe: 11,83 mm, Breite: 11,90 mm
%* Rührerblatt: Durchmesser von Blatt: 29,56 mm, Höhe: 1,50 mm
%* Welle: 7,93 mm Durchmesser
%* DASGIP, Innendurchmesser: 9,93 cm, Innenhöhe bis Glasschulter: 14,4 cm
%* Temperaturfühlerrohr: Außendurchmesser: 3,91
%* Außendurchmesser pO2-Sonde: 11,93 mm
%* Abstand pO2-Sonde zu Boden: 9 mm
%* Abstand pH-Sonde zu Boden: 4 mm
%* Abstand Ende der Welle zu Boden: 1,1 cm
%* Begasungsrohr liegt fast auf Boden auf (max. 1 mm Abstand), 7 Löcher mit ~1 mm Durchmesser, äußerste Löcher voneinander 18,9 mm; mittleres liegt unter Rührermitte 

\paragraph{Fermentation Parameters}
Fermenters \numrange{1}{4} (block 1) contained \SIml{500} SM19 P100 without MOPS (SM19 P100F), fermenters \numrange{5}{8} (block 2) contained \SIml{500} SM0 P100F with \SIpct{30} lignocellulose hydrolysate equalling \SIgpl{24.0} \glc{} and \SIgpl{6.00} \xyl{}, minor amounts of organic acids and inhibitors (SMLCH P100F). Before inoculation, the pH value was set to \num{7.0} and controlled with \SIpct{42.5} phosphoric acid and \SIM{2} sodium hydroxide during the course of the fermentation. Dissolved oxygen was monitored, but not controlled, using a VisiFerm p\ce{O2} probe (HBA). Temperature was maintained at \SIdC{30}. Agitation was constant at \SIrpm{400} and aeration at \SI{0.4}{\vvm} \footnote{Corresponds to \SI{12.0}{\litre\norm\per\hour}.}. \SIml{1.0} antifoam (antifoam B, 1:10 in ultra-pure water) was added before inoculation. Additional antifoam was added manually. All fermenters were inoculated with \SIml{3.7} one day old preculture with $D_{600} = 6.8$.

\SIh{48} after inoculation, the pH control for block 1 was set to \num{6.4} and fermentations were ended after \SIh{91.5}. \SIh{87.5} after inoculation, the pH control for block 2 was set to \num{6.4} and fermentations were stopped after \SIh{134}.
% Block 1: SM19 P100F; Block 2: SMLCH P100F; Säure: 42,5\% H3PO4; Base: 2 M NaOH
% 1 ml Antischaum zu Start/DO-Kalibrierung; 30 °C, pH 7,00(5), 400 rpm, 0,4 vvm (12 lN/h); pH-regelung bei F7 auf eigentlichen pH+0,25 geregelt, da nach Autoklavieren offenbar Sonde was abbekommen hat; Schaumsonden F1, F2, F5, F7 spinnen; Antischaumzugabe daher wie an andere Fermenter; manuell machen; Animpfen: OD mit 1:20 im Ultrospec; Animpf-OD zwischen 0,1 und 0,05 --> 3,68 ml direkt hinzugegeben; Animpfzeitpunkt: 20:38 Uhr
% pH-Wert mittendrin schwachsinnigerweise (habe auf wen gehört, der sich auf unveröffentlichtes Geheimspezialwissen aus der Industrie - DER INDUSTRIE, sage ich! -- na dann muss es ja stimmen :X - berufen hat) auf 6,4 gesenkt mit dem Ziel mehr EPS zu bekommen nach 48 h.
% KEINE DO-Regelung, keine Rührerkaskade, keine Begasungsratenänderung; naiver Batchansatz ohne Bling-Bling

\paragraph{Sampling Plan}
For the first sample, approximately \SIml{2.0} was drawn from each fermenter. $D_{600}$ was determined using a 1:20 dilution in ultra-pure water. \SIul{180} sample was diluted 1:10 with ultra-pure water and centrifuged for \SImin{5} at \SIG{17000} and \SIdC{4}. \SIml{1.0} of the supernatant was transferred to a new tube for the \glc{} assay. The undiluted sample was also centrifuged for \SImin{5} at \SIG{17000} and \SIdC{4} and \SIml{1.0} of the supernatant was transferred into a new tube for inhibitor, \xyl{} and polymer analysis. Both supernatants were stored at \SIdC{-20}. 
%* 1. Probenahme: 2 ml Probe: 180 µl 1:10, 5 min @ max. g zentr., 4 °C; 1 ml Überstand f. Glc-Assay wegfrieren; unverdünnten Rest ebenfalls 5 min@max. g zentr., 4 °C; 1 ml Überstand für PMP (Xyl, Glc, HMF, Fur., Polymer über Zeit); OD gegen ddH2O (1:20) messen

From the second sample on, approximately \SIml{2.0} was drawn from each fermenter. $D_{600}$ was determined using a 1:20 dilution in ultra-pure water. The sample was centrifuged for \SImin{5} at \SIG{17000} and \SIdC{20}. \SIul{100} of the supernatant was diluted 1:10 with ultra-pure water in a new tube for the \glc{} assay. A part of the remaining supernatant was transferred to a fresh tube for inhibitor, \xyl{} and polymer analysis. Both supernatants were stored at \SIdC{-20}. Excess supernatant was discarded and the pellet was used for the determination of cell dry mass. If the sample was still turbid after centrifugation it was diluted 1:5 with ultra-pure water and centrifuged for \SImin{5} at \SIG{17000} and \SIdC{20}. \SIul{500} supernatant was diluted 1:2 with ultra-pure water in a new tube for the \glc{} assay. A part of the remaining supernatant was transferred to a fresh tube for inhibitor, \xyl{} and polymer analysis. Both supernatants were stored at \SIdC{-20}. Excess supernatant was discarded and the pellet was used for the determination of cell dry mass.
%* 2. und weitere Probenahme: 2 ml Proben, 5 min @ 17k x g, 20 °C, 100 µl Überstand 1:10 in ddH2O f. Glc-Assay, Rest f. PMP; wenn noch trüb: 1:5 verdünnen, zentrifugieren, f. Glc-Assay 1:2 verdünnen; restlichen Überstand verwerfen, Pellet f. BTM behalten

Samples were drawn after \SIlist{1.0;14;25.2;37.5;48;62.2;72.2;88}{\hour} for both blocks; one final sample before harvest for block 1 after \SIh{91.5} and further samples for block 2 after \SIlist{96.3;112;120;134}{\hour}.

\paragraph{Polymer Purification\label{par-met-mibi-ferm-dasgip-puri}}
The fermentation broths of each block were collected and centrifuged for \SImin{10} at \SIG{17000} and \SIdC{20}. The supernatants were cross-flow filtered through \SIum{0.45} membranes and samples for precipitation drawn from the feed, the retentate and the permeate.

\subsubsection{7-litre Fermentation\label{subsubsec-met-mibi-ferm-10l}}
\paragraph{Preculture Conditions}
Five days prior to inoculation, two SM1 P100 plates were inoculated from a cryo stock of \strain{} from 2015-02-26 and incubated at \SIdC{30}. Two days prior to inoculation, two times \SIml{10} LB medium in 25 ml Erlenmeyer flasks without baffles were inoculated with a single colony from the plates prepared two days earlier and incubated at \SIdC{30}, \SIrpm{150} and an eccentricity $e$ of \SIcm{1.9}. One day prior to inoculation, two baffled Erlenmeyer flasks with \SIml{100} of SM19 P100 were inoculated with \SIml{5} of a one day old LB culture giving $D_{600}$ of the preculture of \num{0.1}.

\paragraph{Fermenter Setup}
The fermenter (BIOSTAT Cplus, SSB) was equipped with a pH probe, p\ce{O2} probe, temperature probe, foam probe, ports for base, lignocellulose hydrolysate, medium concentrate and anti-foam, air inlet with tube and sparger, exhaust gas cooler, exhaust gas sensors for \ce{CO2} and \ce{O2} (BCP-CO2 and BCP-O2, respectively, BGS) and three baffles. The fermenter had a maximum working volume of \SIl{10}, a total volume of \SIl{14} and was equipped with one six-bladed Rushton type impeller. Stirrer blade width $b_w$ was \SImm{35.0}, stirrer blade height $b_h$ was \SImm{23.0} and the stirrer diameter $d_R$ was \SIcm{11.4}. Three improvised foam breakers were attached to the shaft, the space between the bottommost foam breaker and the stirrer $\Delta{}h_{fR}$ was \SIcm{20}, the space between each foam breaker to neighbouring foam breakers $\Delta{}h_f$ was \SIcm{2.0} and foam breakers were positioned at an angle of \SI{60}{\degree} to each other in line with the stirrer disks. The medium height as measured from the sparger $H'$ was \SIcm{24}, the distance between the stirrer and the medium surface was \SIcm{12}. All pumps were calibrated prior to fermentation. The pump \enquote{ACID} had a maximum feed rate of \SI{16.3}{\milli\litre\per\minute}, the pumps \enquote{BASE}, \enquote{AFOAM} and \enquote{LEVEL} had maximum feed rates of \SI{15.0}{\milli\litre\per\minute}. MFCS/DA 3.0 (SSS) was used for data acquisition.

%Fermentersetup
%1. Anlauf:
%* Rührer: Blattbreite: 35,0 mm, Blatthöhe: 23,0 mm; Rührerbreite: 11,4 cm
%* Schaumbrecher: 2 cm Abstand zueinander, 7 cm lang von Wellenmittellinie aus; unterster 20 cm über Mitte des Rührers
%* Entfernung Wasseroberfläche (7 l) zu Boden des Begasungsrings: 24 cm --> Rührermitte 12 cm von Boden des Begasungsrings entfernt = 8,4 cm vom Wellenende entfernt
%* defekte O2-Analytik (Dank an DS!)
%* CO2-Analytik
%* Säurepumpe rekalibriert: 16,3 ml/min
%* BASE-Kalibrierung, AFOAM-Kalibrierung, LEVEL-Kalibrierung wie gehabt: 15,0 ml/min

\paragraph{Fermentation Parameters}
\begin{table}
	\centering
	\caption[7-Litre Fermentation Feeding Programme of Lignocellulose Hydrolysate]{7-litre Fermentation feeding programme of lignocellulose hydrolysate. Lignocellulose hydrolysate was fed into the fermenter slowly, starting at four hours after the inoculation. Initial hydrolysate concentration was \SIpct{5} or \SIml{250} of \SIl{5} and was increased over the next \SIh{24} time to \SIpct{30} or \SIl{2.1} of \SIl{7.0}. Feed rates between the points given in the table were interpolated linearly. \SIpct{100} pump output gave \SI{16.3}{\milli\litre\per\minute}.\label{tbl-lchf1-lch-feed}}
	\begin{tabular}{SSS}
		\toprule
		{Time after inoculation in \si{\hour}} & 
		{Pump output in \si{\percent}} & 
		{Total volume in \si{\litre}} \\
		\hline
		0.00 & 0.0 & 0.25 \\
		4.00 & 0.0 & 0.25 \\
		4.02 & 0.0 & 0.25 \\
		16.00 & 6.0 & 0.59 \\
		16.02 & 6.0 & 0.59 \\
		28.00 & 25.0 & 2.02 \\
		28.02 & 25.0 & 2.02 \\
		31.00 & 25.0 & 2.10 \\
		31.02 & 0.0 & 2.10 \\
		\bottomrule
	\end{tabular}
\end{table}
\begin{table}
	\centering
	\caption[7-Litre Fermentation Feeding Programme of Medium Concentrate]{7-litre Fermentation feeding programme of medium concentrate. Medium concentrate comprising of peptone, magnesium sulphate, calcium chloride, trace elements and vitamins at \SI{10}{\concfac} the final concentration were added during the course of fermentation to make up for the additional volume added by the lignocellulose hydrolysate. The starting volume was \SIl{5.0} litre which contained \SIml{500} medium concentrate. Feeding started four hours after the hydrolysate feed and ended after eight hours. Feed rates between the points given in the table were interpolated linearly. \SIpct{100} pump output gave \SI{15.0}{\milli\litre\per\minute}.\label{tbl-lchf1-concentrate-feed}}
	\begin{tabular}{SSS}
		\toprule
		{Time after inoculation in \si{\hour}} & 
		{Pump output in \si{\percent}} &
		{Total volume in \si{\milli\litre}} \\
		\hline
		0.00 & 0.0 & 500 \\
		20.00 & 0.0 & 500 \\
		20.02 & 0.0 & 500 \\
		28.00 & 5.6 & 700 \\
		28.02 & 5.6 & 700 \\
		31.00 & 5.6 & 700 \\
		31.02 & 0.0 & 700 \\
		\bottomrule
	\end{tabular}
\end{table}
At the start of fermentation, the fermenter contained \SIl{5.0} SM0 P100F with \SIpct{5.0} lignocellulose hydrolysate\footnote{\SIml{250} of \SIl{5.0}.} and \SI{1.0}{\milli\litre\per\litre} anti-foam B (1:10 in ultra-pure water, stirred). The medium contained phosphate for \SIl{7.0} from the beginning of the fermentation; all other compounds were fed through \SI{10}{\concfac} medium concentrate, so that at the start of fermentation, the fermenter contained \SIml{500} of medium concentrate already. Aeration was set and controlled to \SI{0.4}{\vvm} \footnote{At the beginning of fermentation this corresponded to \SI{120}{\litre\norm\per\hour}, after feeding \SI{168}{\litre\norm\per\hour}.}, temperature to \SIdC{30.0} and agitation to \SIrpm{400}. The initial pH value was set to \num{7.21} and controlled using \SIM{2.0} sodium hydroxide. Lignocellulose hydrolysate was fed according to \vref{tbl-lchf1-lch-feed}, medium concentrate was fed according to \vref{tbl-lchf1-concentrate-feed}. The acid pump was used for pumping lignocellulose hydrolysate, the base pump for \SIM{2.0} sodium hydroxide, the anti-foam pump for anti-foam and the level pump for \SI{10}{\concfac} medium concentrate. The fermenter was inoculated with two one day old precultures at $D_{600}$ of \num{5.8} and \num{6.0}. \SIml{170} preculture was centrifuged for \SImin{10} at \SIG{4000} and \SIdC{20}. The resulting \SIml{86} of jelly-like \enquote{pellet} was used to inoculate the fermenter with an initial $D_{600}$ of \num{0.2}.
% q = 0,4 vvm; n = 400 rpm; theta = 30 °C; Stromstörer, drei Stück; Zufütterung von LCH, fehlenden Mediumkomponenten; Pumpe für Antischaum, Pumpe für NaOH; Start mit 5\% LCH (250 ml von 5 l)
% 4 h n. Animpfen Zufütterung auf 30\% LCH n. weiteren 24 h (28 h n. A.); Zufüttervolumen: 1850 ml (+ 250 ml Vorlage = 2100 ml = 30\% von 7 l); 20 h n. Animpfen Zufütterung Restmediumkonzentrat (28 h n. A. 200 ml)
% ACID-Pumpe: 2,1 l LCH; BASE: 2 M NaOH; FOAM: 1:10 Antischaum B in ddH2O (gerührt); LEVEL: 710 ml Mediumkonzentrat
% Totvolumina: ACID: 10 ml; BASE: 9 ml; FOAM: 9 ml; LEVEL: 8 ml
% Startzustand: ACID: 0 ml (= SUBAT: 250 ml); BASE: 40 ml (initiale pH-Werteinstellung); FOAM: 0 ml; LEVEL: 0 ml (= SUBBT: 500 ml); 
% VK-D600: 6,0 und 5,8; vor Animpfen: 5 ml Antischaum hinzu (1 ml/l); Abzentrifugieren hat nicht wie erwartet geklappt, da deutliche "Wackelpudding"-Bildung --> mit 86 ml aufgeteilt auf 2 50-ml-Spritzen angeimpft, aus 170 ml Kultur zentrifugiert
% nach Animpfen: pH-Messung offline: 7,21 --> angezeigter pH-Wert also um 0,3 geringer als tatsächlicher pH-Wert
Between \SIh{15.5} and \SIh{24} after inoculation, the bottommost foam breaker slid down. An additional \SIml{4.0} anti-foam was fed manually into the fermenter \SIh{24} after inoculation. \SIh{25} after inoculation, aeration was manually set to \SI{2.8}{\litre\per\minute} to keep aeration rate constant at \SI{0.4}{\vvm}.
% q = 0,4 vvm sichergestellt via manuelles Setzen auf 2,8 l/min nach 25 h
% zwischen 15h 30 min und 24 h: unterster Schaumbrecher heruntergerutscht, Schaum daher an zweitem Schaumbrecher gewesen; entdeckt, nachdem 4 ml Antischaum hinzu (insgesamt 9 ml/7 l)
% 39 h: Beobachtung, dass pH-Wert basischer wird (7,4 (realer Wert))

\paragraph{Sampling Plan}
Samples were drawn after \SIlist{0.1;15.5;24.2;39.5;48.0;53.8;72.5;80.0}{\hour}. For the first sample, \SIml{28} was drawn from the fermenter. $D_{600}$ was determined using a 1:10 dilution in ultra-pure water for this and later samples, if not stated otherwise. For the second and third sample, the sampling tube was initially flushed with fresh sample and then the real sample was drawn. \SIml{1.0} was transferred into \SIml{1.5} tubes three times---one of them pre-weighed---and centrifuged for \SImin{5} at \SIG{17000} and room temperature. Supernatants were transferred into new tubes for monomer \& inhibitor analysis, molar mass determination and \glc{} determination. The slimy phase on top of the pellet in the pre-weighed tube was carefully transferred to another pre-weighed empty \SIml{1.5} tube. If slime stuck to the pipette tip, it was removed by flushing the tip with ultra-pure water, collecting it in the pre-weighed tube with the slime, centrifuging for \SImin{5} at \SIG{17000} and room temperature and discarding the supernatant.

For the fourth sample, a relatively stable foam formed inside the broth and was removed by centrifugation for \SImin{1} at \SIG{2000} and room temperature. From this sample on, the samples were diluted 1:2 with ultra-pure water to facilitate formation of a clearly distinguishable pellet, top phase and supernatant. Also, $D_{600}$ determinations were carried out with 1:20 dilutions from this sample on.

For the fifth sample, the 1:20 dilution for the determination of $D_{600}$ was prepared gravimetrically in addition to another measurement using the usual approach of pipetting the fermentation broth. From the sixth sample on, dilution for centrifugation was 1:3 with ultra-pure water. The fermentation broth was harvested \SIh{81.8} after inoculation: \SIl{2} was heat-inactivated at \SIdC{60} for \SImin{30}, another \SIl{2} was autoclaved and the remaining \SIl{3} was diluted 1:10 with ultra-pure water and centrifuged for \SImin{15} at \SIG{12000} and \SIdC{30}. Autoclaved broth, heat-inactivated broth and the supernatant of the 1:10 diluted broth were stored at \SIdC{4} for purification later.
% Probenahme 0: 5 min n. Animpfen; 28 ml; D600-Messung mit 1:10-Verdünnung
% Probenahme 1: D600-Messung wie zuvor; 3 x 1 ml in Eppis (1 davon vorher ausgewogen f. Pellet); 5 min@17kg, RT; Überstände überführen in Eppis für PMP, SEC, Glc-Assay; Schleim mit 1-ml-Pipette in ausgewogenes Schleimeppi überführen; falls Reste des Schleimes an/in Spitze kleben mit ddH2O rausspülen u. Schleimeppi zentrifugieren (5 min ...), Überstand verwerfen
% Probenahme 2: ab hier D600-Messung 1:20 verdünnt
% Probenahme 3: Schaum rel. stabil, sodass 1 min@2000g, RT, um D600-Messung sauber hinzubekommen; n. eigentlicher Zentrifugation keine deutliche "Oberphase" bekommen, daher resuspendiert und 1:2 mit ddH2O; ab hier alle Proben 1:20
% Probenahme 4: ab hier auch immer etwa 50 mg Brühe mit entsprechender Menge Wasser verdünnt
% 2. Schaumschläger irgendwann nach Probe 4 ebenfalls heruntergerutscht
% Probenahme 6: 1:3 für initiale Verdünnung
% Ernte dreigeteilt: 2 l @60 °C f. 30 min hitzeinaktiviert; 2 l autoklavieren; 3 l mit ddH2O verdünnen und zentrifugieren

