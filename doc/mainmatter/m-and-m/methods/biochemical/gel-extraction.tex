\subsection{Gel Extraction\label{subsec-gel-ex}}
Two kits for the purification of double-stranded DNA from agarose gels were used, the NucleoSpin Gel and PCR Clean-Up kit (MNG) and the QIAEX II Gel Extraction kit (QGD). As an alternative to the NucleoSpin kit's purification method, magnetic beads were used as well (Agencourt AMPure XP, BCG). The descriptions given here are short versions of the complete instructions.

\subsubsection{Gel Extraction (Macherey-Nagel Kit)\label{subsubsec-gel-ex-macherey-nagel}}
A tube with \SIul{200} of the buffer NTI per \SImg{100} of gel was prepared, the gel slice transferred into it and the tube was incubated for \SImin{10} at \SIdC{50} and \SIrpm{800}. Up to \SIul{700} of sample was transferred onto a silica column on top of a collection tube. The tube was centrifuged for \SIs{30} at \SIG{11000} and room temperature. The flow-through was discarded and the procedure repeated with the remaining sample.

After the sample was filtered, \SIul{700} of the buffer NT3 was transferred onto the column. The tube was centrifuged for \SIs{30} at \SIG{11000} and room temperature. The flow-through was discarded and the procedure repeated once. Then, the tube was centrifuged for \SImin{1} at \SIG{11000} and room temperature to remove buffer NT3 completely. Residual ethanol was removed by incubation at \SIdC{70} for \SIrange{2}{5}{\minute}.

The column was transferred onto a \SIml{1.5} tube and \SIrange{15}{30}{\micro\litre} of buffer NE was added to the top of the column. The tube was incubated at room temperature for \SImin{1} and centrifuged for \SImin{1} at \SIG{11000} and room temperature. The supernatant was transferred into a new tube and contained the purified DNA.

\paragraph{Purification with Magnetic Beads\label{subsubsec-gel-ex-magnetic-beads}}
After gel dissolution, \num{1.8} times the volume of the PCR product of resuspended magnetic beads was added to the dissolved gel. The beads were mixed thoroughly by pipetting and incubated for \SImin{5} at room temperature.

The tube was put on a magnetic rack and incubated until the liquid became clear and all the beads had gathered at the bottom. The supernatant was discarded. Then, \SIml{1.0} buffer NT3 was added to the pellet and incubated for \SIs{30} at room temperature. The supernatant was removed and another \SIml{1.0} buffer NT3 was added and the procedure repeated.

The beads were dried for approximately \SImin{3} at room temperature and the tube removed from the magnetic rack. The DNA was eluted by adding \SIul{40} buffer NE and resuspending the beads in it. The tube was put on a magnetic rack and incubated until the liquid became clear and all the beads had gathered at the bottom. The supernatant was transferred to a new tube and contained the purified DNA.

\subsubsection{Gel Extraction (Qiagen Kit)\label{subsubsec-gel-ex-qiagen}}
A tube with \SIul{300} of the buffer QX1 per \SImg{100} of gel was prepared and the slice transferred into it. A QIAEX II tube was vortexed for \SIs{30} and the appropriate volume of QIAEX II was transferred into the tube with the gel slice. The tube was incubated for \SImin{10} at \SIdC{50} and was vortexed every \SImin{2}. Then, the sample was centrifuged for \SIs{30} at \SIG{17000} and room temperature and the supernatant was discarded.

\SIul{500} of buffer QX1 was added to the pellet, the tube vortexed and centrifuged for \SIs{30} at \SIG{17000} and room temperature. The supernatant was discarded and the pellet was resuspended in \SIul{500} buffer PE, the tube vortexed and centrifuged for \SIs{30} at \SIG{17000} and room temperature. The supernatant was removed and the PE buffer step repeated.

The sample was air-dried for \SIrange{10}{15}{\minute} and \SIul{20} of \SImM{10} \ce{Tris*HCl} at pH value \num{8.5}, TE buffer or water was added. The pellet was resuspended by vortexing and incubated at room temperature for \SImin{5}.

The tube was centrifuged for \SIs{30} at \SIG{17000} and room temperature and the supernatant was transferred into a new tube and contained the purified DNA.

