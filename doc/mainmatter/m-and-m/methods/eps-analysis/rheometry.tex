\subsection{Determination of Rheological Properties}
All rheometric measurements were conducted on an MCR 300 (APG) using cone-plate geometry and a measurement temperature of \SIdC{20}. All measurements were performed using \SIpct{0.5} \eps{} solutions in \SIpct{1} \ce{KCl} in ultra-pure water. Total measurement times were limited to \SImin{10} due to sample evaporation and a lack of know-how to prevent this.

\subsubsection{Dynamic Viscosity}
Dynamic viscosities were determined at \SIps{1} and \SIps{1000}. The sample was applied carefully using cut \SIul{1000} tips. Measurements at \SIps{1} comprised 300 points each measured for \SIs{2} for a total measurement time of \SImin{10}. Measurements at \SIps{1000} comprised 120 points each measured for \SIs{5}. The viscosity was determined by using the arithmetic mean of the last \SIs{60} of each measurement at \SIps{1} and the points from \SIs{180} to \SIs{240} at \SIps{1000}. Total measurement times, number of points, measurement time per point and---as a consequence---the ranges used for the determination of the dynamic viscosity were varied.

\subsubsection{Thixotropy}
Thixotropy, time-dependent shear-thinning, and the regain of the initial viscosity was determined using a three-step protocol: the sample was sheared for 120 points each measured for \SIs{2} at \SIps{1}, then for 24 points of \SIs{5} at \SIps{100} and again for \SIs{240} at \SIps{1}. In order to achieve greater time resolution, the third step was split into 20 points of \SIs{0.5} each and 115 points \SIs{2} each.

The first step was used to calculate the dynamic viscosity before shearing from the values from \SIs{180} to \SIs{240}. The time needed to regain \SIlist{10; 25; 50; 75; 90}{\percent} of the initial viscosity after the shearing of the second step stopped was calculated from the data of the third step.

