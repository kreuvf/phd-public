\subsection{Determination of Molar Mass}
% Description for scleroglucan/schizophyllan not included on purpose
% SK5, p. 187+
Samples of crude fermentation broth were centrifuged for \SImin{5} at \SIG{17000} and \SIdC{20} and the clear supernatant was transferred to an HPLC vial and analysed. Samples purified by cross-flow filtration were diluted 9:10 with \SIM{1} \ce{LiNO3} to give a final \ce{LiNO3} concentration of \SIM{0.1}.

\nomenclature[latabbr_GPC]{GPC}{gel permeation chromatography}
Samples were analysed on a PSS SECcurity SEC system which featured a SECcurity vacuum degasser, a 1260 Infinity isocratic pump, a 1260 Infinity Standard autosampler, a SECcurity TCC6000 column oven, a BI-MwA multi-angle laser light scattering detector and a 1260 Infinity refractive index detector. The complete system was purchased from PSS Polymer Standards Service GmbH, Mainz and contained parts from Agilent Technologies, Waldbronn (pump, autosampler, refractive index detector) and Brookhaven Instruments Corporation, Holtsville, New York, USA (MALLS detector) as well. Separation was facilitated by using a TSK-GEL Alpha-M column with guard column (TBG). \SIul{100} of sample was injected, the mobile phase consisted of \SIM{0.1} \ce{LiNO3}, the flow rate was \SI{0.6}{\milli\litre\per\minute} and the column temperature was \SIdC{50}. Data were recorded using PSS WinGPC UniChrom V 8.10, Build 2830 (PSS) on a personal computer running Microsoft Windows 7, data were processed using the same software on a Microsoft Windows XP laptop.

For all unknown polymers the molar mass could only be estimated from a pullulan reference curve consisting of the following standards: \SID{342} (\num{1.00}), \SIkD{1.08} (\num{1.23}), \SIkD{6.10} (\num{1.05}), \SIkD{9.60} (\num{1.09}), \SIkD{21.1} (\num{1.09}), \SIkD{47.1} (\num{1.07}), \SIkD{107} (\num{1.13}), \SIkD{200} (\num{1.11}), \SIkD{344} (\num{1.15}), \SIkD{708} (\num{1.27}), \SIMD{1.22} (\num{1.37}) and \SIMD{2.35} (\num{1.49}). The molar masses refer to the molar mass at the refractive index peak and the parenthesized values denote the polydispersity index of the corresponding standard derived from the refractive index signal.

