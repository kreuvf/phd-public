\subsection[Quantification of β-1,6-linked \GLC{} Side Chains]{Quantification of the β-1,6-linked \GLC{} Content in β-1,3-β-1,6-Glucans\label{periodate-test}}
Periodate specifically reacts with β-1,3-β-1,6-glucans \cite{Johnson1963, BuchRobyt1998}. While the β-1,3-glucose backbone remains intact, the β-1,6-linked \glc{} residues react with periodate: per mole β-1,6-linked \glc{} two moles periodate are consumed and one mole \fora{} is formed. Periodate can be quantified spectrophotometrically at \SInm{290} and \fora{} via HPLC. Also, the resulting pH shift can be measured. Then, the absolute periodate consumption and the absolute \fora{} formation as well as the ratio of the two can be used to compare different polymer samples.

The protocol presented here is in part based on an experimental in-house protocol by Broder Rühmann.

\paragraph{Periodate Calibration Curve\label{periodate-calibration}}
Sodium periodate was dissolved in ultra-pure water to prepare a calibration curve with the following concentrations: \SIlist{20000; 15000; 10000; 8000; 6000; 4000; 1500; 1000; 800; 600; 400; 150; 0}{\micro\M}.

\paragraph{Periodate Reaction and Sampling\label{periodate-sampling}}
First, \SIml{5} of polymer solution was put into a \SIml{15} reaction tube with an aluminium foil sheath. The reaction was started by the addition of \SIml{5} \SImM{20} sodium periodate solution and mixing by vigorous pipetting. Then, samples of the reaction mixtures were taken immediately and after \SIlist{1; 2; 3; 4; 5}{\day} to determine the absorption at \SInm{290} and later the concentration of \fora{}.

For every sample, \SIul{300} reaction mixture was transferred into a UV micro-cuvette and the absorption at \SInm{290} against ultra-pure water measured. Then, \SIul{980} of reaction mixture was transferred into \SIml{1.5} reaction tubes which were prepared with \SIul{20} \SIM{1} sodium thiosulfate solution to stop the reaction. The stopped sample was incubated at room temperature over night to ascertain complete removal of periodate. On the next day, the samples were stored at \SIdC{-20} until the HPLC measurement. The pH value of the reaction mixture was measured only for the first and last samples. Between sampling, the reaction mixtures were kept in the dark.

\paragraph{HPLC Measurements\label{periodate-hplc}}
\Fora{} was determined via HPLC from thawed samples. The samples were filtered through modified PES spin filters with \SIkD{10} cut-off to remove polymeric substances and then analysed via HPLC using the same setup as described under \vref{subsec-inh-acid}. The following concentrations were used for the \fora{} calibration curve: \SIlist{2000; 1600; 1200; 800; 400; 200; 160; 120; 80; 40}{\micro\M}.

