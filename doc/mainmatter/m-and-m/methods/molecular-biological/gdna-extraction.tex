\subsection{Extraction of Genomic DNA}
\nomenclature[formula_g]{\textit{g}}{average gravitational acceleration at sea level on Earth, approximately \SI{9.81}{\metre\per\square\second}}
\nomenclature[chem_SDS]{SDS}{sodium dodecyl sulphate}
\subsubsection{Gram Positive Bacteria}
\SIml{4} of liquid culture containing the Gram positive bacterium was centrifuged for \SImin{10} at \SIG{21000} and \SIdC{4} (Heraeus Fresco 21, TFS). The pellet was washed two times with ultra-pure water and centrifuged for \SImin{10} at \SIG{21000} and \SIdC{4}. The pellet was resuspended in \SIul{200} \ce{Tris*HCl} buffer (\SImM{50} \ce{Tris*HCl}, \SImM{10} EDTA, pH value 8). \SIul{25} lysozyme solution (\SImgpml{25}) was added to the buffered suspension. Incubation for \SIrange{60}{90}{\minute} at \SI{450}{\rpm} and \SIdC{37} (Tmix, AJA) followed. Then, the lysed cells were incubated with \SIul{50} SDS solution (\SIpct{10}) for \SImin{10} at room temperature. \SIul{2} RNase A/T1 mix (TFS) was added, followed by incubation for \SImin{60} at \SI{450}{\rpm} and \SIdC{37}. \SIul{3} proteinase K (TFS or NEB) was added, followed by incubation for \SImin{60} at \SI{450}{\rpm} and \SIdC{60}. In order to reduce shear stress, from this point on, only cut tips were used. \SIul{300} acetate buffer (\SIM{3} sodium acetate, pH value \num{4.8}) was added and the sample carefully inverted several times. After centrifugation for \SImin{10} at \SIG{21000} and room temperature, the clear supernatant was transferred into a fresh \SIml{1.5} tube. DNA extraction was facilitated by adding \SIul{150} phenol (CRG) and \SIul{150} C/I mix (CRG), inverting carefully, spinning down and transferring the top phase to a new tube for three times. \SIml{1} of \SIpct{96} ethanol at \SIdC{-20} was added and the tube inverted carefully. The tube was stored overnight at \SIdC{-20}. On the following day, the tube was centrifuged for \SImin{15} at \SIG{21000} and \SIdC{4}. The supernatant was discarded and the DNA air dried in-place for approximately \SImin{10}; care was taken to not overdry the DNA. Finally, the DNA was dissolved in \SIul{50} ultra-pure water and stored at \SIdC{4}. This protocol is based on \cite{Saha2005}.

\nomenclature[chem_C/I]{C/I}{chloroform/isoamyl alcohol}
\subsubsection{Gram Negative Bacteria}
\SIml{2} of a liquid culture was centrifuged for \SImin{10} at \SIG{21000} and \SIdC{4} (Heraeus Fresco 21, TFS). \SIul{267} lysis buffer (\SImM{40} Tris acetate at pH value \num{7.8}, \SImM{20} sodium acetate, \SImM{1} EDTA, \SIpct{1} SDS) was added to the pellet. The pellet was resuspended by vigorous pipetting, not by vortexing, to prevent foam formation. \SIul{3} RNase A/T1 mix (TFS) was added, followed by incubation for \SImin{60} at \SI{450}{\rpm} and \SIdC{37} (Tmix, AJA). \SIul{90} \SIM{5} NaCl solution was added, proteins and cell debris precipitated and the solution became viscous. After centrifugation for \SImin{10} at \SIG{21000} and \SIdC{4}, the clear supernatant was transferred to a new tube. One volume of C/I mix (CRG) was added and the tube inverted gently for at least \num{50} times until the liquid became milky. After centrifugation for \SImin{10} at \SIG{21000} and \SIdC{4}, the clear supernatant was transferred to a new tube. \SIml{1} of \SIpct{96} ethanol at \SIdC{-20} was added and the tube inverted carefully. The tube was stored overnight at \SIdC{-20}. On the next day, the tube was centrifuged for \SImin{15} at \SIG{21000} and \SIdC{4}. The supernatant was discarded. The DNA was further purified by performing the following procedure twice: the pellet was washed with \SIml{1} of \SIpct{70} ethanol, centrifuged for \SImin{15} at \SIG{21000} and \SIdC{4} after which the supernatant was discarded. The DNA was air dried for approximately \SImin{10}; care was taken to not overdry the DNA. Finally, the DNA was dissolved in \SIul{50} ultra-pure water and stored at \SIdC{4}. This protocol is based on \cite{Chen1993}.

\subsubsection{Additional Steps for \EPS{} Producers}
While highly viscous cultures were a minor obstacle, lysis of \eps{} producers was more difficult. The former resulted in small or no pellets at all and was tackled by using different media or a younger culture or dilution prior to centrifugation. Lysis of \eps{} producers was improved by additional lysis steps.

Instead of using \SIml{4} of microorganism suspension directly, \SIml{2} was transferred into a \SIml{50} tube, diluted to \SIml{35} with ultra-pure water, well mixed and centrifuged for \SImin{20} at \SIG{4580} and \SIdC{4} (ROTANTA 460 R, AHT). The supernatant was discarded, the pellet resuspended in \SIml{1} ultra-pure water and additional \SIml{2} of microorganism suspension was transferred into the tube, diluted to \SIml{35} with ultra-pure water, well mixed and centrifuged for \SImin{20} at \SIG{4580} and \SIdC{4}. The supernatant was discarded, the pellet resuspended in \SIml{1.5} ultra-pure water and centrifuged for \SImin{10} at \SIG{21000} and \SIdC{4} (Heraeus Fresco 21, TFS). The supernatant was discarded and the pellet resuspended in \SIml{1.5} ultra-pure water. The tube was subjected to quick-freezing for \SImin{1} in liquid nitrogen and heating for \SImin{10} at \SIdC{60} in a water bath for three times, the same method used by Michael Loscar for cell lysis. The tube was centrifuged for \SImin{5} at \SIG{21000} and room temperature and the supernatant was discarded. At this point, the cell wall was weak enough and the Gram positive protocol was used starting with the resuspension of the pellet in \SIul{200} \ce{Tris*HCl} buffer.

