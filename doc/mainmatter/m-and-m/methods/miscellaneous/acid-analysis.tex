\subsection{Determination of \ACET{}, \FORA{} and \LAEV{}\label{subsec-inh-acid}}
Microorganisms were grown in the presence of an acid---\acet{}, \fora{} or \laev{}---and had to be removed. Cultures in deep 96-well plates were centrifuged for \SImin{30} at \SIG{3710} and \SIdC{20}. \SIul{20} supernatant was mixed with \SIul{180} ultra-pure water and filtered in a \SIkD{10} PES filtration plate (AcroPrep Advance 350 10K Omega, PC) by centrifuging for \SImin{30} at \SIG{1200} and \SIdC{20}. If the filtrate volume was less than \SIul{155}, only \SIul{80} filtrate was taken and diluted with \SIul{80} ultra-pure water. % entgegen ursprünglichem Plan bei Verdünnung NICHT 2,5 mM H2SO4 eingestellt! SK5, p. 61/62

The samples were analysed on an UltiMate 3000~RS HPLC system (TFS) which featured an SRD-3400 degassing module, an HPG-3400RS binary pump, a WPS-3000TRS autosampler, a TCC-3000RS column department, a PDA-3000 photo-diode array detector and an RI-101 refractive index detector (SDK). System control and data collection were done by a PC running Microsoft Windows XP and Chromeleon 6.80 SR8 Build 2623 (156243) (DC).

The sample injection volume was \SIul{10} and the mobile phase consisted of \SImM{2.5} sulphuric acid. Samples were separated at \SI{0.5}{\milli\litre\per\minute} and \SIdC{70} on a Rezex ROA-Organic Acid H+ (\SIpct{8}) column (PL). Acids were detected at \SInm{210}. Samples and calibration standards were run for \SImin{25}. Standard 1 comprised of \SIgpl{5.0} \acet{} and \fora{}, standard 2 comprised of \SIgpl{5.0} \laev{}. Standards were injected at different volumes: \SIlist{10;6;3;1;0.6;0.3;0.1}{\micro\litre} corresponding to concentrations of \SIlist{5.0;3.0;1.0;0.5;0.3;0.1;0.05}{\gram\per\litre}, respectively.

