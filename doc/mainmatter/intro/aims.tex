% Ausnahmsweise
\newpage
\section{Aims of the Thesis\label{sec-intro-aims}}
\subsection{Bacterial Conversion of \LCH{} to \EPS{}s}
\EPS{}s are a diverse class of polymers with the potential to replace commonly used petrol-based polymers. But, competition with fuel or food resources might arise using the usual carbon sources used in fermentation processes such as \glc{} or sucrose. In order to avoid this, \lch{} will be used as a rather cheap resource supplied by an industry partner. Due to toxic by-products in the \lch{}, microbial growth and/or production might be inhibited.

Therefore, the aims of this project are:
\begin{itemize}
	\item \EPS{} Screening: Using a state-of-the-art method developed in-house \cite{Ruehmann2015a} the \amc{}s of the polymers of strains from a strain collection care analysed.
	\item Substrate Screening: \LCh{}s contain not only \glc{}, but other sugars such as \xyl{} as well. As resource efficiency is key in industrial processes, only strains exhibiting a broad substrate spectrum are considered.
	\item Inhibitor Screening: Single isolated toxic by-products of \lch{}s and \lch{} itself are tested with the strains from the strain collection to pick out those which are most resistant.
	\item Strain Selection: From all the data gathered on the strains, some strains are chosen for in-depth experiments using classical microbiological and molecular biological methods for characterization.
	\item Fermentation \& Downstream Processing: At least one strain is fermented in bioreactors of varying size to determine basic process and scale-up data.
\end{itemize}

The final aim strived for is the robust production of one \eps{} at \si{\gram\per\litre} scale.

\subsection{Fermentative Production of \SCL{} and \SHZ{}}
\SHZ{} and \SCL{} are a products already being marketed. The polymer \scl{} appears to be identical to \shz{}, yet identity claims are generally based on fractions of the polymers only. In order to get an answer to the question whether \scl{} and \shz{} are truly the same or not, this project's aims are:
\begin{itemize}
	\item Fermentations of \comm{} and \rolf{} for the production of the polymers \shz{} and \scl{}, respectively.
	\item Purification and analyses of the polymers to compare \scl{} and \shz{}.
	\item If necessary, analytical methods are developed.
\end{itemize}
 