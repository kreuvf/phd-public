\section{\EPS{}s---Valuable Microbial Slime\label{sec-intro-eps}}
The microbial product in the focus of this work are \eps{}s and is introduced in this section. The start makes a definition of the term \enquote{\eps{}}, followed by details on the general structure of \eps{}s. The next subsection deals with the analytical methods used to assess various \eps{} properties. Since the production of \eps{}s is a central topic of this work, the fermentative production is detailed thereafter.

The final subsection is dedicated to two very different commercially successful \eps{}s: the bacterial \eps{} xanthan, which is approved for use in food, and the fungal \eps{}s \scl{} and \shz{}, which are used as thickening agents or in drilling fluids \cite{Schmid2011}.

\subsection{Definition\label{subsec-intro-eps-def}}
\nomenclature[latabbr_EPS]{EPS}{exopolysaccharide}
\nomenclature[latabbr_IUPAC]{IUPAC}{International Union of Pure and Applied Chemistry}
\EPS{}s are polysaccharides which are either produced extracellularly or intracellulary and exported out of the cell. \enquote{EPS} is commonly used as an abbreviation for \enquote{exopolysaccharide} \cite{Kumar1998, Sutherland2001, Broadbent2003, RuasMadiedo2005, Freitas2011}, but also for \enquote{extracellular polymeric substance} \cite{Beech1991, Foster2000, Geyik2016}. While all \eps{}s are also extracellular polymeric substances, the reverse is not true.
In addition to \enquote{pure} polysaccharides, \eps{}s may be further modified with e.g. \acet{} or pyruvic acid \cite{Jansson1975}. Like other polymers, \eps{}s can be categorized based on their monomers as either homo- or heteropolymers as well. Homopolymers exclusively consist of one species of monomer following the IUPAC definition \cite{IUPACgoldbook}, while heteropolymers are made up of different monomers and correspond to copolymers as defined by IUPAC \cite{IUPACgoldbook}. The commercial examples below (\vpageref{subsec-intro-eps-commercial}) cover both types.

A comprehensive definition was given by \textcite{Sutherland1990} and includes additional features such as the contribution of the polysaccharide to microbial structure:
\begin{quote}
The surface of the microbial cell is a rich source of \chc{} molecules. Some of these are unique types, confined to a limited range of microorganisms. These are the components of the microbial cell walls [...]. However, in addition to these wall components, polysaccharides may be found either associated with other surface macromolecules or totally dissociated from the microbial cell. These are \textbf{\eps{}s}, extracellular polysaccharides [...].

[...]

Definition of \eps{}s is more difficult than definition of the \chc{} polymers found in microbial walls. [...] The term \textit{glycocalyx}, introduced by Costerton, fails to differentiate between the different chemical entities found at the microbial surface. [...]

The \eps{}s do not in themselves normally contribute to microbial structure; the other components of the cell surface are unaltered if exopolysaccharides are absent. [...]

[...]

The presence of \eps{}s associated with microbial cells grown on solid surfaces is frequently recognisable from the mucoid colony morphology. [...] In liquid medium, \eps{}-producing cultures may become very viscous or, exceptionally, may solidify as a gel. The \eps{} may form part of a \textit{capsule} firmly attached to the bacterial cell surface. Alternatively, it may be observed as loose \textit{slime} secreted by microorganisms but not directly attached to the cell. [...]
\end{quote}
The book by \textcite{Sutherland1990} is recommended to all readers who wish to get an in-depth introduction on \eps{}s.

\subsection{Structure\label{subsec-intro-eps-structure}}
\begin{figure}
	\subfloat[\GLC{}]{
			\label{fig-intro-glc}%
			\includegraphics[width=0.1460\textwidth]{fig/Molecule_glucose_b-d-pyr_600dpi.png}
	}
	\hfill
	\subfloat[\GLCN{}]{
			\label{fig-intro-glcn}%
			\includegraphics[width=0.1460\textwidth]{fig/Molecule_glucosamine_a-d_600dpi.png}
	}
	\hfill
	\subfloat[\Glcua{}]{
			\label{fig-intro-glcua}%
			\includegraphics[width=0.1460\textwidth]{fig/Molecule_glucuronic_acid_b-d-pyr_600dpi.png}
	}
	\hfill
	\subfloat[\GLCNAC{}]{
			\label{fig-intro-glcnac}%
			\includegraphics[width=0.1460\textwidth]{fig/Molecule_n-acetyl-glucosamine_a-d_600dpi.png}
	}
	\hfill
	\subfloat[\IIDGLC{}]{
			\label{fig-intro-2dglc}%
			\includegraphics[width=0.1460\textwidth]{fig/Molecule_2-deoxy-glucose_b-d-pyr_600dpi.png}
	}

	\subfloat[\GAL{}]{
			\label{fig-intro-gal}%
			\includegraphics[width=0.1460\textwidth]{fig/Molecule_galactose_b-d-pyr_600dpi.png}
	}
	\hfill
	\subfloat[\GALN{}]{
			\label{fig-intro-galn}%
			\includegraphics[width=0.1460\textwidth]{fig/Molecule_galactosamine_a-d_600dpi.png}
	}
	\hfill
	\subfloat[\textsc{d}-Ga\-lac\-turonic~acid]{
			\label{fig-intro-galua}%
			\includegraphics[width=0.1460\textwidth]{fig/Molecule_galacturonic_acid_b-d-pyr_600dpi.png}
	}
	\hfill
	\subfloat[\GALNAC{}]{
			\label{fig-intro-galnac}%
			\includegraphics[width=0.1460\textwidth]{fig/Molecule_n-acetyl-galactosamin_a-d_600dpi.png}
	}
	\hfill
	\subfloat[\FUC{}]{
			\label{fig-intro-fuc}%
			\includegraphics[width=0.1460\textwidth]{fig/Molecule_fucose_b-l-pyr_600dpi.png}
	}

	\subfloat[\MAN{}]{
			\label{fig-intro-man}%
			\includegraphics[width=0.1460\textwidth]{fig/Molecule_mannose_a-d-pyr_600dpi.png}
	}
	\hfill
	\subfloat[\RHA{}]{
			\label{fig-intro-rha}%
			\includegraphics[width=0.1460\textwidth]{fig/Molecule_rhamnose_a-l-pyr_600dpi.png}
	}
	\hfill
	\subfloat[\ARA{}]{
			\label{fig-intro-ara}%
			\includegraphics[width=0.1460\textwidth]{fig/Molecule_arabinose_a-l-fur_600dpi.png}
	}
	\hfill
	\subfloat[\XYL{}]{
			\label{fig-intro-xyl}%
			\includegraphics[width=0.1460\textwidth]{fig/Molecule_xylose_b-d-pyr_600dpi.png}
	}
	\hfill
	\subfloat[\RIB{}]{
			\label{fig-intro-rib}%
			\includegraphics[width=0.1460\textwidth]{fig/Molecule_ribose_b-d-pyr_600dpi.png}
	}
	\caption[Overview on Possible Carbohydrate Monomers of \EPS{}s]{Overview on possible carbohydrate monomers of \eps{}s. Carbohydrate monomers found in \eps{}s. Subfigures \subref{fig-intro-glc} to \subref{fig-intro-2dglc} depict \glc{} \cite{Bryan1986, Tait1986, Cerning1994, Lee1995, Grobben1996, Grobben1997, Kai2003, Lee2007, Raza2011} and its derivatives \glcn{} \cite{Kurita1977, Hia2004}, \glcua{} \cite{Tait1986, Raza2011}, \glcnac{} \cite{Kurita1977} and \iidglc{}, while subfigures \subref{fig-intro-gal} to \subref{fig-intro-fuc} depict \gal{} \cite{Bryan1986, Cerning1994, Lee1995, Grobben1996, Grobben1997, Kai2003, Lee2007, Raza2011} and its derivatives \galn{} \cite{Hia2004}, \galua{} \cite{Mort1982}, \galnac{} \cite{Doco1990} and \fuc{} (6-deoxy-\textsc{l}-ga\-lac\-tose) \cite{Lee1995}. Here, \fuc{} and not \dfuc{} is depicted for its abundance in microbial \eps{}s.
Subfigures \subref{fig-intro-man} and \subref{fig-intro-rha} depict two other C6 sugars commonly found: \man{} \cite{Bryan1986, Tait1986, Cerning1994, Lee1995, Kai2003, Lee2007, Raza2011} and \rha{} (6-deoxy-\textsc{l}-mannose) \cite{Bryan1986, Cerning1994, Grobben1996, Grobben1997}. Subfigures \subref{fig-intro-ara} to \subref{fig-intro-rib} depict the C5 sugars \ara{} \cite{Burdman2000}, \xyl{} \cite{Cerning1994, Lee1995, Lee2007} and \rib{} \cite{Hisamatsu1997}. This list is not exhaustive; further carbohydrates including variants of the sugars depicted here such as the corresponding furanose or pyranose forms cannot be excluded.
%\subref*{fig-intro-glc}: β-\textsc{d}-glucose;
%\subref*{fig-intro-glcn}: α-\textsc{d}-glucosamine;
%\subref*{fig-intro-glcua}: α-\textsc{d}-glucuronic acid;
%\subref*{fig-intro-glcnac}: \textit{N}-acetyl-α-\textsc{d}-glucosamine;
%\subref*{fig-intro-2dglc}: 2-deoxy-β-\textsc{d}-glucose;
%\subref*{fig-intro-gal}: β-\textsc{d}-galactose;
%\subref*{fig-intro-galn}: α-\textsc{d}-galactosamine;
%\subref*{fig-intro-galua}: β-\textsc{d}-galacturonic acid;
%\subref*{fig-intro-galnac}: \textit{N}-acetyl-α-\textsc{d}-galactosamine;
%\subref*{fig-intro-fuc}: β-\textsc{l}-fucose;
%\subref*{fig-intro-man}: α-\textsc{d}-mannose;
%\subref*{fig-intro-rha}: α-\textsc{l}-rhamnose;
%\subref*{fig-intro-ara}: α-\textsc{l}-arabinose;
%\subref*{fig-intro-xyl}: β-\textsc{d}-xylose;
%\subref*{fig-intro-rib}: β-\textsc{d}-ribose;
\label{fig-intro-eps-monomers}}
\end{figure}
Different sugar monomers occur in \eps{}s: plain carbohydrates with the empirical formula \ce{C_{n}H_{2n}O_{n}} such as \glc{}, \gal{} or \xyl{}, and also amino sugars, deoxy sugars, uronic acids and acetylated amino sugars such as \glcn{}, \fuc{}, \galua{} and \glcnac{}, respectively. \Vref{fig-intro-eps-monomers} lists some of these monomers, but one should not be under the impression that all \eps{}s could be generated with the monomers given in this figure: alginate for example consists of α-\textsc{d}-gu\-lu\-ron\-ic acid and β-\textsc{d}-man\-nu\-ron\-ic acid \cite{Fischer1955, Hirst1958, Whistler1959}, both monomers are missing from \vref{fig-intro-eps-monomers}.

The predominant linkage in \eps{}s is the glycosidic bond and the exact type of the linkage directly influences the \eps{} properties. This is generally true for all polysaccharides, not only \eps{}s: alginates form gels with \ce{Ca^{2+}} ions \cite{Smidsrod1965}, cellulose forms highly crystalline regions \cite{Fan1982} and xanthan is approved as thickener in food \cite{EURegulation11292011}. Further examples and more details on different \eps{}s can be found in a review by \textcite{Kumar2007}.

\subsection{Analytical Methods\label{subsec-intro-eps-analytics}}
\nomenclature[latabbr_HPLC]{HPLC}{high pressure liquid chromatography}
\nomenclature[latabbr_MS]{MS}{mass spectrometry}
\nomenclature[latabbr_HPLC-MS]{HPLC-MS}{high pressure liquid chromatography coupled with mass spectrometry}
\nomenclature[latabbr_GC-MS]{GC-MS}{gas chromatography coupled with mass spectrometry}
Characterization of \eps{}s encompasses many different analytical techniques, most of them requiring expensive equipment, and is time-consuming. The overall monomer composition can be analysed by hydrolyzing the polymer under acidic conditions to free the monomers. Then, the monomers can be analysed directly via HPLC \cite{Baker1986, Lopes2008} or can be converted into volatile alditol acetates and analysed via gas chromatography \cite{Englyst1984, Verbruggen1995}. Aldose monomers can be derivatized with 3-methyl-1-phenyl-2-py\-raz\-o\-line-5-one and analysed via HPLC-MS \cite{Honda1989, McRae2011, Ruehmann2014, Ruehmann2015a}. Linkages can be analysed using methylation analysis via GC-MS \cite{Sandford1966, Verbruggen1995}. Pyruvylation can be quantified enzymatically \cite{Hashimoto1998} and acetylation via hydroxamic acid \cite{McComb1957, Lopez2004b}.

\nomenclature[formula_dn/dc]{$\frac{dn}{dc}$}{refractive index increment}
\nomenclature[latabbr_i.e.]{i.e.}{in exemplum}
Polymer size and molar mass can be analysed via size-exclusion chromatography. Often, no calibration standard for the polymer in question is available, hence molar masses can only be given in relation to a reference such as dextran or polystyrene. Using multi-angle laser light scattering, it is possible to determine the polymer molar mass without a reference when the refractive index increment, $\frac{dn}{dc}$ is known. The $\frac{dn}{dc}$ can be determined with a refractometer. For very bulky molecules, size-exclusion columns cannot separate the largest molecules anymore; hence, other methods such as field-flow fractionation should be used instead \cite{Arndt1996}.

In the next step, more application-relevant properties, i.e. rheological properties such as the dynamic viscosity at specific shear rates, concentrations, temperatures or ionic strengths, adhesion properties, binding properties for certain metal ions, emulsification properties or mechanical properties of test specimen are characterized. For most of these analyses, special equipment is needed.

\subsection{Fermentative \EPS{} Production\label{subsec-intro-eps-production}}
Microbial \eps{}s are excreted into the medium by the respective microorganism. Therefore, fermentation processes are used for \eps{} production. In a fermentation process, the respective microorganism is grown in a sterile medium with aeration, agitation, temperature and pH control. Since the microorganisms used are usually aerobic, the dissolved oxygen concentration is monitored or controlled. The off-gas oxygen and carbon dioxide contents can be monitored as well. Additional medium components can be fed during the course of the fermentation.

Several \eps{}s increase the broth viscosity over time. This negatively affects mixing, which in turn affects all mixing-dependent parameters such oxygen transfer, mass transfer, and the time needed to measure the effect of acid/base additions to the medium. With poor mixing, dissolved oxygen levels decrease which could impair microbial growth and/or production.

\nomenclature[formula_D600]{$D_{600}$}{attenuance at \SInm{600}}
The measurement of other parameters such as the attenuance at \SInm{600} $D_{600}$ for the estimation of the microbial cell concentration, the cell dry mass, the concentration of the carbon source used or other molecules of interest such as inhibitors to microbial growth or the molar mass distribution are realized via off-line analysis of dedicated samples at sensible sampling times. With off-line samples, more elaborate analysis techniques can be used to give insights that are not usually obtainable from on-line measurements.

At the end of the fermentation, the contents of the fermenter are harvested and subjected to downstream processing---the various techniques used to separate the product from all impurities. Commonly, cells are separated by centrifugation and the supernatant is purified further. In the case of \eps{}s, the supernatant may be precipitated directly using alcohol or purified using cross-flow filtration to leave only big molecules such as \eps{} chains in the solution and remove all small molecules. Cross-flow filtration may be used to concentrate the product and save alcohol for the precipitation. The precipitated product is dried and may be ground to defined particle sizes for faster dissolution.

\subsection{Commercial \EPS{}s\label{subsec-intro-eps-commercial}}
\begin{figure}
	\begin{center}
		\includegraphics[width=0.6179\textwidth]{fig/Molecule_xanthan_600dpi.png}
		% Textwidth per pixel for unified sizes of molecules across figures: 1.463145105E-4
		\caption[Structure of the Repeating Unit of Xanthan]{Structure of the repeating unit of xanthan. Xanthan consists of a dimeric β-1,4-linked \glc{} backbone (\circi{}) with branches consisting of a trimer at every second \glc{} residue. The trimer is connected via an α-1,3-glycosidic link to \man{} (\circii{}). The branch consists of a β-\man{} (\circiv{}), which is 1,4-linked to a β-\glcua{} (\circiii{}), which in turn is connected via a 1,2-link to the α-\man{} (\circii{}). The α-\man{} (\circii{}) can be \textit{O}-acetylated (\circv{}) at the C6 position. The terminal β-\man{} (\circiv{}) can carry a pyruvate ketal (\circvi{}) formed from the hydroxy groups in the C4 and C6 positions and pyruvic acid \cite{Jansson1975}.\label{fig-intro-eps-xanthan}}
	\end{center}
\end{figure}

In this section, three very different microbial \eps{}s are introduced: xanthan and \scl{}/\shz{}. All three polymers are exploited commercially and cover bacterial heteropolymers and fungal homopolymers. A general review of hydrocolloids for thickening by \textcite{Saha2010} is recommended for further reading.

\subsubsection{Xanthan\label{subsubsec-intro-eps-commercial-xanthan}}
Bacteria of the genus \mo{Xanthomonas}, especially \mo{Xanthomonas campestris} produce the heteropolymer xanthan. The general structure of xanthan is shown in \vref{fig-intro-eps-xanthan}. Xanthan is a branched heteropolymer with a β-1,4-linked \glc{} backbone, which is the same structure as in cellulose. Sidechains are attached to every other backbone monomer to the C3 position of the backbone. The sidechain consists of a β-\man{} linked via a 1,4-bond to β-\glcua{}. The β-\glcua{} is 1,2-linked to an α-\man{}, which is linked via a 1,3-bond to the backbone. Up to \SIpct{50} of the terminal β-\man{} carries a ketal of pyruvate and the hydroxy groups at the C4 and C6 positions \cite{Jansson1975}. Compared to most other \eps{}s, the biochemical pathways and genes involved in xanthan synthesis are well-known \cite{Becker1998}.

The rheology of aqueous solutions of xanthan has been studied in detail \cite{Whitcomb1975, Rinaudo1978, Milas1986, Richardson1987, Rochefort1987, Nolte1992, Becker1998}. Solutions of xanthan are highly viscous at room temperature and exhibit shear-thinning behaviour \cite{Rinaudo1978, Milas1986, Richardson1987, Nolte1992} and quickly regain their viscosity after shearing \cite{Chen1980}. Xanthan solutions are stable with regards to ionic strength \cite{Rinaudo1978} and temperatures up to \SIdC{80} \cite{Rochefort1987}. The high viscosity during fermentation of the strictly aerobic strains reduces the oxygen transfer rate \cite{Becker1998}, which can also decrease the polymer molar mass \cite{Suh1990}. For more in-depth information on the production, recovery and properties of xanthan, the review by \textcite{GarciaOchoa2000} is recommended to the reader.

Xanthan is approved for food use in the European Union \cite{EURegulation11292011}. The thickening properties of xanthan in foods have been examined. Nowadays, it is used routinely in a wide array of foodstuffs such as beer (foam stabilization), cheese (syneresis inhibition), ice cream (stabilization, crystallization control), mayonnaise and salad dressings (emulsifier), sauces (thickener) and syrups (pseudoplasticity), but also in industrial applications such as explosives (gelling agent), textile dyeing (pseudoplasticity) and water clarification (flocculant) \cite{Sandvik1977, Becker1998, Ma1995a, Ma1995b, Sandford1983, Sutherland1993}.

\subsubsection{\SCL{}/\SHZ{}\label{subsubsec-intro-eps-commercial-shz}}
\SCL{} is produced by fungi belonging to the genus \mo{Sclerotium}, most notably \longrolf{}. It is chemically identical to \shz{} \cite{Tabata1981} which is produced by fungi of the genus \mo{Schizophyllum}, most notably \longcomm{} \cite{Kikumoto1970}. In this section, \enquote{\scl{}} is used synonymously with \enquote{\shz{}}.

\nomenclature[chem_DMSO]{DMSO}{dimethylsulphoxide}
\nomenclature[formula_Da]{Da}{Dalton, used as \si{\gram\per\mole}}
\SCL{} is an unbranched homopolymer of \glc{} consisting of β-1,3-links with single β-1,6-linked \glc{} residues per three \glc{} units, on average \cite{Kikumoto1971, Tabata1981, Rinaudo1982}. The polymer can be dissolved in water or DMSO and molar masses are usually on the order of \SIrange[retain-unity-mantissa = false]{1E5}{1E7}{\dalton} \cite{Farina2001, PatentEP_2675866_B1, PatentUS_4950749}. A peculiar feature of \scl{} is the formation of very stable triple helices in aqueous solutions in the pH value range of \numrange{2}{12} \cite{Yanaki1981, Bluhm1982, Yanaki1983a, Sato1983, Farina2001, PatentUS_20040265977}. Solutions are highly viscous (several \si{\pascal\second} \cite{PatentEP_2675866_B1}), shear-thinning and quickly regain their viscosity after shearing \cite{PatentDE_4012238}. Also, the solutions are stable at high salt concentrations of up to \SI{350}{\gram\per\kilo\gram} \cite{PatentEP_2675866_B1} and temperatures of up to \SIdC{135} \cite{Yanaki1985, Farina2001, PatentUS_20040265977}.

\SCL{} is used in the production of crude oil and in cosmetics. The recovery factor of oil reservoirs is usually in the range of \SIrange{30}{40}{\percent}. Drilling fluids with \scl{} can enhance the recovery factor of oil reservoirs to \SIpct{50} \cite{Davison1982, webBASF, webWintershall}. As an ingredient in cosmetics, \scl{} serves as a moisturizer to relieve dry skin conditions, atopic diseases and itching \cite{PatentAppUS20080160043}. The use as part of a drug delivery system has also been researched \cite{Coviello2005} or patented \cite{PatentUS_5215752}. For further information, the recently published review on the production and industrial applications of β-1,3-glucans by \textcite{Zhu2016} and a review on the biosynthesis, production and industrial applications of \scl{} by \textcite{Schmid2011} are recommended.

