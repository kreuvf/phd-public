\section{Outlook\label{sec-lch-eps-outlook}}
\subsection{Expansion of Analytics}
The \lch{} contained \acet{} at a concentration of \SIgpl{4.6} (externally supplied value) or \SIgpl{9.7} (custom analysis). In the medium SMLCH, this would equal \SIrange{1.4}{2.9}{\gram\per\litre} \acet{}. Under the impression of the aforementioned different phenotypes of \strain{} (see \vref{subsubsec-lch-eps-disc-lch-pf-ph-shift}), the quantification of \acet{}, but also \fora{} would give new insights into the metabolism of \strain{}, which could be leveraged for further process optimizations.

The main drawback of the PMP-based HPLC-MS method is the need to remove all monomeric aldoses for the most exact analysis of the \amc{} of the fermentation samples, with and without \lch{}. The specificity for aldehydes is another, but a far less severe limitation. For single fermentations and, thus, low sample numbers, the more laborious method of precipitating the \eps{} of single samples and dissolving the precipitate in water becomes feasible.

For high-throughput applications, the following adaptations could be investigated:
\begin{itemize}
	\item Enhance gel filtration separation characteristics by sample dilution or higher column volumes.
	\item Downscaling the precipitation and re-dissolution process.
	\item Selective aldehyde removal via unspecific oxidation \cite{Hult2007, Odebunmi2008}.
\end{itemize}

The gel filtration plates used were run with \SIul{30} of sample. Scaling up the column volume and/or diluting the sample to make the high small molecule load manageable, could solve the issues at hand with the least amount of changes to the overall process. Unfortunately, the gel filtration plates are not available in a deep-well format by default, so that diluting and numerous parallel runs of the diluted sample would be needed. The final sample would need to be concentrated, probably by another gel filtration. Alternatively, small molecules could also be removed via 96-well microdialysis \cite{Banker2003}.

While precipitation at \SIml{1} scale should not pose any problems, the re-dissolution is expected to be challenging. Although the dissolution does not need to be perfect for the hydrolysis, each well should be homogeneous and given that some polymers can be difficult to redissolve after precipitation, considerable effort might be needed to get this approach to work reliably.

The major downside of any reactive method is the introduction of additional small molecules. If the reaction does not yield gaseous or solid products, the number of small molecules most likely cannot be reduced. If these products do not interfere with hydrolysis and derivatization, the relatively hydrophobic PMP-derivates could be selectively adsorbed, washed and desorbed prior to HPLC-MS analysis removing all less hydrophobic contaminants. The inverse process---adsorbing all hydrophilic molecules---is conceivable as well. Since side reactions could also affect the polymers, this approach would require extensive work to be reliable.

\subsection{Process Optimization}
The conversion of \lch{} to \eps{}---the grand goal of this work---must be as efficient as possible. In order to achieve the highest possible efficiency, not only the fermentation process itself needs to be improved from a process engineering point of view, but also the strain, the pre-treatment of the substrates used and the purification of the final product.

\subsubsection{Pre-Treatment of \LCH{}}
The fewer interfering substances are in the \lch{}, the better. If the \lch{} contained less inhibitors many of the challenges faced in this work would be alleviated. While appearing as very promising, ionic liquids were not employed in this work for their prohibitive costs. But, recently, low-cost ionic liquids have been developed enabling the complete removal of hemicelluloses and removal of over \SIpct{80} of the lignin. \cite{Gschwend2017} Using these ionic liquids, the resulting \lch{} will cause considerably less issues during fermentation and downstream processing.

\subsubsection{Fermentation}
The \SIl{7} fed-batch fermentation must be repeated using proper equipment to prevent contamination and be able to gather data on the fermentation of pure \strain{}. Further fermentations, probably in the form of \enquote{scout fermentations} at \SIml{500} scale, could give further insights into the behaviour of \strain{}. Parameters for optimization include: temperature, pH value (constant, shifting), dissolved oxygen concentration, impeller design, stirrer speed, co-fermentation and the medium (list not exhaustive). Especially impeller design and stirrer speed strongly influence power consumption and can have beneficial or detrimental effects on microorganism growth and/or \eps{} quality.

Generally, anti-foam should not be used, if not absolutely necessary \cite{Routledge2012}. In the fermentations carried out for this work, foaming was not an issue and anti-foam was mainly used to avoid foam from entering the sensitive gas analytics. Still, anti-foam was used and found to have unexplained effects on the apparent dissolved oxygen levels in the \lch{} \SIml{500} fermentations (see \vref{subsubsec-lch-eps-disc-lch-pf-antifoam}). These hitherto unreported effects of anti-foam addition to the \lch{} fermentations would need to be investigated thoroughly in the case of anti-foam use. Further unexplored effects on the process as a whole, including downstream processing, cannot be ruled out. The added complexity could be avoided in a large scale fermentation by employing foam centrifuges. At the scales available, the aforementioned cable ties \cite{Riedel2011} work satisfactorily. An alternative could be using metal foam breakers which are attached to the same shaft as the stirrer.

\subsubsection{Product Purification\label{subsubsec-lch-eps-outlook-purification}}
Purified precipitated \eps{}s of block 2 retained the brown colour of the \lch{}. Therefore, the purification of the product needs to be improved. Three starting points for such enhancements are:
\begin{itemize}
	\item Adaptation of the cross-flow filtration process, e.g. use of different membranes and buffers.
	\item Sugaring-out lignin and inhibitors directly in the \lch{} \cite{Wang2008, PatentAppUS20090090894A1, Dhamole2010}.
	\item Genetic engineering of \strain{} to degrade lignin. Outlined in more detail in \vref{subsubsec-lch-eps-outlook-strain-lignin}.
\end{itemize}
Cross-flow filtration was carried out using Hydrosart membranes with nominal molar mass cut-offs of \SIkD{10}, \SIkD{100} and \SIum{0.45} and was only used to remove small molecules by diafiltrating with ultra-pure water. A more elaborate process could separate lignin from \eps{} by using different solvent conditions to affect the hydrodynamic radii of both. One such process might look like the following and, of course, depends on the unknown physico-chemical characteristics of lignin and \eps{}:
\begin{enumerate}
	\item Diafiltration with ultra-pure water to remove all small non-target molecules, \eps{} and lignin of medium size or greater are retained. The maximum membrane pore size is \SIkD{100}.
	\item Diafiltration to exchange the solvent with one with low \eps{} solubility (and---ideally---high lignin solubility) to facilitate the compression of the \eps{} chains, while the more bulky lignin molecules do not change considerably or even relax. The maximum membrane pore size for this step is \SIkD{10}, so that neither \eps{} nor lignin are removed.
	\item Selective removal of compressed \eps{} chains and retention of lignin. The membrane pore size must be carefully chosen, possibly a \SIkD{100} membrane proves worthwhile. The permeate contains virtually lignin-free \eps{}.
	\item Diafiltration to exchange the solvent with one with good \eps{} solubility---most likely ultra-pure water or a slightly buffered solution---to relax the \eps{} chains again. The maximum membrane pore size for this step is \SIkD{10}, so that no \eps{} is removed.
	\item Concentration of the relaxed polymer solution with the same membrane as before to reduce the volume and, in turn, the volume for precipitation.
\end{enumerate}
The corresponding experiments regarding the hydrodynamic radius could be carried out using SEC-MALLS using the same conditions as during the cross-flow filtration.

A relatively recent method, called \enquote{sugaring-out}, was published by \textcite{Wang2008} and could be used to selectively remove lignin and inhibitors from \lch{}s. Initially, the process was studied with an acetonitrile-water solution and syringic acid, \fur{}, \textit{para}-coumaric acid, ferulic acid and \hmf{}. Upon the introduction of monomeric \glc{} or \xyl{} at concentrations of \SIgpl{15} and \SIgpl{25}, respectively, at \SIdC{1}, two phases formed. Most of the sugar was retained in the bottom phase and acetonitrile concentrations in the top phase were greater than \SI{90}{\percent}, the concentrations in the bottom phase were between \SIrange{16}{26}{\percent}. The process was already developed further in \cite{PatentAppUS20090090894A1} and mentions applications suitable for \lch{} detoxification: \enquote{The extraction system [...] can be used [...] to extract inhibitors and by-products, for instance organic acids, plant phenolics, furfural and 5-hydroxymethylfurfural (HMT), from fermentation broths and biomass hydrolysates [..]}. Additionally, a 1:1 (mass) mix of \glc{} and \xyl{} showed synergistic effects. The use of a sugared-out bottom phase for fermentation is described as well.

Since some acetonitrile will always remain in the bottom phase, the microorganism used must be able to consume it. Acetonitrile consumption has been shown for several genera \cite{Li2007}, including \mo{Arthrobacter} \cite{Asano1982a, Asano1982b}, \mo{Bacillus} \cite{Graham2000}, \mo{Brevibacterium} \cite{Thiery1986}, \mo{Candida} \cite{Dias2001}, \mo {Chromobacterium} \cite{Chapatwala1990}, \mo{Geotrichum} \cite{Rezende2004}, \mo{Klebsiella} \cite{Kao2006}, \mo{Kluyveromyces} \cite{Prasad2005}, \mo{Nocardia} \cite{DiGeronimo1976, Bhalla2005}, \mo{Pseudomonas} \cite{Nawaz1989, Chapatwala1990, Babu1995} and \mo{Rhodococcus} \cite{Watanabe1987, Acharya1997, Langdahl1996}. Therefore, chances are good that \strain{} can utilize acetonitrile and using the novel method of sugaring-out for \lch{} detoxification might pose a feasible alternative to the other ideas presented here.

\subsection{Strain Engineering}
\subsubsection{Molecular Characterization}
\nomenclature[latabbr_RAST]{RAST}{Rapid Annotations using Subsystems Technology}
The strain used, \strain{}, has not been characterized on a molecular level yet. Using RAST \cite{Aziz2008, Overbeek2014, Brettin2015}, access to the whole genome sequence would allow to quickly find known genes and clusters thought to be responsible for \eps{} production and detoxification of inhibitors. Investing more time and effort, the complete genetic basis including regulation and export of the \eps{} production of \strain{} and possible quorum sensing mechanisms could be elucidated. If necessary, probably to expand the substrate spectrum, additional tolerance genes might be introduced. On the other hand, known production strains are more genetically accessible and are easier regulation-wise. Therefore, it might be a viable alternative to confer the \eps{} production capabilities and inhibitor tolerance mechanisms to a well-described host instead.

\subsubsection{Lignin Degradation\label{subsubsec-lch-eps-outlook-strain-lignin}}
Residual lignin degraded the quality of the final product and could not be removed completely by the approach used. In addition to the two methods suggested above (see \vref{subsubsec-lch-eps-outlook-purification}), the degradation of lignin may be assumed by \strain{} through genetic engineering. By using laccases and peroxidases, the strain could be enabled to degrade lignin \cite{Xu2014}. Ideally, all the lignin would be degraded below a certain maximum molar mass, such that high-purity \eps{} could be produced from the fermentation broth by one centrifugation, one cross-flow filtration/diafiltration and one precipitation step.

\subsection{Product Characterization}
In order for the product to be used, it must be studied in more detail. Using relatively pure \eps{}, re-solubilization studies should be performed to determine the effect different purification steps have on the solubility of the polymer purification, e.g. precipitation, drying and milling. Highly pure and easily soluble \eps{} should be used for more detailed analyses on the monomeric composition and possible modifications such as acetylation or pyruvylation. The ketose content and different ketoses must also be studied as only around \SIpct{40} of the mass used for monomer analysis were also found as monomers.

\nomenclature[latabbr_FFF]{FFF}{field-flow fractionation}
While the monomeric composition allows to draw conclusions on the \eps{}, the repeating unit and types of linkages need to be elucidated for a deeper understanding of the molecular basis of its properties. Accurate molar mass determinations can be facilitated by the determination of the $dn/dc$ value followed by SEC-MALLS of known concentrations of the \eps{}. An alternative for too large polymers is field flow fractionation. Using SEC-MALLS or FFF-MALLS, super-structures could be discovered as well.

For possible applications, an extensive rheological characterization of the product would be beneficial. Since no such experiments have been carried out, studies will have to start from scratch: dynamic viscosities of concentration series of the polymer in ultra-pure water and \SIpct{1} \ce{KCl}, shear-stability and thixotropic behaviour, temperature and pH stability. If gels are formed at high concentrations or with certain ions, then the gels need to be characterized as well. If the properties suggest use in a certain field or for a certain application, further studies specific to that field or application would follow.

