\section{High-Throughput Screening for Inhibitor Tolerance\label{sec-inh-hts}}
% SK4, p. 76 ff.
The growth of the strains on plates Xyl1 and Xyl2 was screened in 96-well format by incubation in \SIml{1.0} of SM18 P30S and one inhibitor at \SIgpl{2.0}: \fur{}, \hmf{}, \van{}, \acet{}, \fora{} and \laev{}. As a reference, cultures were grown on medium without inhibitors as well. The precultures were inoculated from the corresponding cryostocks in \SIml{1.0} of SM18 P30S without inhibitors and incubated for \SIh{48} at \SIdC{30} and \SIrpm{1000}. All main culture plates were inoculated from the same preculture plates by transferring \SIul{10} of the preculture to \SIul{990} of the medium. The main cultures were incubated for \SIh{48} at \SIdC{30} and \SIrpm{1000}. At the end of the incubation, $D_{600}$ of each well was determined to assess microbial growth in the presence of one inhibitor in relation to the growth without inhibitors.

\nomenclature[latabbr_ISp]{ISp}{short for plate IS1r2pmp}
\nomenclature[latabbr_ISr]{ISr}{short for plate IS1r2rez}
\nomenclature[latabbr_IS1r2pmp]{IS1r2pmp}{plate of the inhibitor screening 1, round 2 for PMP analysis}
\nomenclature[latabbr_IS1r2rez]{IS1r2rez}{plate of the inhibitor screening 1, round 2 for Rezex analysis (acids)}
For each inhibitor, \num{27} or \num{28} of the best-growing strains of this screening were gathered in two new plates designated as \enquote{ISp} and \enquote{ISr}\footnote{
	The original names were \enquote{IS1r2pmp} for \enquote{ISp} and \enquote{IS1r2rez} for \enquote{ISr} and were shorthand notations, too. The original names were used in the laboratory notebook, files and on 96-well plates. The names correspond to \enquote{\textbf{I}nhibitor \textbf{S}creening \textbf{1}, \textbf{r}ound \textbf{2}} and the means for the determination of the inhibitor concentrations: \enquote{pmp} for all aldehydes (\fur{}, \hmf{} and \van{}) and \enquote{rez} for the acids. See also \vref{subsec-inh-acid} and \vref{subsec-inh-ald}. \enquote{round 2} was the internal name used for the high-content screening with inhibitors (see \vref{sec-inh-hcs}). The even shorter shorthand names were chosen for convenience and their brevity.
}
and were used for the next step---the high-content screening on inhibitors (see \vref{sec-inh-hcs}).

\subsection{Controls \& Deviations}
\nomenclature[formula_pKa]{$pK_a$}{cologarithm of $K_a$}
\nomenclature[formula_Ka]{$K_a$}{dissociation constant of an acidic group}
The introduction of \SIgpl{2.0} acid---be it acetic, formic or laevulinic---would have had an impact on the medium pH value or, at least, on the remaining buffer capacity. In order to compensate for the introduction of acid, the three acid inhibitors were prepared as stock solutions with \SIgpl{100} acid, which were neutralized with \SIM{10} sodium hydroxide. The final pH values were \numlist{7;11;12} for \acet{}, \fora{} and \laev{}, respectively. Since the ideal buffering pH value of a weak acid is at its $pK_a$ even striving for a neutral pH easily resulted in excess \ce{OH-} which, in turn, was responsible for seemingly\footnote{The concentration of \ce{OH-} at the pH values 11 and 12 is \SImM{1} and \SImM{10}, respectively. The buffer concentration of SM18 P30S was \SImM{48} and only the 50th part of the medium was made up of the neutralized acids, so that the buffer easily absorbed the excess \ce{OH-} ions.} dramatically high pH values. The medium contained some buffering agents which kept the medium at the desired pH value. The initial pH value of the preculture medium was \num{6.72}, not \num{7.0}.

Due to a calculation error, the concentration of \ce{MgSO4*7H2O} was \SIgpl{2.66}, twice the concentration given for slime media (see \vref{subsec-slime-media}).

Judging with bare eyes, strains \xyli{F2} and \xyli{F4} did not seem to grow in the preculture. In the main culture, however, both grew. The negative control, \xyli{E12}, did not grow in any culture. In the reference main culture, strains \xyli{A4}, \xyli{E11}, \xyli{F11}, \xyli{G4}, \xyli{G6}, \xyli{G8}, \xyli{G10} and \xylj{C1} did not grow. Therefore, the calculations of this screening step could not be carried out with these strains and, thus, they were not considered for subsequent analyses.

\subsection{Inhibitor Tolerance\label{subsec-inh-hts-results}}
\begin{pycode}
import re
import copy
import numpy as np

# inh_tol[ ][ ][ ][ ]
#         |  |  |  ^- coordinate (number, x-axis)
#         |  |  ^---- coordinate (letter, y-axis)
#         |  ^------- plate
#         ^---------- category
#
# Category: 0..6 = reference, furfural, hydroxymethylfurfural,
#                  vanillin, acetic acid, formic acid,
#                  laevulinic acid
# plate: 0 = Xyl1, 1 = Xyl2
# Coordinate (letter): 0...7 = A..H
# Coordinate (number): 0..11 = 1..12
# First value: use flag
#    0: Do not use this value for calculation at all
#    1: Use this value for calculation as a normal value
#    2: Use this value for calculation of background intensity only
# Second value: raw attenuance value (float)
inh_tol = [
    [
        [
            [
                [0, ""] for coord_number in range(12)
            ]
             for coord_letter in range(8)
        ]
        for plate in range(2)
    ]
    for category in range(7)
]

# Define categories for automated conversion of filename to index and back
category = dict()
category['ref'] = 0
category['fur'] = 1
category['hmf'] = 2
category['van'] = 3
category['acet'] = 4
category['form'] = 5
category['laev'] = 6
category[0] = 'Reference'
category[1] = 'Furfural'
category[2] = 'Hydroxymethylfurfural'
category[3] = 'Vanillin'
category[4] = 'Acetic acid'
category[5] = 'Formic acid'
category[6] = 'Laevulinic acid'
shortcategory = dict()
shortcategory[0] = 'Ref.'
shortcategory[1] = 'Fur.'
shortcategory[2] = 'HMF'
shortcategory[3] = 'Van.'
shortcategory[4] = 'Acet.'
shortcategory[5] = 'Form.'
shortcategory[6] = 'Laev.'

# Generate list of input files
# First/second value: filename
# Third/fourth value: f: full plate; t: top half; b: bottom half
#     necessary to transform measured wells to original wells
filelist = [
    ['ref1.txt', 'ref2.txt', 'f', 'b'],
    ['fur1.txt', 'fur2.txt', 'f', 'b'],
    ['hmf1.txt', 'hmf2.txt', 'f', 't'],
    ['van1.txt', 'van2.txt', 'f', 'b'],
    ['acet1.txt', 'acet2.txt', 'f', 'b'],
    ['form1.txt', 'form2.txt', 'f', 't'],
    ['laev1.txt', 'laev2.txt', 'f', 't']
]
dir = 'data/lch-eps/inh-tol/'

# Read files and populate inh_tol (this one is huge when printed in the shell)
for item in filelist:
    for plate in range(2):
        with open(dir + item[plate], 'r') as f:
            f.readline() # skip first line
            cat_in = category[re.match('[a-z]{3,4}', item[plate]).group(0)]
            for coord_letter in range(8):
                for coord_number in range(12):
                    line = f.readline()
                    inh_tol[cat_in][plate][coord_letter][coord_number] = [1, float(re.match('[^\t]+\t[^\t]+\t([0-9]\.[0-9]+)', line).group(1))]

# Transform bottom halves to top halves
# Reduces the amount of differences we will have to deal with
# Basically: tr E-H A-D
# only for Xyl2 plates, that means: plate = 1
for cat in (0, 1, 3, 4):
    for coord_letter in range(4,8):
        # deepcopy due to lists being mutable
        inh_tol[cat][1][coord_letter-4] = copy.deepcopy(inh_tol[cat][1][coord_letter])

# Set use flag to zero for all bottom halves
for cat in range(7):
    for coord_letter in range(4,8):
        for coord_number in range(12):
            inh_tol[cat][1][coord_letter][coord_number][0] = 0

# Set use flag for other wells:
#    7 strains did not grow in the reference --> 2
#    1 strain did not grow in the reference --> 2
#    1 empty well on Xyl1 --> 2
#    8 empty wells on Xyl2 --> 2
#    1 empty well on laev2 grew (contamination) --> 0

for cat in range(7):
    # 7 strains from Xyl1:      A4, E11, F11, G4, G6, G8, G10
    inh_tol[cat][0][0][3][0]  = 2
    inh_tol[cat][0][4][10][0] = 2
    inh_tol[cat][0][5][10][0] = 2
    inh_tol[cat][0][6][3][0]  = 2
    inh_tol[cat][0][6][5][0]  = 2
    inh_tol[cat][0][6][7][0]  = 2
    inh_tol[cat][0][6][9][0]  = 2
    # 1 strain from Xyl2:       C1
    inh_tol[cat][1][2][0][0]  = 2
    # 1 empty well on Xyl1:     E12
    inh_tol[cat][0][4][11][0] = 2
    # 8 empty wells on Xyl2:    D5, D6, D7, D8, D9, D10, D11, D12
    inh_tol[cat][1][3][4][0]  = 2
    inh_tol[cat][1][3][5][0]  = 2
    inh_tol[cat][1][3][6][0]  = 2
    inh_tol[cat][1][3][7][0]  = 2
    inh_tol[cat][1][3][8][0]  = 2
    inh_tol[cat][1][3][9][0]  = 2
    inh_tol[cat][1][3][10][0] = 2
    inh_tol[cat][1][3][11][0] = 2

# 1 contamination in laev2: D9
inh_tol[6][1][3][8][0] = 0

# Calculation of median background attenuance on a per-plate basis
# First value: average background attenuance, second value: list of all values,
# third value: inter-quartile range
bg = [[ [0.0, [], 0.0] for plate in range(2)] for cat in range(7)]

for cat in range(7):
    for plate in range(2):
        for coord_letter in range(8):
            for coord_number in range(12):
                well = inh_tol[cat][plate][coord_letter][coord_number]
                if well[0] == 2:
                    bg[cat][plate][1].append(well[1])
        bg[cat][plate][0] = round(np.median(bg[cat][plate][1]), 7)
        bg[cat][plate][2] = round(np.subtract(*np.percentile(bg[cat][plate][1], [75, 25])), 7)

# Subtract background from each plate's used (=1) values
for cat in range(7):
    for plate in range(2):
        for coord_letter in range(8):
            for coord_number in range(12):
                # Use mutability...
                well = inh_tol[cat][plate][coord_letter][coord_number]
                if well[0] == 1:
                    well[1] = round(well[1] - bg[cat][plate][0], 7)

# Divide attenuance of inhibitor plate by attenuance of the reference plate
# Multiply with 100 to get percent values
# Reference plates remain untouched
for cat in range(1,7):
    for plate in range(2):
        for coord_letter in range(8):
            for coord_number in range(12):
                # Use mutability...
                ref = inh_tol[0][plate][coord_letter][coord_number]
                well = inh_tol[cat][plate][coord_letter][coord_number]
                if well[0] == 1:
                    #print("{0}.{1}.{2}.{3}: {4}; Ref.: {5}".format(cat, plate, coord_letter, coord_number, well[1], ref[1]))
                    well[1] = round(100 * (well[1] / ref[1]), 1)

# Generate list for stats, never use index 0
# List is generated with 7 entries for consistency's sake
# stats contains the amount of strains in a certain interval
# The intervals are:
# 0:  <   5%          no growth
# 1: >=   5%, <  20%  rudimentary growth
# 2: >=  20%, <  40%  strongly inhibited growth
# 3: >=  40%, <  60%  moderately inhibited growth
# 4: >=  60%, <  80%  slightly inhibited growth
# 5: >=  80%, < 100%  normal growth
# 6: >= 100%, < 120%  normal growth
# 7: >= 120%          overshooting growth
stats = [ [0, 0, 0, 0, 0, 0, 0, 0] for cat in range(7) ]
for cat in range(1,7):
    for plate in range(2):
        for coord_letter in range(8):
            for coord_number in range(12):
                well1 = inh_tol[cat][plate][coord_letter][coord_number]
                if well1[0] == 1:
                    if well1[1] >= 120:
                        stats[cat][7] = stats[cat][7] + 1
                    elif well1[1] >= 100:
                        stats[cat][6] = stats[cat][6] + 1
                    elif well1[1] >= 80:
                        stats[cat][5] = stats[cat][5] + 1
                    elif well1[1] >= 60:
                        stats[cat][4] = stats[cat][4] + 1
                    elif well1[1] >= 40:
                        stats[cat][3] = stats[cat][3] + 1
                    elif well1[1] >= 20:
                        stats[cat][2] = stats[cat][2] + 1
                    elif well1[1] >= 5:
                        stats[cat][1] = stats[cat][1] + 1
                    else:
                        stats[cat][0] = stats[cat][0] + 1

# Output data
# Background data
bg_attenuance = []
for cat in range(7):
    # Curly braces are escaped not by a '\' but by another curly brace!
    bg_line = "{} & \\num{{{:.4f} \\pm {:.4f}}} & \\num{{{:.4f} \\pm {:.4f}}} \\\\".format(category[cat], round(bg[cat][0][0], 4), round(bg[cat][0][2]/2.0, 4), round(bg[cat][1][0], 4), round(bg[cat][1][2]/2.0, 4))
    # Store strings
    bg_attenuance.append(bg_line)

# Distribution data
dist_data = []
for cat in range(1,7):
    dist_line = "{} & \\num{{{}}} & \\num{{{}}} & \\num{{{}}} & \\num{{{}}} & \\num{{{}}} & \\num{{{}}} & \\num{{{}}} & \\num{{{}}} \\\\".format(category[cat], stats[cat][0], stats[cat][1], stats[cat][2], stats[cat][3], stats[cat][4], stats[cat][5], stats[cat][6], stats[cat][7])
    # Store strings
    dist_data.append(dist_line)
\end{pycode}
\begin{table}
	\centering
	\caption[Definitions of Growth Classes for Evaluating Inhibitor Tolerance]{Definitions of the growth classes used in the inhibitor tolerance and \lch{} tolerance experiments. Class numbers are used when longer names would pose space problems. The descriptions are used in the running text to make comprehension easier. The intervals define relative attenuance ranges. Relative attenuances at \SInm{600} were calculated by subtracting the median attenuance of empty wells from the attenuance of every other well on a per-plate basis. The relations of each attenuance in inhibitor presence to the corresponding attenuance in the reference plate gave the relative attenuances. These relative attenuances were used for classing.\label{tbl-inh-tol-classes}}
	\begin{tabular}{lrc}
		\toprule
		{Class} & {Description} & {Interval} \\
		\hline
		\romi{}    & no growth
		& (\minus\infinity, \SIpct{5}) \\
		\romii{}   & rudimentary growth
		& [\SIpct{5}, \SIpct{20}) \\
		\romiii{}  & strongly inhibited growth
		& [\SIpct{20}, \SIpct{40}) \\
		\romiv{}   & moderately inhibited growth
		& [\SIpct{40}, \SIpct{60}) \\
		\romv{}    & slightly inhibited growth
		& [\SIpct{60}, \SIpct{80}) \\
		\romvi{}   & 
		& [\SIpct{80}, \SIpct{100}) \\
		\romvii{}  & \multirow{-2}*{normal growth}
		& [\SIpct{100}, \SIpct{120}) \\
		\romviii{} & excessive growth
		& [\SIpct{120}, +\infinity) \\
		\bottomrule
	\end{tabular}
\end{table}
\begin{figure}
	\subfloat[Tolerance towards \fur{}]{
			\label{fig-inh-tol-fur}%
			\includegraphics[width=0.45\textwidth]{fig/inh-tol_fur_600dpi.png}
	}
	\hfill
	\subfloat[Tolerance towards \hmf{}]{
			\label{fig-inh-tol-hmf}%
			\includegraphics[width=0.45\textwidth]{fig/inh-tol_hmf_600dpi.png}
	}

	\subfloat[Tolerance towards \van{}]{
			\label{fig-inh-tol-van}%
			\includegraphics[width=0.45\textwidth]{fig/inh-tol_van_600dpi.png}
	}
	\hfill
	\subfloat[Tolerance towards \acet{}]{
			\label{fig-inh-tol-acet}%
			\includegraphics[width=0.45\textwidth]{fig/inh-tol_acet_600dpi.png}
	}

	\subfloat[Tolerance towards \fora{}]{
			\label{fig-inh-tol-form}%
			\includegraphics[width=0.45\textwidth]{fig/inh-tol_form_600dpi.png}
	}
	\hfill
	\subfloat[Tolerance towards \laev{}]{
			\label{fig-inh-tol-laev}%
			\includegraphics[width=0.45\textwidth]{fig/inh-tol_laev_600dpi.png}
	}
	\caption[Tolerance Towards Six Different Inhibitors]{Tolerance towards six different inhibitors. The strains of the plates Xyl1 and Xyl2 were grown with and without an inhibitor at a concentration of \SIgpl{2.0}. After subtraction of the background attenuance, the attenuance in the presence of an inhibitor was divided by the attenuance without any inhibitor. The strains of both plates were classed into eight classes (see \vref{tbl-inh-tol-classes}) and the results plotted as bar graphs. The data are available in \vref{tbl-inh-tol-dist}.\label{fig-inh-tol-dist}}
\end{figure}
The median background attenuance at \SInm{600} was calculated from empty wells of each plate and subtracted from every other well. The median background values, inter-quartile ranges and the amount of background values used are summarized in \vref{tbl-inh-tol-bg}. In the next step, the relation of each attenuance to the corresponding attenuance in the reference plate was calculated. The resulting percentages were grouped into eight classes. The classes, their descriptive short names and their boundaries are given in \vref{tbl-inh-tol-classes}. The data are visualized in \vref{fig-inh-tol-dist}.

\textbf{Example:} The median background attenuance at \SInm{600} was \num{0.008}. The strain was part of the \fur{} and the \fora{} screenings. The raw attenuances at \SInm{600} were \num{1.116}, \num{0.892} and \num{1.345} for the reference, \fur{} and \fora{}, respectively. In the first step, the median background attenuance was subtracted yielding \num{1.108}, \num{0.884} and \num{1.337} for the reference, \fur{} and \fora{}, respectively. The relation of the attenuance of the \fur{} screening to the reference and the \fora{} screening to the reference are \SIpct{78.9} and \SIpct{121}, respectively. The result of the \fur{} screening was grouped into class \romv{} (slightly inhibited growth), while the result of the \fora{} screening was grouped into class \romviii{} (excessive growth).

The response to \textbf{\fur{}} was mixed: 25 of 127 strains did not grow at all, 50 strains were inhibited moderately at least, only 28 strains were slightly inhibited, 18 showed normal growth and six strains showed excessive growth.

Compared to \fur{}, \textbf{\hmf{}} was less inhibiting: only nine strains showed no growth, while 42 strains exhibited at least moderate inhibition, another 39 showed only slight inhibition, 31 grew normally and six strains showed excessive growth.

\textbf{\VAN{}} was the most potent inhibitor of the ones tested: 100 of the 127 strains tested or \py{"\SIpct{" + str((int(round(100*100.0/127.0,0)))) + "}"} showed no growth at all, only one strain grew normally, while 14 strains were at least moderately inhibited and twelve strains slightly inhibited only.

\textbf{\Acet{}} showed a low inhibitory potential: two strains did not grow, another two showed rudimentary growth, 24 strains were inhibited more or less, while 83 strains showed normal growth and 16 excessive growth.

\textbf{\Fora{}} inhibited microbial growth to a lesser extent than \acet{}: three strains did not grow, one strain grew rudimentarily, 20 strains were inhibited at least slightly, while 80 strains showed normal growth and 23 strains excessive growth.

No strains were susceptible to \textbf{\laev{}}: all strains grew. Six were strongly, ten moderately and 24 slightly inhibited, while 85 strains grew to their normal attenuance values and two showed excessive growth.

\subsection{Preparation of Plates for the High-Content Screening with Inhibitors\label{subsec-preparation-hcs-tol}}
\nomenclature[chem_Fur.]{Fur.}{furfural or furan-2-carbaldehyde}
\nomenclature[chem_HMF]{HMF}{hydroxymethylfurfural or 5-(hydroxymethyl)furan-2-carbaldehyde}
\nomenclature[chem_Van.]{Van.}{vanillin}
\nomenclature[chem_Acet.]{Acet.}{acetic acid}
\nomenclature[chem_Form.]{Form.}{formic acid}
\nomenclature[chem_Laev.]{Laev.}{laevulinic acid}
\begin{table}
	\centering
	\caption[Strains Appearing in the Top 27/28 of at Least Four Inhibitors]{Strains appearing in the top 27/28 of at least four inhibitors. In this table all strains in at least four top 27/28 are highlighted. Abbreviations: Fur.: \fur{}; HMF: \hmf{}; Van.: \van{}; Acet.: \acet{}; Form.: \fora{}; Laev.: \laev{}.\label{tbl-inh-tol-special-strains}}
	\begin{tabular}{l*{6}r}
		\toprule
		 & \multicolumn{6}{c}{Part of Top 27/28 of Inhibitor ... ?} \\
		\multirow{-2}*{Strain} & {Fur.} & {HMF} & {Van.} & {Acet.} & {Form.} & {Laev.} \\
		\hline
		\xyli{A10} & yes & yes & yes &  no & yes &  no \\
		\xyli{C4}  &  no & yes & yes & yes & yes &  no \\
		\xyli{C5}  &  no & yes & yes & yes & yes &  no \\
		\xyli{G5}  & yes & yes & yes & yes & yes & yes \\
		\xyli{G11} & yes & yes & yes & yes & yes &  no \\
		\xylj{A1}  &  no & yes & yes &  no & yes & yes \\
		\xylj{A6}  & yes & yes & yes & yes &  no & yes \\
		\xylj{A9}  & yes &  no & yes & yes &  no & yes \\
		\xylj{B7}  & yes & yes & yes &  no &  no & yes \\
		\xylj{B8}  & yes & yes &  no & yes & yes & yes \\
		\xylj{C4}  & yes & yes & yes & yes &  no & yes \\
		\xylj{C5}  & yes & yes & yes & yes & yes &  no \\
		\bottomrule
	\end{tabular}
\end{table}
Two 96-well plates for the high-content screening were prepared. Using the growth data, each strain's growth in the presence of each inhibitor was ranked from best to worst. The top 27 strains of \fur{}, \hmf{} and \van{} were collected in plate ISp and the top 28 strains of \acet{}, \fora{} and \laev{} were collected in plate ISr. Plate layouts are given and explained in detail in tables \ref{tbl-inh-tol-layout-isp} and \ref{tbl-inh-tol-layout-isr} on page~\pageref{tbl-inh-tol-layout-isp}. Ranks of all strains and for all inhibitors and \lch{} are given in \vref{tbl-inh-lch-tol-ranks}. As the selection process allowed each strain to appear once for every of the six inhibitors, some strains appeared more than once. Strains appearing in at least four of the six inhibitor top 27/28 are highlighted in \vref{tbl-inh-tol-special-strains}. Given the low number of strains growing in the presence of \van{}, it should be noted here that only one of these strains, \xylj{B8}, was \textit{not} among the top 27 strains of the \van{} screening.
%five inhibitors: 1.G11 (-laev), 2.A6 (-form), 2.B8 (-van), 2.C4 (-form), 2.C5 (-laev); four inhibitors: 1.A10 (-form -laev), 1.C4 (-fur -laev), 1.C5 (-fur -laev), 2.A1 (-fur -acet), 2.A9 (-hmf -form), 2.B7 (-acet -form).

