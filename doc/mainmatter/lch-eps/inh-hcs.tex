\section{High-Content Screening with Inhibitors\label{sec-inh-hcs}}
The previous round served to screen the \eps{} producers growing on \xyl{} for their ability to grow in the presence of single inhibitors or \lch{}. The next step was the high-content screening of a consolidated set of strains to have a closer look at the \eps{} concentrations and \amc{}s of the \eps{}s produced.

\SIml{1.0} SM18 P30S in 96-well plates was inoculated from the plates ISp and ISr for \SIh{48} at \SIdC{30} and \SIrpm{1000}. Then, these plates were used to inoculate new plates with \SIul{990} SM18 P30S with the respective inhibitor at a final concentration of \SIgpl{2.0} using \SIul{10} of the preculture. These plates were incubated for \SIh{48} at \SIdC{30} and \SIrpm{1000}.

The cultures were subjected to a high-throughput \eps{} purification (see page~\pageref{hteps-purification}), hydrolysis (see page~\pageref{pmp-hydrolysis}) %using \SIul{67} of \SIpct{3.2} ammonia for the neutralization of the plate ISp and \SIul{68} for the neutralization of the plate ISr
, derivatization (see page~\pageref{pmp-deriv}) and HPLC-MS analysis (see page~\pageref{pmp-hplc-ms}). Calibration standards 1 and 2, both with a TFA matrix, were used.

Since \glc{} was part of the carbon source, \glc{} assay samples were taken after centrifugation of the cultures, after gel filtration and after neutralization. The samples were diluted 1:10 with ultra-pure water. Inhibitor concentrations were assessed using an adapted PMP derivatization method (see page~\pageref{subsec-inh-ald}) or HPLC-UV detection (see page~\pageref{subsec-inh-acid}).

\subsection{Controls \& Deviations\label{subsec-inh-hcs-controls}}
The introduction of aldehydes or neutralized acids slightly shifted pH values from \num{7.0} to \num{6.9} for \fur{} and \van{} and \num{6.8} for the acids. The pH value of the medium with \hmf{} was not affected. Precultures and main cultures were incubated for \SIh{47.2} and \SIh{48.3}, respectively.

The major deviation was the stretching of the analyses over several days. Initially, all analyses were planned to be finished two days after the end of the incubation of the main cultures, which proved to be unfeasible. Since storage might have had an influence on the \eps{} concentration or \amc{}, in the following text, the time of most steps is given relative to the end of incubation.

\label{intext-inh-hcs-controls-nopellet}Five wells of ISp and one well of ISr\footnote{No pellet after centrifugation: B6, C6, E11, F7, G7 of ISp and B12 of ISr. This corresponds to the strains \xyli{F2}, \xyli{F3}, \xylj{A7} and \xylj{A8} for \fur{}, \xylj{B7} for \van{} and \xylj{A6} for \laev{}.} of the main culture appeared to be empty upon visual inspection and no pellet had formed after centrifugation. All blank or control wells\footnote{Blank or control wells: H1 to H12 of both plates and G4, G8 and G12 of ISp.} were empty. Thirtytwo wells\footnote{Low or no sedimentation of bacteria after centrifugation: A3, A6, B3, C3, D6, E6, F3, F6 and G3 of ISp and A1, A4, A7, A8, B1, C10, C12, D1, D3, D6, D8, D10, E1, E3 to E5, E10, F4, F5, F7, G1, G4 and G7 of ISr. This corresponds to the strains \xyli{F1}, \xyli{F4}, \xyli{F8} and \xyli{F9} for \fur{}, \xyli{F4}, \xyli{F8}, \xyli{F9}, \xylj{A1} and \xylj{A6} for \hmf{}, \xyli{C4}, \xyli{C5}, \xyli{F1}, \xyli{H8}, \xylj{A6}, \xylj{A7}, \xylj{A8} and \xylj{B1} for \acet{}, \xyli{C4}, \xyli{C5}, \xyli{F2}, \xyli{F4}, \xyli{F8}, \xylj{A1}, \xylj{A2}, \xylj{B10}, \xylj{C12}, \xylj{D2} and \xylj{D3} for \fora{} and \xyli{D8}, \xyli{D9}, \xyli{D10} and \xylj{A7} for \laev{}.} were still turbid after centrifugation indicating low or no sedimentation of bacteria. Pellets were found in all remaining wells.

Since the inhibitor analysis of ISr started only on the next day, \SIul{700} of the supernatants were stored at \SIdC{4} overnight in a deep well plate, which was covered with parafilm. For well F9, comparatively much of the pellet was transferred.

The aldehyde inhibitor standards were not filtered through the \SIkD{10} plate. Filtrate volumes of all wells of ISp were sufficient for the next step. For ISr, more than \SIpct{50} of the filtrate volumes were insufficient for the next step. Hence, a 1:10 dilution with ultra-pure water of all wells was made and then centrifuged. At least \SIul{155} filtrate gathered in the receiving plate for all but twelve wells\footnote{Insufficient filtrate volumes after ten-fold dilution: A1, A7, B1, C10, D3, D10, E4, E5, E10, F4, F5 and G4 of ISr. This corresponds to the strains \xyli{C4}, \xyli{C5} and \xyli{H8} for \acet{}, \xyli{C4}, \xyli{C5}, \xylj{A1}, \xylj{C12}, \xylj{D2} and \xylj{D3} for \fora{} and \xyli{D8}, \xyli{D9} and \xyli{D10} for \laev{}.}. The insufficient filtrate of the aforementioned wells was diluted 1:2 with ultra-pure water to give \SIul{160}. \SIul{155} of each well was transferred into a new 96-well plate, sealed with a mat and stored at \SIdC{4}.

Acid samples were run for \SImin{25} on an HPLC. Initially, \SImin{50} were planned for the samples to be safe from ghost peaks arising from slow eluting peaks from prior samples.%fxnote{inh-hcs: Take care of plate layout changes introduced here (see below, in the source)!} 

\GLC{} assays of the supernatants were conducted on the following day. Slightly lower volumes were transferred to the wells ISp.E6 and ISp.F6 (tested with \fur{} harbouring the strains \xyli{F8} and \xyli{F9}, respectively), because these two samples were sticky. All samples for the \glc{} assay were stored at \SIdC{4} in sealed 96-well plates.

Sugar standards for \eps{} \amc{} analysis were split between the analyses of ISp and ISr. Sugar standard 1 was analysed with ISp, sugar standard 2 was analysed with ISr.

Two days after the end of incubation, the stored supernatants were centrifuged for \SImin{15} at \SIG{3710} and \SIdC{20} again. The \SIum{1.0} glass filtration of both plates was conducted on the same day. The supernatants of ten wells\footnote{Glass filtration incomplete: A3, B3, C3, D6, E6 and F6 of ISp and D1, D6, E1 and G1 of ISr. This corresponds to the strains \xyli{F4}, \xyli{F8} and \xyli{F9} for \fur{}, \xyli{F4}, \xyli{F8} and \xyli{F9} for \hmf{}, \xyli{F1} for \acet{} and \xyli{F2}, \xyli{F4} and \xyli{F8} for \fora{}.} were not filtered completely. There was at least \SIul{70} of filtrate in every well. The filtered supernatants of ISr were stored at \SIdC{4} again, while the filtered supernatants of ISp were used on the same day for \eps{} \amc{} analysis.

The sealing mat of ISp during hydrolysis was fastened too firmly. As a result, the mat was destroyed and some volume was missing in the following wells after hydrolysis: A3 to A9, G1. Neutralization, however, was successful in all wells.

Four days after the end of incubation, the supernatants of ISr which were stored after glass-filtration were subjected to \eps{} \amc{} analysis. Gel filtration plates were washed and equilibrated only \SIh{3} prior to use and not overnight as stated in \vref{hteps-purification}. Since the sugar standards were put into wells H1 to H10, the original contents of H9 and H10 (uninoculated media) were discarded and replaced by two of the standards.

Five days after the end of incubation, the remaining \glc{} assays were run: one plate after gel filtration, one plate after neutralization for each, ISp and ISr.

%Medium: SM18 P30S
%Allgemein: Da Glucose Teil der C-Quelle: nach Gelfiltration 5 µl 1:10 in ddH2O verdünnt f. Glucoseassay; nach Neutralisation 5 µl 1:10 in ddH2O verdünnt f. Glucoseassay; f. Glucoseverbrauch auch Glucoseassays VOR Gelfiltration
% SK5, p. 56+
%pH-Werte der fertigen Medien:
%	SM18 P30S f. Aldehyde: 7,00; SM18 P30S + Ameisensäure: 6,78; SM18 P30S + Essigsäure: 6,79; SM18 P30S + Lävulinsäure: 6,82; SM18 P30S + HMF: 6,99; SM18 P30S + Vanillin: 6,91; SM18 P30S + Furfural: 6,87
% Vorkultur angeimpft SM18 P30S o. I.: 2015-02-25, 15:46 Uhr; Hauptkultur angeimpft: 2015-02-27, 14:52 Uhr (47 h 10 min); Inkubation bis: 2015-03-01, 15:10 Uhr (48 h 20 min)

%2015-03-01:
%n. Zentrifugation @3710 g f. 30 min, 20 °C:
%	Anhand Bilder bestätigt:
%		Pellet:
%			PMP: A: 1, 2, 4, 5, 7-12; B: 1, 2, 4, 5, 7-12; C: 1, 2, 4, 5, 7-12; D: 1-5, 7-12; E: 1-5, 7-10, 12; F: 1, 2, 4, 5, 8-12; G: 1, 2, 5, 6, 9-11
%			Rezex: A: 2, 3, 5, 6, 9-12; B: 2-11; C: 1-9, 11; D: 2, 4, 5, 7, 9, 11, 12; E: 2, 6-9, 11, 12; F: 1-3, 6, 8-12; G: 2, 3, 5, 6, 8-12
%		leer (Augenschein):
%			PMP: B: 6; C: 6; E: 11; F: 7; G: 4, 7, 8, 12 (4, 8, 12 = Negativkontrollen); H: 1-12 (ohnehin nicht benutzt)
%			Rezex: B: 12; H: 1-12 (ohnehin nicht benutzt oder Leerkontrolle)
%		trüb:
%			PMP: A: 3, 6; B: 3; C: 3; D: 6; E: 6; F: 3, 6; G: 3
%			Rezex: A: 1, 4, 7, 8; B: 1; C: 10, 12; D: 1, 3, 6, 8, 10; E: 1, 3-5, 10; F: 4, 5, 7; G: 1, 4, 7
%
%ISr-Inhibitorenanalytik nicht am gleichen Tag startbar aufgrund eines (fremdverschuldeten!) Defektes
%	* Überstände aus Zentrifugation über Nacht bei 4 °C lagern; 700 µl Überstand; Deep-Well-Platte mit Parafilm abgedeckt; F9: viel Pellet aufgesaugt
%
%10-kDa-Filtration: Inhibitorstandards NICHT filtriert
%* Vormischung f. PMP-PLatte: 120 µl ddH2O, 20 µl Überstand, 150 µl ACN; 30 min@1200 x g, 20 °C; für PMP okay, meiste nahezu komplett durch --> genug f. Derivatisierung; Rezex: viele (>50\%) n. ausreichend filtriert; Rezex: 1:10 mit ddH2O verdünnt, nochmal zentrifugiert; 155 µl bei meisten Wells, überführt in neue Platte; 4 °C, über Nacht aufbewahrt unter HPLC-Matte; restliche Wells (A1, A7, B1, C10, D3, D10, E4, E5, E10, F4, F5, G4): 80 µl ddH2O in Eppi, 80 µl Filtrat gemischt (= 1:20 verdünnt!)
% nach PMP-Derivatisierung: Aufkonzentrierung in folgenden Wells möglich: A7-10, B1, B12, C12, E12, G1, H2, H3, H5, H6, H9, H10, H12; folgende Wells blutrot (viel Vanillin übrig?): F8, F12; weinrot: E4, F4
%
%2015-03-02:
%Abweichung vom Plattenlayout ISr: H1: 5 g/l Ameisensäure + 5 g/l %Essigsäure; H2: 5 g/l Lävulinsäure f. Test; f. echte Messungen: 170 µl mit 5 g/l Ameisensäure und 5 g/l Lävulinsäure in H3, 170 µl mit 5 g/l Essigsäure in H4
%Abweichungen: Messungen der Proben nur 25 min statt 50 min!
%
%Glucose-Assays von ISp, ISr (beide n. Zentr.), PLatte mit Standards + MR-Proben; PMP: E6 und F6 weniger Volumen, da schleimig; PMP-Platte um 180° gedreht
%
%Vorquellen der GPC-Platte f. ISp-Hydrolyse am folgenden Tag
%
%Zuckerstandard 1 bei ISp gelaufen
%
%2015-03-03:
%Überstände nochmal 15min@3710xg, 20°C
%beide Platten über Glasfilterplatten gefiltert; ISr wieder @4 °C
%	* nicht vollständig filtriert PMP: A3, B3, C3, D6, E6, F6
%	* nicht vollständig filtriert Rez: D1, D6, E1, G1
%	* min. 70 µl bei allen.
%Verarbeitung ISp
%Hydrolyse ab 12:50 Uhr; Matte zu stark angedrückt, Matte zerstört; weniger Volumen n. Hydrolyse bei A3-9, G1
%neutralisiert mit 67 µl
%NEutralisation bei allen Wells OK
%Derivatisierung ab 17:44 Uhr; Verlust n. gemessen
%HPLC-MS-Lauf ab 20:23 Uhr
%
%2015-03-04:
%nichts Relevantes
%
%2015-03-05:
%Gelfiltrationsplatten 3 h vor Benutzung quellen gelassen
%Verarbeitung ISr
%Hydrolyse ab 13:18 Uhr
%Neutralisation mit 68 µl
%Standards in Wells H1 bis H10, dafür mussten Originale von H9 und H10 weichen
%Derivatisierung ab 16:25 Uhr
%HPLC-MS ab 19:41 Uhr
%
%2015-03-06:
%restliche Glucoseassays (4 Platten + 1/2 Platten Standards)

\subsection{Inhibitor Degradation}
\begin{table}
	\centering
	\sisetup{
		table-number-alignment = center,
		table-text-alignment = center,
		table-figures-integer = 1,
		table-figures-decimal = 2,
		table-format = 1.2
	}
	\caption[Summary Statistics of the Inhibitor Concentrations After \SIh{48} Incubation]{Summary statistics of the inhibitor concentrations after \SIh{48} incubation. The plates ISp and ISr were incubated with \SIml{1.0} SM18 P30S with \SIgpl{2.00} of inhibitor for \SIh{48} at \SIdC{30} and \SIrpm{1000}. Afterwards, the inhibitor concentrations were determined using PMP derivatization and HPLC-MS analysis (\fur{}, \hmf{} and \van{}) or HPLC-UV analysis (acids). In this table, summary statistics of each single inhibitor are given. For the calculation, negative concentrations were set to zero and the concentrations of the aldehyde inhibitors were clipped at \SIgpl{2.00}. Since acid production might have occurred, acid inhibitor concentrations exceeding \SIgpl{2.00} were not changed. The results were rounded to two decimals. The data are visualized in \vref{fig-inh-hcs-inh-stats}. The complete raw data are given in \vref{tbl-inh-hcs-inh-isp-full} and \vref{tbl-inh-hcs-inh-isr-full}.\label{tbl-inh-hcs-inh-sum}}
	\begin{tabular}{l*{3}{S}}
		\toprule
		 & \multicolumn{3}{c}{Inhibitor concentration in \si{\gram\per\litre} after \SIh{48}} \\
		\multirow{-2}*{Inhibitor} & {Lower quartile} & {Median} & {Upper quartile} \\
		\hline
		\TablesafeInputIfFileExists{data/lch-eps/inh-hcs/inhibitors-stats.tex}{}{\fxfatal{File not found: data/lch-eps/inh-hcs/inhibitors-stats.tex}}
		\bottomrule
	\end{tabular}
\end{table}
\begin{figure}
	\begin{center}
		\includegraphics[width=0.45\textwidth]{fig/inh-hcs_inh-stats_600dpi.png}
		\caption[Median Inhibitor Concentrations After \SIh{48} Incubation]{Median inhibitor concentrations after \SIh{48} incubation. The plates ISp and ISr were incubated with \SIml{1.0} SM18 P30S with \SIgpl{2.00} of inhibitor for \SIh{48} at \SIdC{30} and \SIrpm{1000}. Afterwards, the inhibitor concentrations were determined using PMP derivatization and HPLC-MS analysis (\fur{}, \hmf{} and \van{}) or HPLC-UV analysis (acids). In this figure, the medians (bars) and the lower and upper quartiles (error bars) of the inhibitor concentration after incubation are depicted. For the calculation, negative concentrations were set to zero and the concentrations of the aldehyde inhibitors were clipped at \SIgpl{2.00}. Since acid production might have occurred, acid inhibitor concentrations exceeding \SIgpl{2.00} were not changed. The results were rounded to two decimals. Summarized data are available in \vref{tbl-inh-hcs-inh-sum}. The complete raw data are given in \vref{tbl-inh-hcs-inh-isp-full} and \vref{tbl-inh-hcs-inh-isr-full}. Abbreviations: Fur.: \fur{}; HMF: \hmf{}; Van.: \van{}; Acet.: \acet{}; Form.: \fora{}; Laev.: \laev{}.\label{fig-inh-hcs-inh-stats}}
	\end{center}
\end{figure}
Generally, all inhibitors with the exception of \laev{} were degraded. The results are summarized in \vref{tbl-inh-hcs-inh-sum} and the summarized data visualized in \vref{fig-inh-hcs-inh-stats}. Full raw data are given in \vref{tbl-inh-hcs-inh-isp-full} for ISp and \vref{tbl-inh-hcs-inh-isr-full} for ISr. The strains which did not appear to grow did not show inhibitor degradation, except for F7 (more than \SIpct{50}) and G7 (around \SIpct{30}) of ISp, both tested with \fur{} harbouring the strains \xylj{A7} and \xylj{A8}, respectively. The residual inhibitor concentrations of thirtytwo wells\footnote{No inhibitor degradation: A12, B10, B12, C2, E10 and G2 of ISp; A6, A11, B6, B9, C6, C7, C9, C10, C12, D9, D10, D12, E6, E7, E9, E10, E12, F6, F8, F9, F10, F12, G8, G9, G11 and G12 of ISr. This corresponds to the strains \xyli{C10} and \xyli{D12} for \hmf{}, \xyli{G5}, \xyli{G11}, \xylj{C5} and \xylj{C7} for \van{}, \xyli{D12}, \xyli{E1}, \xyli{F6}, \xyli{G5}, \xylj{A2}, \xylj{A5}, \xylj{C4} and \xylj{C5} for \acet{}, \xyli{E2} for \fora{} and \xyli{A5}, \xyli{A6}, \xyli{A7}, \xyli{A8}, \xyli{A9}, \xyli{A11}, \xyli{D8}, \xyli{D9}, \xyli{D10}, \xyli{D12}, \xyli{G5}, \xylj{A1}, \xylj{A7}, \xylj{A9}, \xylj{B7}, \xylj{B8}, \xylj{C4} for \laev{}.} were not indicative of inhibitor degradation. Some of the residual inhibitor concentrations exceeded the initial concentration, especially when the inhibitor was an acid.

\paragraph{Performance of Highlighted Strains}
\begin{table}
	\centering
	\caption[Inhibitor Degradation of Selected Strains]{Summary of the inhibitor degradation of the previously highlighted strains (see \vref{subsec-preparation-hcs-tol}). No values reported here were clipped at \SIgpl{2.0}. Abbreviations: Fur.: \fur{}; HMF: \hmf{}; Van.: \van{}; Acet.: \acet{}; Form.: \fora{}; Laev.: \laev{}.; n.t.: not tested.\label{tbl-inh-hcs-highlights}}
	\begin{tabular}{l*{6}{r}}
		\toprule
		 & \multicolumn{6}{c}{Inhibitor concentration in \si{\gram\per\litre} after \SIh{48}} \\
		\multirow{-2}*{Strain} & {Fur.} & {HMF} & {Van.} & {Acet.} & {Form.} & {Laev.} \\
		\hline
		\TablesafeInputIfFileExists{data/lch-eps/inh-hcs/tbl-inh-hcs-highlights.tex}{}{\fxfatal{File not found: data/lch-eps/inh-hcs/tbl-inh-hcs-highlights.tex}}
		\bottomrule
	\end{tabular}
\end{table}
In \vref{subsec-preparation-hcs-tol}, several strains have been highlighted for their inclusion in the top 27/28 of at least four different inhibitors. The results are summarized in \vref{tbl-inh-hcs-highlights}. In comparison to the medians of the set of all strains tested in this step, \fur{} and \hmf{} were degraded to the same extent, while \van{} was degraded to a lesser extent. \Acet{} appeared to be either degraded or produced more as evidenced by the lower median and higher upper quartile. \Fora{} was degraded completely in the majority of the strains, in two cases \fora{} was not or only slightly consumed. No \laev{} was degraded by the highlighted strains.

\subsection{\GLC{} Consumption\label{subsec-lch-eps-inh-hcs-glc-con}}
\begin{table}
	\centering
	\sisetup{
		table-number-alignment = center,
		table-text-alignment = center,
		table-figures-integer = 2,
		table-figures-decimal = 2,
		table-format = 2.2
	}
	\caption[Summary Statistics of the Residual \GLC{} After \SIh{48} Incubation]{Summary statistics of the residual \glc{} after \SIh{48} incubation. The plates ISp and ISr were incubated with \SIml{1.0} SM18 P30S with \SIgpl{2.00} of inhibitor for \SIh{48} at \SIdC{30} and \SIrpm{1000}. Afterwards, the residual \glc{} concentrations were determined using a \glc{} assay. In this table, summary statistics of each single inhibitor are given. For the calculation, negative concentrations were set to zero. The results were rounded to two decimals. The complete raw data are given in \vref{tbl-inh-hcs-glcc-isp-full} and \vref{tbl-inh-hcs-glcc-isr-full}.\label{tbl-inh-hcs-glcc-sum}}%glcc = glucose consumption
	\begin{tabular}{l*{3}{S}}
		\toprule
		 & \multicolumn{3}{c}{\GLC{} concentration in \si{\gram\per\litre} after \SIh{48}} \\
		\multirow{-2}*{Inhibitor} & {Lower quartile} & {Median} & {Upper quartile} \\
		\hline
		\TablesafeInputIfFileExists{data/lch-eps/inh-hcs/glc-consumption-stats.tex}{}{\fxfatal{File not found: data/lch-eps/inh-hcs/glc-consumption-stats.tex}}
		\bottomrule
	\end{tabular}
\end{table}
\GLC{} consumption differed between inhibitors, summary statistics are given in \vref{tbl-inh-hcs-glcc-sum}. Sorting the inhibitors from best \glc{} consumption to worst yielded the following order: \fora{}, \van{}, \acet{}, \laev{}, \hmf{}, \fur{}. Nonetheless, \glc{} was consumed in all but four cases to an extent of at least \SIpct{50}.

\paragraph{Non-Growing Strains}
The strains without visible turbidity and pellets after centrifugation most likely did not grow. Four of them were part of the \fur{} screening and in two cases no \fur{} was degraded, in one case \SIpct{30} and in another case \SIpct{50} of the \fur{} was degraded. Oddly, no \glc{} was consumed in two cases, one exhibiting \SIpct{50} \fur{} degradation. In the other two cases \van{} and \laev{} were used and no degradation was observed. Among all six cases, at most \SIpct{10} \glc{} was consumed. \Amc{} analysis yielded apparent cumulative monomer concentrations of up to \SImgpl{392}. In all cases however, \glc{} constituted the majority (at least \SIpct{50}) of the monomers. The only other monomer found was \man{}.
%No pellet after centrifugation (aka no growth probable):
%B6 (\fur{}, no degradation, no \glc{} consumption, AMC: \SImgpl{123} of which \SImgpl{114} \glc{}),
%C6 (\fur{}, no degradation, \SIpct{10} \glc{} consumption, AMC: \SImgpl{39} of which \SImgpl{29} \glc{}),
%F7 (\fur{}, \SIpct{50} degradation, no \glc{} consumption, AMC: \SImgpl{91} of which \SImgpl{82} \glc{}),
%G7 (\fur{}, \SIpct{30} degradation, \SIpct{7} \glc{\ consumption, AMC: \SImgpl{94} of which \SImgpl{94} \glc{}) of ISp and
%E11 (\van{}, no degradation, \SIpct{10} \glc{} consumption, AMC: \SImgpl{392} of which \SImgpl{383} \glc{}),
%B12 (\laev{}, no degradation (improbable anyway), \SIpct{10} \glc{} consumption, AMC: \SImgpl{118} of which \SImgpl{107} \glc{}) of ISr.
%Remaining monomer content exclusively \man{}.

\subsection{\AMC{}\label{subsec-lch-eps-inh-hcs-eps-amc}}
\begin{table}
	\centering
	\sisetup{
		table-number-alignment = center,
		table-text-alignment = center,
		table-figures-integer = 3,
		table-figures-decimal = 0,
		table-format = 3.0
	}
	\caption[Summary Statistics of the Monomer Concentration After \SIh{48} Incubation]{Summary statistics of the monomer concentration after \SIh{48} incubation. The plates ISp and ISr were incubated with \SIml{1.0} SM18 P30S with \SIgpl{2.00} of inhibitor for \SIh{48} at \SIdC{30} and \SIrpm{1000}. Afterwards, the \eps{} \amc{} was determined. In this table, summary statistics for each single inhibitor are given. The results were rounded to two decimals. The complete raw data are given in \vref{tbl-inh-hcs-monomers}.\label{tbl-inh-hcs-eps-sum}}
	\begin{tabular}{l*{3}{S}}
		\toprule
		 & \multicolumn{3}{c}{Monomer concentration in \si{\milli\gram\per\litre} after \SIh{48}} \\
		\multirow{-2}*{Inhibitor} & {Lower quartile} & {Median} & {Upper quartile} \\
		\hline
		\TablesafeInputIfFileExists{data/lch-eps/inh-hcs/monomers-stats.tex}{}{\fxfatal{File not found: data/lch-eps/inh-hcs/monomers-stats.tex}}
		\bottomrule
	\end{tabular}
\end{table}
%Auffällige Glucosemonomerwerte (n. Hydrolyse weniger als vorher):
%ISp: A4, A6, A10, A12, C6, G2
%ISr: B12, C8, F10
%
%Übereinstimmung mit "kein Wachstum":
%ISp: C6
%ISr: B12
%
%Übereinstimmung mit "kein Inhibitorabbau":
%ISp: A12, G2
%ISr: F10
The \glc{} which passed through 96-well gel filtration was quantified using the \glc{} assay. Another \glc{} assay after neutralization was used to assess the \glc{} freed during hydrolysis. Assuming that no \glc{} was lost during hydrolysis, the difference of the \glc{} concentration after neutralization and after gel filtration should be greater than or equal to zero. This was not the case for nine wells\footnote{Less \glc{} after hydrolysis: A4, A6, A10, A12, C6 and G2 of ISp and B12, C8 and F10 of ISr. This corresponds to the strains \xyli{F1} and \xyli{F3} for \fur{}, \xyli{D12} and \xylj{B7} for \hmf{}, \xyli{F10} and \xylj{C5} for \van{}, \xylj{A12} for \acet{} and \xyli{D12} and \xylj{A6} for \laev{}.}. % This footnote's contents must be kept in sync with subsubsec-lch-eps-discussion-inh-hcs-glc-hydrolysis.

\paragraph{Top-Performing Strains}
The summary statistics in \vref{tbl-inh-hcs-eps-sum} show that the lower quartiles and medians of the cumulative monomer concentrations are in a narrow range of \SIrange{25}{104}{\milli\gram\per\litre} and \SIrange{48}{169}{\milli\gram\per\litre}, respectively, but the difference in the upper quartiles covered almost one order of magnitude: \SIrange{117}{823}{\milli\gram\per\litre}.

\label{intext-lch-eps-inh-hcs-top-strains}Sixteen strains\footnote{Cumulative monomer concentrations greater than \SIgpl{1.0}: none in the presence of \fur{},
\xyli{C4} and \xyli{C5} in the presence of \hmf{},
none in the presence of \van{},
\xyli{C4}, \xyli{C5}, \xyli{H8}, \xylj{A5}, \xylj{A7}, \xylj{A8} in the presence of \acet{},
\xyli{C4}, \xyli{C5}, \xyli{F5}, \xylj{A1}, \xylj{A2}, \xylj{B10} and \xylj{D3} in the presence of \fora{} and
\xylj{A7} in the presence of \laev{}.} % This footnote's contents must be kept in sync with subsubsec-lch-eps-discussion-inh-hcs-inh-vs-eps-p.
exhibited cumulative monomer concentrations exceeding \SIgpl{1.0}. The two strains per inhibitor with the highest cumulative monomer concentrations were \xylj{A2} (\SImgpl{352}) and \xyli{G5} (\SImgpl{253}) for \fur{}, \xyli{C5} (\SImgpl{1079}) and \xyli{C4} (\SImgpl{1053}) for \hmf{}, \xyli{C5} (\SImgpl{449}) and \xyli{C4} (\SImgpl{448}) for \van{}, \xylj{A8} (\SImgpl{2055}) and \xyli{C5} (\SImgpl{1875}) for \acet{}, \xyli{C4} (\SImgpl{1930}) and \xyli{C5} (\SImgpl{1807}) for \fora{} and \xylj{A7} (\SImgpl{1009}) and \xyli{A9} (\SImgpl{881}) for \laev{}. Altogether, this makes for seven different strains; \xyli{C5} appears four times and \xyli{C4} appears three times. Since the \eps{} \amc{}s of \xyli{C4} and \xyli{C5} appear to be highly similar and \xyli{C5} exhibited the higher cumulative \amc{} of the two strains, \xyli{C5} was chosen as an example for the comparison of the \eps{} \amc{}s with and without inhibitors in \vref{fig-inh-hcs-comp}.

\paragraph{Comparison with Compositions on \GLC{} without Inhibitors}
\begin{figure}
	\begin{center}
		\includegraphics[width=\textwidth]{fig/inh-hcs_comp_600dpi.png}
		\caption[\AMC{}s of \xyli{C5} With and Without Inhibitors]{\Amc{}s of \xyli{C5} with and without inhibitors. In order to reliably assess the \amc{}s, the cumulative \amc{} threshold was \SImgpl{900}. Results with lower cumulative \amc{}s were not used. Compositional data of inhibitor-free growth was adapted from \textcite{Ruehmann2015b}. Abbreviations: none: no inhibitor, data by \textcite{Ruehmann2015b}; HMF: \hmf{}; Acet.: \acet{}; Form.: \fora{}; Gal:~\gal{}; Glc:~\glc{}; GlcUA:~\glcua{}; Rha:~\rha{};.\label{fig-inh-hcs-comp}}
	\end{center}
\end{figure}
The \amc{}s of \xyli{C5} grown in inhibitor presence are shown alongside data published by \textcite{Ruehmann2015b} without inhibitor in \vref{fig-inh-hcs-comp}. In all cases, \glc{} and \rha{} made up the majority of the polymer with around \SIpct{90}, while \glcua{} and \gal{} made up \SIpct{6} and \SIpct{2}, respectively, on average.

