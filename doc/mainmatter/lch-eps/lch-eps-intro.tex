\chapter[From Lignocellulose Hydrolysate to \EPS{}s]{Bacterial Conversion of Lignocellulose Hydrolysate to \EPS{}s\label{chap-lch-eps}}
\label{intext-lch-eps-overview}In a multi-step process, bacterial \eps{} producers were screened singling out the most promising strains for conversion of \lch{} to \eps{} in a fermentation.

The strains of the \eps{} producers collection (plates \enquote{EPS1} and \enquote{EPS2}) were subjected to a growth screening on \xyl{} in \vref{sec-xyl-growth}. Strains growing on \xyl{} were taken to the next round in newly prepared plates: \enquote{Xyl1} and \enquote{Xyl2}.

The strains on Xyl1 were grown in the presence of \xyl{} again in \vref{sec-xyl-hcs}. \XYL{} consumption, \eps{} concentrations and the \eps{} \amc{}s were quantified using PMP derivatization and HPLC-MS analysis. \EPS{} \amc{} data were compared with those published by \textcite{Ruehmann2015b}.

During \lch{} preparation, numerous substances which can act as inhibitors to microbial growth might form. The tolerance towards six of these inhibitors---\fur{}, \hmf{}, \van{}, \acet{}, \fora{} and \laev{}---was examined in \vref{sec-inh-hts}. Plates Xyl1 and Xyl2 were subjected to a growth screening: the growth in the presence of inhibitors was compared to the growth of a reference without inhibitors. The best-growing strains were transferred to new plates: \enquote{ISp} for \fur{}, \hmf{} and \van{}; \enquote{ISr} for \acet{}, \fora{} and \laev{}.

Industrially produced \lch{} was used in a growth screening of the plates Xyl1 and Xyl2 to quantify how well the results from single inhibitor studies match the results with the real, far more complex, substrate in \vref{sec-lch-hts}.

In \vref{sec-inh-hcs}, the plates ISp and ISr were grown with the respective inhibitors in each well. \GLC{} consumption, inhibitor degradation, \eps{} production and \amc{}s were analysed and used as a basis for the strain selection, described in \vref{sec-lch-eps-strain-selection}.

Parallel fermentations of the strain \strain{} at \SIml{500} scale were carried out to get reliable growth data in \vref{sec-lch-pf}. Four fermenters were set up with a medium containing \glc{} and \xyl{} at the same concentrations as a \SIpct{30} solution of \lch{}; another four fermenters were set up using a medium with \lch{} as carbon source. The data at \SIml{500} scale were used to run a \SIl{7} fermentation with an improved fermentation strategy.

