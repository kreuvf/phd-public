\cleardoublepage
\phantomsection
\pagestyle{plain}
\addcontentsline{toc}{chapter}{Abstract of \enquote{Bacterial Conversion of \LCH{} to \EPS{}s}}
\chaptermark{Abstract of \enquote{Bacterial Conversion of \LCH{} to \EPS{}s}}
\section*{Abstract of \enquote{Bacterial Conversion of \LCH{} to \EPS{}s}}
A bank of \eps{}-producing microorganisms was successfully subjected to a multi-step screening process to single out strains with robust growth and \eps{} production in the presence of the commonly encountered growth inhibitors in \lch{}: \fur{}, \hmf{}, \van{}, \fora{}, \acet{} and \laev{}. The best strains were tested in more detail regarding growth and production and the strain \strain{} was selected for comparative fermentations at the \SIml{500} scale. The \fur{} concentration correlated with the onset of the exponential phase and from the insights of the small-scale fermentations an improved fermentation strategy for the \SIl{7} scale was derived.

\paragraph{\XYL{} Screening}
191 \eps{}-producing bacteria were tested for their \xyl{} utilization capabilities and 135 (\SIpct{71}) were found to grow well with \xyl{} as the only carbon source. The good-growing strains were transferred to new approximately one and a third 96-well plates and used in the next screening steps.

\paragraph{High-Content Screening with \XYL{}}
\nomenclature[latabbr_et_al]{\textit{et al.}}{\textit{et alia}}
95 \xyl{}-consuming strains were grown on a \xyl{}-rich medium for \SIh{48} and \xyl{} consumption, \eps{} \amc{} and production were assessed. 66 strains (\SIpct{69}) consumed at least half the \xyl{}. 13 strains (\SIpct{14}) were found to produce at least \SImgpl{560} of \eps{}. A comparison of the \eps{} \amc{} with data from \textcite{Ruehmann2015a} showed major and minor differences in two strains each. The other strains were not affected. It is hypothesized that apparent \eps{} compositional changes were not caused by a changed composition of a single \eps{}, but rather the production of different \eps{}s at different levels.

\paragraph{High-Throughput Screening for Inhibitor Tolerance}
The 135 \xyl{}-utilizing strains were screened in the presence of one and only one \lch{} inhibitor at \SIgpl{2.0} to remove non-growing strains from the respective high-content screening (see next step). \VAN{} was by far the strongest inhibitor shutting down microbial growth in \SIpct{79} of the strains. \FUR{} and \hmf{} totally inhibited \SIpct{20} and \SIpct{7} of the strains, respectively. Acids did not appreciably inhibit microbial growth. The comparison to \lch{} showed that using just the 27 and 28, respectively, best strains of \van{} and \acet{} sufficed to predict \SIpct{57} of the top 28 strains of \lch{}.

\paragraph{High-Content Screening with Inhibitors}
27 or 28 of the best growing strains per inhibitor were screened to assess inhibitor degradation, \eps{} \amc{} and production. Except for \laev{}, all inhibitors were consumed to varying degrees: \fur{}, \hmf{} and \fora{} were degraded completely after \SIh{48} in the majority of strains. \Acet{} was degraded in some, produced in other strains, while \van{}'s inhibitory potential led to decreased degradation. \Laev{} remained untouched in most cases (>\,\SIpct{75}). Changes in the exopolysaccharide compositions based on the inhibitor were not observed.

\paragraph{Strain Selection}
\nomenclature[latabbr_rDNA]{rDNA}{ribosomal DNA}
\nomenclature[latabbr_DNA]{DNA}{desoxyribonucleic acid}
Among the seven final candidate strains selected based on the previous results, the strain \strain{} exhibited the most robust growth and also the most robust \eps{} production in the presence of different inhibitors and was selected for the fermentations. Based on 16S rDNA sequencing, the closest relatives were determined to be different unclassified \mo{Paenibacillus} spp., \mo{P. cineris} and \mo{P. favisporus}.

\paragraph{Parallel Fermentation with \LCH{}}
\nomenclature[formula_pH]{pH}{power of hydrogen}
\strain{} was used in 2 x 4 \SIml{500} parallel fermentations in \lch{} and a reference with pure \glc{} and \xyl{}. The onset of the exponential growth phase in the \lch{} fermenters was correlated with the \fur{} concentration and lagged approximately \SIh{36} behind the reference fermentations. The \eps{} molar masses were around \SI{1E7}{\gram\per\mole} and unaffected by the hydrolysate as long as the pH value was controlled at 7.0. The fermentation courses of all eight fermenters were comparable to each other from the onset of the exponential growth phase on. The results at the \SIml{500} scale were used to devise improvements to the fermentation parameters at the \SIl{7} scale: lower initial \lch{} concentration, fed-batch instead of batch and higher initial bacteria concentration.

\addcontentsline{toc}{chapter}{Abstract of \enquote{Fermentative Production of Scleroglucan and Schizophyllan}}
\chaptermark{Abstract of \enquote{Fermentative Production of Scleroglucan and Schizophyllan}}
\section*{Abstract of \enquote{Fermentative Production of Scleroglucan and Schizophyllan}}
% Sorting affected when using \mo{S. rolfsii} ... :(
\nomenclature[mo_S. commune]{\mo{S. commune}}{\mo{Schizophyllum commune}}
\nomenclature[mo_S. rolfsii]{\mo{S. rolfsii}}{\mo{Sclerotium rolfsii}}
In order to examine the differences between \scl{} and \shz{}, eight fermentations of \rolf{} and \comm{} at the \SIml{500} scale from \SIh{48} to \SIh{144} were conducted and the \eps{}s purified. The \eps{} concentrations at the end of the fermentations increased over time to \SIgpl{3.2} and \SIgpl{1.4} for \scl{} and \shz{}, respectively, with one exception: \comm{} produced \SIgpl{2.2} \shz{} after \SIh{120}.

The \eps{}s exhibited poor solubility which hampered all further analyses. In addition, precipitations of samples showed a sharp drop in the \eps{} concentrations after the \SIh{24} sample due to an adaptation of the protocol to thicker fermentation broths: the samples were diluted 1:10 with ultra-pure water. All other tests---dynamic viscosity and thixotropy, molar mass, periodate test and metabolite analyses---were without clear and reliable results. Suggestions for repetitions of the experiments are discussed and the development of a quantitative fluorometric assay for the β-1,3-β-1,6-glucans \scl{} and \shz{} is briefly covered.
\clearpage

\phantomsection
\setquotestyle[quotes]{german}
\selectlanguage{ngerman}
\addcontentsline{toc}{chapter}{Zusammenfassung von \enquote{Bakterielle Umwandlung von Lignocellulosehydrolysat zu Exopolysacchariden}}
\chaptermark{Zusammenfassung von \enquote{Bakterielle Umwandlung von Lignocellulosehydrolysat zu Exopolysacchariden}}
\section*{Zusammenfassung von \enquote{Bakterielle Umwandlung von Lignocellulosehydrolysat zu Exopolysacchariden}}
Eine Mikroorganismenbank mit Exopolysaccharidproduzenten wurde erfolgreich einem mehr\-schrit\-tigen Screeningverfahren unterzogen, um jene Stämme zu finden, die robustes Wachstum und robuste Exopolysaccharidproduktion in Gegenwart von Wachstumsinhibitoren aufweisen, die für gewöhnlich in Lignocellulosehydrolysaten vorkommen: Furfural, Hydroxymethylfurfural, Vanillin, Ameisensäure, Essigsäure und Lävulinsäure. Die besten Stämme wurden in Bezug auf Wachstum und Produktion detaillierter untersucht und der Stamm \strain{} wurde für vergleichende Fermentationen im 500-ml-Maßstab ausgewählt. Die Furfuralkonzentration korrelierte mit dem Einsetzen der exponentiellen Wachstumsphase und aus den Erkenntnissen dieser Fermentationen im kleinen Maßstab wurde eine verbesserte Fermentationsstrategie für den 7-l-Maßstab abgeleitet.

\paragraph{\XYL{}-Screening}
191 Exopolysaccharidproduzenten wurden hinsichtlich ihrer \XYL{}-Verwertungsfähigkeit untersucht und 135 (\SIpct{71}) zeigten gutes Wachstum mit \XYL{} als einziger Kohlenstoffquelle. Die gutwachsenden Stämme wurden auf etwa eineindrittel 96-Well-Platten überführt und in den nächsten Screeningrunden benutzt.

\paragraph{High Content Screening mit \XYL{}}
95 \XYL{}verwerter wurden \SIh{48} in \xyl{}haltigem Medium inkubiert und der \XYL{}verbrauch, die Exopoly\-saccharid-Aldose\-monomer\-zusammen\-setzung und -produktion untersucht. 66 Stämme (\SIpct{69}) verbrauchten mindestens die Hälfte an \XYL{}. 13 Stämme (\SIpct{14}) produzierten mindestens \SImgpl{560} Exopolysaccharid. Ein Vergleich der Exopoly\-saccharid-Aldose\-monomer\-zusammen\-setzungen mit Daten von \textcite{Ruehmann2015a} zeigte größere und kleinere Unterschiede in jeweils zweien der Stämme. Die übrigen Stämme waren nicht betroffen. Es wird die Hypothese aufgestellt, dass die augenscheinlichen Veränderungen der Exopolysaccharidzusammensetzungen nicht durch eine Veränderung der Zusammensetzung eines Exopolysaccharids verursacht wurden, sondern durch die unterschiedlich starke Produktion verschiedener Exopolysaccharide.

\paragraph{Hochdurchsatzscreening für die Inhibitortoleranz}
Die 135 \XYL{}verwerter wurden in Anwesenheit von \SIgpl{2.0} genau eines Lignocellulosehydrolysatinhibitors untersucht, um nichtwachsende Stämme aus den jeweiligen High Content Screenings zu entfernen (siehe nächster Schritt). \VAN{} war der mit Abstand stärkste Inhibitor und stoppte der Wachstum von \SIpct{79} der Stämme vollständig. \FUR{} und Hydroxymethylfurfural hemmten das Wachstum von \SIpct{20} beziehungsweise \SIpct{7} der Stämme vollständig. Die Säuren hemmten das mikrobielle Wachstum nicht nennenswert. Der Vergleich mit Lignocellulosehydrolysat zeigte, dass die Verwendung von nur 27 beziehungsweise 28 der besten Vanillin- und Essigsäurestämme bereits ausreichte, um \SIpct{57} der besten 28 Lignocellulosehydrolysatstämme vorherzusagen.

\paragraph{High Content Screening mit Inhibitoren}
27 oder 28 der am besten wachsenden Stämme pro Inhibitor wurden untersucht, um den Inhibitorabbau, die Exopoly\-saccharid-Aldose\-monomer\-zusammen\-setzung und -produktion zu bewerten. Bis auf Lävulinsäure wurden sämtliche Inhibitoren in unterschiedlichem Ausmaße abgebaut: Furfural, Hydroxymethylfurfural und Ameisensäure wurden im Gros der Stämme nach \SIh{48} vollständig abgebaut. Essigsäure wurde in einigen Stämmen abgebaut, in anderen aufgebaut, während Vanillins inhibitorisches Potential zu einem verringerten Abbau führte. Lävulinsäure blieb in den meisten Fällen unberührt (> \SIpct{75}). Es wurden keine Veränderungen der Exopolysaccharidzusammensetzungen in Abhängigkeit vom Inhibitor  festgestellt.

\paragraph{Stammauswahl}
Von den sieben finalen Stammkandidaten, die aus den Voruntersuchungen ausgewählt wurden, zeigte \strain{} das robusteste Wachstum und auch die robusteste Exopolysaccharidproduktion in Gegenwart verschiedener Inhibitoren und wurde daher für die Fermentationen ausgewählt. Basierend auf einer 16S-rDNA-Sequenzierung wurden unklassifizierte Paenibacillen, \mo{P. cineris} und \mo{P. favisporus} als nächste Verwandte identifiziert.

\paragraph{Parallelfermentation mit Lignocellulosehydrolysat}
\strain{} wurde in 2 x 4 500-ml-Parallelfermentationen in Lignocellulosehydrolysat und einer Referenz mit reiner \GLC{} und reiner \XYL{} eingesetzt. Der Beginn der exponentiellen Wachstumsphase in den Lignocellulosehydrolysatfermentern korrelierte mit der Furfuralkonzentration und war gegenüber den Referenzfermentationen etwa \SIh{36} verzögert. Die molaren Massen der Exopolysaccharide lagen bei etwa \SI{1E7}{\gram\per\mole} und waren vom Lignocellulosehydrolysat unbeeinflusst, solange der pH-Wert auf 7,0 geregelt war. Die Fermentationsverläufe aller acht Fermenter waren ab dem Beginn der exponentiellen Phase untereinander vergleichbar. Die Ergebnisse im 500-ml-Maßstab wurden benutzt, um Verbesserungen an den Fermentationsparametern für den 7-l-Maßstab vorzunehmen: geringere anfängliche Lignocellulosehydrolysatkonzentration, Fed-Batch-Verfahren statt Batchverfahren und eine höhere anfängliche Bakterienkonzentration.

\addcontentsline{toc}{chapter}{Zusammenfassung von \enquote{Fermentative Herstellung von Scleroglucan und Schizophyllan}}
\chaptermark{Zusammenfassung von \enquote{Fermentative Herstellung von Scleroglucan und Schizophyllan}}
\section*{Zusammenfassung von \enquote{Fermentative Herstellung von Scleroglucan und Schizophyllan}}
Um die Unterschiede zwischen \SCL{} und \SHZ{} herauszuarbeiten, wurden acht Fermentationen mit \rolf{} und \comm{} im 500-ml-Maßstab über \SIh{48} bis \SIh{144} durchgeführt und die Exopolysaccharide aufgereinigt. Die Exopolysaccharidkonzentrationen am Ende der Fermentationen stiegen über die Zeit auf 3,2~\si{\gram\per\litre} und 1,4~\si{\gram\per\litre} für \SCL{} beziehungsweise \SHZ{} mit einer Ausnahme: \comm{} produzierte nach \SIh{120} 2,2~\si{\gram\per\litre} \SHZ{}.

Die Exopolysaccharide zeigten eine schlechte Löslichkeit, was sämtliche weiteren Analysen behinderte. Zusätzlich fielen die Exopolysaccharidkonzentrationen aus den Probenfällungen nach der 24-h-Probe stark ab, was an der Anpassung des Protokolls für dickflüssigere Fermentationsbrühen lag: die Proben wurden 1:10 mit hochreinem Wasser verdünnt. Alle anderen Untersuchungen~--~dynamische Viskosität und Thixotropie, molare Masse, Periodattest und Metabolitanalysen~--~blieben ohne klare und verlässliche Ergebnisse. Vorschläge für Wiederholungen der Experimente werden diskutiert und auf die Entwicklung eines quantitativen fluorometrischen Nachweises der β-1,3-β-1,6-Glucane \SCL{} und \SHZ{} wird kurz eingegangen.
\cleardoublepage
\setquotestyle[british]{english}
\selectlanguage{british}
\pagestyle{headings}

